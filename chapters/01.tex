\chapter{Introduction}\label{ch:introduction}

Since the early days of formal semantics \citep[starting with][]{montague1973proper} issues concerning quantification, pluralities, and countability have been of central importance for the study of meaning in natural language. A substantial body of research has been dedicated to these topics, which led to a number of influential theories \citep[e.g.,][]{barwise_cooper1981generalized,scha1981distributive,link1983logical,hoeksema1983plurality,gillon1987readings,krifka1989nominal,schwarzschild1996pluralities,landman2000events,winter2001flexibility}. Within this tradition since at least \citet{scha1981distributive} to present, e.g., \citet{champollion2017parts}, the prevailing view is that denotations of count nouns are atomic or, in other words, involve atoms, i.e., entities that have no proper parts in a mereological sense. Though at first blush this seems unintuitive since we know very well from our common experience that often things do have parts, there seemed to be good reasons to assume that compositional semantics ignores this fact and treats referents of count noun as indivisible units. The achievements of that strand of research are indisputable. At the same time, I believe that the attachment to the notion of atomicity originates from a limited scope of investigation and in the face of novel linguistic evidence to be presented needs to revised. The motivation behind this book is to explore what could be gained for semantic theory if we adopted a different perspective.

\section{What's this all about}\label{sec:whats-this-all-about}

This book is about parts of singular concrete things such as apples, walls, and crowns, and how we quantify over them. It is also about parts of pluralities of things such as collections of apples, walls, and crowns, and why we do not quantify over them unless very specific individuating criteria are satisfied. It is about various types of partitives, e.g., structures such as \textit{half of the apple}, multiplier phrases, e.g., expressions like \textit{double crown}, and other natural language expressions, e.g., adjective phrases such as \textit{whole apple}. What all these constructions have in common is that they involve what I will call throughout this study \textsc{subatomic quantification}. Subatomic quantification is quantification over parts of things that constitute building blocks of denotations of nominal expressions such as \textit{apple}, \textit{wall}, and \textit{crown}. Though the use of the word ``subatomic'' in the title of this book suggests that I will also adopt the concept of an atom, in fact I will argue against approaches to nominal semantics based on the notion of atomicity. I decided for the use of that term since I believe that it intuitively evokes what I intend to focus on. Ironically, however, I believe the evidence concerning subatomic quantification forces us to reconsider what building blocks of denotations of count nouns actually are. The great source of inspiration for this study has been the work of \citet{grimm2012degrees,grimm2012number} who proposes a novel view on how to capture the semantic relevance of the mass/count distinction without reference to atoms. 

The empirical aim of this book is to provide novel linguistic evidence showing that natural language semantics is sensitive to subatomic part-whole structures and to topological relations holding between elements within such structures. Furthermore, I intend to demonstrate the linguistic relevance of subatomic quantification as well as the fact that quantification over parts of a whole is subject to identical restrictions as quantification over wholes. The evidence to be examined indicates that some quantificational operations including counting presuppose particular topological relations.

From a theoretical point of view, this book is intended to provide a novel argument for adopting a theory of wholes called mereotopology, i.e., a system in which standard mereology based on parthood is extended with the primitive relation of connectedness as well as nuanced derived topological notions that allow for capturing various spatial configurations of parts. The origins of mereotopology trace back to \citet{whitehead1920concept,whitehead1929process} and the theory was further developed within formal ontology and artificial intelligence \citep[e.g.,][]{clarke1981calculus,smith1996mereotopology,roeper1997region,casati_varzi1999parts,varzi2007spatial}. It was introduced to linguistics by \citet{grimm2012degrees,grimm2012number}.

Some of the key questions concerning natural language semantics I want to address in this book are the following.

\begin{enumerate}
\item[Q1:] Are there different types of parts of entities similar to different types of wholes?
\item[Q2:] How do we count parts of an object?
\item[Q3:] Why can we not count parts of a plurality as one thing unless certain individuating criteria are satisfied?
\end{enumerate}

\noindent At first blush, the answers at which I arrive might seem somewhat surprising. 

\begin{enumerate}
\item[A1:] Yes, there are continuous and discontinuous parts. The former resemble referents of singular count nouns, whereas the latter are more like pluralities.
\item[A2:] We count parts of an object in the very same way to how we count whole objects. The mechanism is unified, thus identical restrictions apply. 
\item[A3:] For the same reason we cannot count pluralities as one thing unless specific individuating criteria are met. Only objects with certain topological characteristics can be assigned a number in numerical quantification.
\end{enumerate}

The most important result of this study is a proposal of an account for subatomic quantification. I will argue not only that it correctly predicts why only some parts are countable, but also that extending it to the level of wholes will enable us to provide more advantageous explanations for known problems. For instance, it could explain why a natural language expression such as \textit{two apples} refers to a plurality of two apples rather than to two pluralities of apples. Or why object mass nouns such as \textit{furniture} are not countable despite involving reference to discrete entities. There are also a number of other questions I will attempt to tackle, many of which will remain without a definite answer. However, I believe that both the evidence presented here and the proposed account provide a novel exciting perspective to think about partitivity and countability. 

Note that though I will sometimes mention certain nominal categories in passing, this study is not about abstract terms \citep[see, e.g.,][]{asher1993reference,tovena2001between,nicolas2002mass,moltmann2013abstract}, event nominals \citep[see, e.g.,][]{grimshaw1990argument,grimshaw2011deverbal,bierwisch1990event,borer2005normal}, or collective nouns \citep[see, e.g.,][]{landman1989groupsi,landman1989groupsii,barker1992group,schwarzschild1996pluralities,pearson2011new,de_vries2015shifting}, nor about parts of their referents. Despite the fact that in principle I believe that the approach developed here can be extended to these expressions, for the most part I will not consider part-whole structures of abstract entities. Before I introduce the main claims of this book in a bit more detail and then delve into intricacies of linguistic evidence, let us first consider my view on what it intuitively means to be a whole and part.

\subsection{Intuitive notions of a whole and part}\label{sec:intuitive-notions-of-a-whole-and-part}

	The starting point of this journey concerns an ontological intuition dating back at least to the Pre-Socratics that entities are often made up of smaller entities which are related to each other in a particular manner (see \citealt{varzi2016mereology} for a historical overview). In other words, when thinking about what is, humans often assume that objects are generally configurations of parts. This ontological intuition seems to stem from a cognitive fact that human beings often conceive of entities as being made up of smaller entities related to each other in certain ways. In fact, psychological evidence demonstrates that at least from early childhood humans are able to perceive entities simultaneously in a twofold way \citep[e.g.,][]{elkind_koegler_go1964studies,kimchi1993basic,boisvert_standing_moller1999successful}. On the one hand, we can discriminate parts from a whole making them more salient than the entire object, whereas on the other hand we have the ability to integrate the parts in such a way that the perception of a complete whole emerges in our cognition. In other words, humans are able to see entities as collections of parts and as integrated wholes simultaneously. 
    
    At the same time, since the very early considerations on the part-whole relation in Plato's \textit{Parmenides} and \textit{Theaetetus} a topic of much concern has been unity. What interests Plato is how to differentiate between a true unity and an arbitrary sum of parts. The crucial property of the former is that it is provided by structure which distinguishes it from the latter. Without going into details of Plato's ontological views \citep[but see, e.g.,][]{priest2014one}, it seems that the key intuition behind his deliberations is that the crucial component of what it means to be a whole is given by the manner of arrangement of parts. In other words, a whole is not simply reducible to the sum of its parts \citep[see also][]{casati_varzi1999parts}, e.g., a glass is not an arbitrary sum of shards, but rather a particular configuration of shards. It happens that I also share this intuition.

When considering how objects are conceptualized, I believe it is instructive to make use of categories such as unity, boundedness, and integrity. The first is about capturing an individual as a complete whole in itself, i.e., distinguishing between discrete objects as opposed to fragmentary portions of a continuum of a larger entity. The second category concerns boundaries. Objects are well-defined in space in contrast to scattered fragments of entities or a diffused continuum of matter. Finally, integrity characterizes a mutual bond between entities perceived as being parts of a certain whole. Though this bond can be conceptualized in many different ways, one significant mode of integrity is viewed in terms of topology. Specifically, objects involve elements that are connected to each other, i.e., stuck together, and thus move along the same trajectories as one unit. The vital question that inspired this study is to what extent these categories are relevant for the architecture of natural language semantics? In the following chapters, I will argue that there are good reasons to believe that it is indeed relevant. Specifically, I will provide linguistic evidence not only that the meaning of nominal expressions involves information concerning integrity but also that grammar is sensitive to such a notion.

Crucially, I argue that the same way of thinking should apply when talking about parts of wholes. When describing parthood with respect to entities extended in space or time, employing the topological notion of contiguity might allow us to capture some non-trivial facts about part-whole structures. In particular, some parts constitute individuated continuous strings of matter within a whole, whereas other parts are just arbitrary sums of portions of substances. To demonstrate the difference let us discuss the following example \citep[see][pp. 90--93]{acquaviva2008lexical}. Consider two parts of a table, its leg and a splinter. Under ordinary circumstances, we would definitely agree that both these entities are parts of the table; however, each of them in a somewhat different sense. A splinter is just some arbitrary portion of matter making up the table, an element among numerous similar fragments of the whole. Before mentioning it, we probably had no or at least very little expectations with respect to its appearance. We might even question whether it exists as a splinter before being detached from the whole since normally tables are not perceived as collections of splinters. On the other hand, a leg is easily identifiable among a definite number of similar elements. We perceive it as having a clearly specified function and expect certain traits of its appearance. Furthermore, there is also a sense in which two separate disconnected splinters or two disjoint legs are part of the table since they constitute some fragment of its material make-up. Importantly, however, this type of parthood seems to resemble the relationship between a splinter and the whole rather than the relationship between a leg and the whole table. In other words, under ordinary circumstances a plurality of disjoint legs seems to be a material rather than a functional part of the table despite the fact that each of the legs constituting that plurality definitely is its functional part.

In the philosophical literature, this distinction has been attributed to different modes of partitioning an object. For instance, \citet{krecz1986parts} proposes different terms for specific subdivisions of a whole, i.e., parts, as opposed to arbitrary subdivisions, i.e., pieces. The former would correspond, e.g., to a leg of a table, whereas the latter would be used with respect to, e.g., a splinter. In a similar vein, \citet{markosian1998brutal} draws an analogous distinction between what he calls metaphysical parts, i.e., parts that can be considered as objects in their own right, and conceptual parts, i.e., equivalents of arbitrary parts. Finally, \citet{jennings2010against} proposes to differentiate between parts viewed as arbitrary bits of an object one could cut off, and functional parts defined as parts that are considered essential for an object since they fulfill a particular function. Crucially, the distinction is conceptually valid but it is also reflected in grammar. For instance, English distinguishes between the two different flavors of parthood discussed above syntactically as can be witnessed in \ref{ex:part-unstructured-structured-mass-count} \citep[see also][]{champollion_krifka2016mereology}. The use of the bare mass noun \textit{part} in the sentence in \ref{ex:part-unstructured-mass} indicates arbitrary partitioning of the table. On the other hand, the full DP \textit{a part} in \ref{ex:part-structured-count} corresponds to a specific division in parts. Notice also that only the latter use is countable.

\ex. English \citep[p. 90, adapted]{acquaviva2008lexical}\label{ex:part-unstructured-structured-mass-count}
\a. A splinter is part of the table.\label{ex:part-unstructured-mass}
\b. A leg is a part of the table.\label{ex:part-structured-count} 

It has been acknowledged that the distinction discussed above seems to have some further linguistic implications \citep{champollion_krifka2016mereology}. For instance, consider the minimal pairs in \ref{ex:structured-parts-transitivity} and \ref{ex:structured-parts-intransitivity}.\footnote{In the original examples in \ref{ex:structured-parts-transitivity-hand} and \ref{ex:structured-parts-intransitivity-arm}, \citeauthor{champollion_krifka2016mereology} use the bare count NP \textit{thumb} inside the PP, which some speakers find marked. However, using an indefinite does not seem to change the felicity judgments.} The contrast suggests that transitivity does not hold for specific parts.\footnote{\citeauthor{champollion_krifka2016mereology} use the terms structured and unstructured parthood in order to refer specific and arbitrary subdivisions, respectively.} In other words, the fact that the thumb is a part of a hand does not guarantee that it would be linguistically treated as a part of what that hand is part of, i.e., an arm.   

\ex. English \citep[p. 513; adapted]{champollion_krifka2016mereology}\label{ex:structured-parts-transitivity}
\a. the thumb of this hand\label{ex:structured-parts-transitivity-thumb}
\b. a hand without a thumb\label{ex:structured-parts-transitivity-hand}

\ex. English \citep[p. 513; adapted]{champollion_krifka2016mereology}\label{ex:structured-parts-intransitivity}
\a. \#the thumb of this arm\label{ex:structured-parts-intransitivity-thumb}
\b. \#an arm without a thumb\label{ex:structured-parts-intransitivity-arm}

Despite these facts the role of the distinction between arbitrary and specific parts in natural language semantics is usually either totally neglected or reduced to lexical relations such as meronymy, i.e., relations between meronyms and holonyms, e.g. \textit{wing} and \textit{bird}, and hyponymy, i.e., the relation between hyponyms and hyperonyms, e.g., \textit{sparrow} and \textit{bird}. Although it has been recognized that meronymy and hyponymy have some general properties that establish taxonomies and structure large parts of the lexicon in natural language \citep{cruse1986lexical}, compositional approaches to meaning did not assume that such categories play any role in the proposed models. One of the aims of this book is to provide linguistic evidence for the relevance of the two aspects of what it means to be part of something and to propose how first steps towards accommodating a less naive view on parthood into semantic theory could look like.

\subsection{The claims in a nutshell}\label{sec:the-claims-in-a-nutshell}

In this book, I will argue that a proper treatment of countability in natural language should take into account the interaction between partitivity, topology, and quantification. In other words, I believe that exploring subatomic quantification can reveal some non-trivial phenomena that were otherwise obscure, and thus contribute to our understanding of the interaction between nominal semantics and the meaning of numerical expressions in general. The central claims of this book are the following. 

First, natural language is sensitive to the fact that entities denoted by nominal expressions have parts as well as topological relations holding between them. The relevance of part-whole structures is typically acknowledged with respect to pluralities. However, my claim is stronger. I postulate that natural language is also sensitive to subatomic part-whole structures. This is in discordance with the mainstream view that utilizes the notion of atomicity in order to capture semantic properties of count nouns. Such an approach postulates that atoms, i.e., building blocks of denotations of expressions like \textit{apple} and \textit{crown}, are entities that from a linguistic perspective are treated as having no proper parts. Consequently, countability can be accounted for in terms of quantification over atoms. Nevertheless, the linguistic evidence I will present in the following chapters will demonstrate that in fact there are a number of natural language expressions involving quantification over such subatomic parts. Crucially, based on the distribution of partitives I will argue for a unified parthood relation for both singularities and pluralities. Furthermore, I will postulate that the difference between the two is a result of different topological relations encoded in the corresponding part-whole structures. The claim is that prototypical referents of count singulars are conceptualized as integrated wholes, whereas regular plurals require entities in their denotations to comprise such cohesive objects as their parts but impose no topological constraints on a spatial configuration of those parts. As we will see from the data concerning topologically sensitive partitive expressions, the distinction between integrated wholes and scattered entities also applies at the subatomic level. Specifically, the novel evidence shows that the contrast between continuous and discontinuous parts is relevant for the interpretation of natural language expressions. Hence, the notion of topological integrity should be accommodated into semantic theory.

The second claim concerns counting. In particular, I will argue that countability is not some syntactic feature nor is it a meaning postulate on a certain lexical item, but rather it follows from what I refer to as the general counting principles. In particular, the principle of non-overlap ensures that entities one quantifies over are disjoint, i.e., they do not share a part. In other words, things can be counted once and once only. This rule excludes a possibility of counting entities involving multiple overlapping parts as denoted by mass nouns and pluralia tantum. Furthermore, the principle of maximality guarantees that when counted entities are associated with numbers in their entirety, i.e., no part is left out. This is especially relevant with respect to homogeneous entities referred to by nouns such as \textit{twig} and \textit{fence}. Given a particular counting situation, what counts as one always needs to be the maximal entity no matter how its part-whole structure is defined in that situation. Finally, the principle of integrity requires that what can be counted needs to be conceptualized as an entity that comes in one piece. This restriction rules out arbitrary sums of entities as well as discontinuous portions of substances. Consequently, counting pluralities as one thing is not allowed, unless there are certain topological conditions that apply.\footnote{Notice that group nouns are more than pluralities. In fact, I assume that they denote abstract singular entities with whom pluralities of members can be associated.} In other words, being a plurality is neither necessary nor sufficient for counting as a single entity. All things considered, only predicates denoting entities that satisfy the general counting principles can be modified by cardinals.

The third and final claim extends the general mechanism behind counting to the subatomic level. More specifically, I will postulate that the general counting principles constitute a universal set of constraints regulating what is fit for being counted and what is not, irrespective of whether it is a whole or a part. In other words, I will argue that it follows from the principles of non-overlap, maximality, and integrity that certain parts are countable whereas others are not. The evidence from partitive constructions involving cardinal numerals demonstrates that it is not allowed to count discontinuous parts of entities. Moreover, the claim is further corroborated by the fact that natural language developed expressions dedicated to subatomic quantification.

Linguistic evidence for the three claims introduced above comes from a number of natural language expressions. Specifically, I will examine different types of partitives, whole-adjectives as well as multipliers such as English \textit{double}. Some parts of this book will also investigate the data considering plurals and cardinal numerals. In order to provide a more general picture, the evidence will be examined from a cross-linguistic perspective. The languages which will be discussed mostly include Polish, German, Italian, and English, but throughout the text I will also address examples from (in alphabetical order) Basque, Bosnian/Croatian/Serbian (BCS), Brazilian Portuguese, Czech, Dutch, Finnish, French, Hebrew, Hungarian, Irish, Japanese, Lithuanian, Maltese, Mandarin, Mi'gmaq, Russian, and Yucuna.

\section{General assumptions}\label{sec:general-assumptions}

There are several issues concerning conceptual grounds that require a short commentary. In this section, I will present my view on natural language semantics and discuss the ontology assumed for the purpose of developing an account for subatomic quantification. 

\subsection{Cognitive view of meaning}\label{sec:cognitive-view-of-meaning}

The view on theory of meaning I adopt here is on a par with the position expressed by \citet{krifka1998origins} and shared, e.g., by \citet{partee2018changing} and \citet{grimm2012number} \citep[see also][]{chomsky2000new}. Though model-theoretic semantics in the tradition of Montague has commonly been assumed to necessarily endorse some form of semantic externalism \citep[see, e.g.,][]{putnam1975meaning,davidson1987knowing}, i.e., a view on which expressions of natural language designate objects in the world, and as such is incompatible with approaches to meaning that seek to develop cognitive models of the world (assuming that meanings are in the head after all), such a sharp distinction does not seem plausible. In fact, as pointed out by \citeauthor{krifka1998origins}, it is possible to combine the use of the methods and techniques developed within the model-theoretic tradition with an assumption that natural language expressions are interpreted by conceptual structures that in turn are associated with external entities by some pragmatic mechanism responsible for how we use language. Therefore, I assume that model-theoretic representations discussed in this study are compatible with a cognitive view of meaning and can be seen as tools one can use to try to grasp human mental reality. 

In other words, in the system developed here I make no metaphysical claims, nor do I embrace any form of semantic externalism. Instead, I adopt a view that natural language ontology does not reveal what there is, but rather what people talk as if there is  \citep[see][]{bach1986natural,pelletier2011descriptive,bach-chao2012metaphysics,moltmann2017natural}. Therefore, I would like to see the postulated notions and structures as attempts to capture the way we conceptualize certain aspects of external reality. That is to say, the theory of subatomic quantification to follow is assumed to characterize some properties of how human beings perceive things in the world, not to describe them how they are. This means that both primitive objects and more complex conceptual forms are assumed to ``be in the head'' \citep[see][]{chomsky2000new}. Of course, it is not unlikely that properties of such semantic objects may often correspond to the properties of mind/language-external entities. Consequently, the former could be used to talk about the latter assuming matching by some extra-linguistic mechanism. In any case, I do not presume that meanings are out there in the world. Rather, I will often emphasize the significance for natural language semantics of how entities are conceptualized. In the next section, I will describe the minimal assumptions concerning primitive objects grouped together in the domains of the model.

\subsection{Ontology}\label{sec:ontology}

In standard Montague semantics, the domain of discourse is simply a collection of disjoint non-empty sets of entities from which denotations of basic expressions as well as more complex constituents are built. It is usually assumed that the domain of entities is supplemented with the set of truth values as well as sets of events, possible worlds, degrees and others. Since the focus of this study is very specific, I will limit the number of distinct types of primitive objects almost to a minimum. In doing so, I will try to be faithful to the position famously formulated by \citet{link1983logical} that the guide in ontological considerations should be natural language itself.

I will use the term \textsc{entity} to talk about anything that can be referred to by proper names, such as \textit{Noam Chomsky} and \textit{Nim Chimpsky}, and by definite descriptions, like \textit{the author of ``Syntactic Structures''} and \textit{that chimpanzee}, as well as denoted by common nouns such as \textit{apple} and \textit{juice}. I assume that entities can be either well-defined discrete objects, pluralities thereof, or shapeless amorphous substances. Though there are significant differences between them, they all fall into one domain. In general, I will refer to a thing in the denotation of a count noun as \textsc{individual} or \textsc{object}. Sometimes, I will use the term \textsc{scattered entity} to indicate referents of mass terms and \textsc{arbitrary sum} to talk about denotations of plurals. The notions \textsc{entity} and \textsc{thing} are assumed to be general terms covering both individuals/objects, portions of matter as well as pluralities.

I do not assume that all nouns designate entities. Some classes of nominals such as nominalizations, measure words as well as role nouns arguably make reference to eventualities, degrees, and roles, respectively. For instance, \textit{murder} denotes a set of murdering events \citep[e.g.,][]{grimshaw1990argument,grimshaw2011deverbal}, \textit{liter} refers to an interval or a set of degrees on a scale of volume \citep[e.g.,][]{von_stechow1984comparing,heim1985notes}, whereas \textit{president} designates a social role, i.e., a function or capacity independent of the individuals that bear it \citep{sowa1984conceptual,steimann2000representation}. Such social constructs can be associated with individuals by a special shifting mechanism \citep{zobel2017sensitivity}. Though in this study I will restrict my focus to entities, or specifically concrete entities, I believe that at least some of the ideas introduced here can serve as an inspiration for developing a new way of thinking of partialness with respect to somewhat more abstract things such as eventualities, degrees, and roles.   

Since kinds, or more generally concepts \citep[see][]{krifka1995common,mueller-reichau2006sorting}, do not play any role in this study, I will refrain from the discussion concerning their ontological status as well as their role in nominal semantics. Nevertheless, in principle the approach developed here is compatible with accounts modeling kinds in the spirit of \citet{carlson1977reference,carlson1980reference} and \citet{krifka1995common}, where they are treated as entities in their own right that unlike object-level things are not spatio-temporally bounded.

A considerable amount of attention in this study will be dedicated to numerical expressions including cardinal numerals, fractions, and multipliers. I assume that all those lexical items involve some sort of reference to mathematical entities. I will use the term \textsc{number} to refer to abstract entities that definite descriptions such as \textit{the number two} designate. I adopt here an intuitionist perspective on the epistemology of mathematics \citep[see, e.g.,][]{kitcher1984nature}. According to this view, mathematical objects are constructions of the human mind and do not exist in the external world. Notice, however, that numbers in this sense do not necessarily coincide with objects defined by Peano's axioms. For instance, I do not assume that ``artificial'' integers \citep[in the sense of][pp. 238--242]{dehaene1997number}, i.e., non-standard and counterintuitive arithmetical objects that nevertheless satisfy Peano's axioms, are part of the ontology described here. Rather, numbers are what we refer to when we use our everyday language and what might coincide with human number sense.

\section{Conventions}\label{sec:conventions}

Both glossing and notational conventions used in this book are more or less standard in academic linguistics. Nonetheless, for the sake of clarity I provide a complete list of what to expect in examples and formulae.

\subsection{Examples}\label{sec:examples}

Since in many parts of this book I discuss phrases and sentences from various languages some of which are not very well-known, in order to avoid confusion I will always indicate the language of an example, even if that language is English. Furthermore, I will provide information on the bibliographical source or a name of a native speaker or native speakers with whom I have consulted a particular example. In the case of evidence from Polish, almost all examples and judgments are my own though I have often confirmed them with other native speakers who confirmed my intuitions. Most of the data come from introspection with some exceptions found in corpora. I use the symbol $\#$ to indicate both semantic infelicity, i.e., awkwardness in terms of meaning, and constructions that lack a particular interpretation that is crucial from the perspective of a discussed phenomenon. For the sake of clarity, in the latter case I will always provide relevant readings and mark the one that is non-existent with the symbol $\#$. In other words, when $\#$ appears on object language, then it means semantic infelicity; when $\#$ appears on an interpretation line, then it means that that reading is not available. Unless explicitly stated otherwise, I will always consider uncoerced meanings of nominal expressions, i.e., I will ignore the mass-count as well as count-mass shifts. Occasionally, I will use the symbol * in the examples to indicate ungrammaticality.

Concerning glossing, I will introduce the complete grammatical information only when relevant morpho-syntactic issues are discussed or its lack might cause confusion. Otherwise, for the sake of simplicity I limit it to the minimum. The list of abbreviations used in the examples can be found at the beginning of this book. 

\subsection{Notation}\label{sec:notation}

In the following text, I will use a relatively standard notation. For completeness, I specify the following. The symbols $x$, $y$, $z$, and $w$ are used to represent entity variables, whereas $n$ and $P$ are dedicated for number and predicate variables, respectively. The small letters $a$, $b$, $c$, and $d$ stand for entity constants, whereas Arabic numbers, e.g., $1$, $2$, and $3$, stand for integers. On the other hand, metalinguistic constants designating particular properties are transcribed with small capitals, e.g., \textsc{apple} stands for the property of being an apple, whereas logical constants designating various operations are typeset using the sans-serif font, e.g., \cnst{max} for the maximization operation. The primitive semantic types of entities, numbers, and truth values are represented by the symbols $e$, $n$, and $t$, respectively. Only complex types are given in angle brackets, e.g., $\langle e,t\rangle$ as opposed to $e$. Furthermore, the scope of both the $\lambda$ operator and quantifiers $\exists$ and $\forall$ is indicated by square brackets, e.g., $\lambda x[\textsc{apple}(x)]$ and $\exists x[\textsc{apple}(x)]$. For convenience, in complex formulae I use small and big brackets in order to designate scope of particular operators. Finally, presuppositions are introduced always following a colon immediately after the relevant $\lambda$ operator, e.g., in the term $\lambda x:\textsc{fruit}(x)\ \lambda P[\text{*}P(x) \wedge \#(P)(x) = 3]$ the sequence $\textsc{fruit}(x)$ is a presupposition. 

\section{The things to come}\label{sec:the-things-to-come}

A substantial body of linguistic research has been dedicated to quantification in natural language. This monograph is intended to contribute to that enterprise by presenting novel data concerning subatomic quantification and proposing novel arguments in favor of a mereotopological approach to nominal semantics. The book is structured as follows. The next three chapters will provide linguistic evidence on the relevance of subatomic quantification mainly from a cross-linguistic perspective. I will examine a broad range of constructions involving different types of partitives, whole-adjectives, and multipliers such as English \textit{double}. 

In Chapter \ref{ch:partitives-and-part-whole-structures}, I will discuss certain patterns observed in partitive constructions cross-linguistically that suggest a unified meaning of partitive words such as \textit{part} and \textit{half} as well as the importance of the topological notion of integrity in subatomic quantification. The main focus will be dedicated to the interaction between numerals, partitive words, and nominals involving count singulars, regular plurals as well as Italian irregular plurals. 

Chapter \ref{ch:exploring-topological-sensitivity} will provide further evidence for the significance of integrity in quantification over parts. In particular, I will present novel data concerning topo\-logy-sensitive partitive words that require the whole to be an integrated object or yield only integrated parts of the whole. Though the evidence will come mainly from Polish, parallels with other languages will also be drawn including English, German, and Mandarin. 

In Chapter \ref{ch:multipliers}, I will explore the meaning of multipliers, i.e., a neglected class of numerical expressions that is specialized to count parts of entities denoted by the modified noun. The discussed data will come from Slavic where morphological evidence supports semantic complexity of those expressions. 

The main purpose of Chapter \ref{ch:conceptual-background} is to provide the three claims that constitute the conceptual background for the analysis. For that purpose, I will relate the previously introduced linguistic evidence to psychological findings concerning the role integrity and part-whole structures play in human cognition and especially in counting. 

In Chapter \ref{ch:theory-of-parts-and-wholes}, I will introduce mereotopology, i.e., a theory of wholes extending the mereological framework based on the notion of parthood with the topological concept of connectedness. Mereotopology allows us to capture the difference between distinct types of things, such as integrated wholes, arbitrary sums of individuals, and scattered entities corresponding to substances, and will prove advantageous in modeling quantification over parts. 

Chapter \ref{ch:mereotopological-account-for-subatomic-quantification} will be dedicated to spelling out a formal analysis of selected issues in subatomic quantification. The account will be based on the mereotopological framework with the intuitive notion of integrity playing the main role in the semantic representation of countable parts. 

Finally, Chapter \ref{ch:conclusion} will conclude the book and suggest how further research could extend the approach in order to tackle some open questions. For convenience, the most relevant contribution of a given part is summarized at the end of each chapter.



	\chapter{Mereotopological account for subatomic quantification}\label{ch:mereotopological-account-for-subatomic-quantification}
	
	Given the mereotopological framework introduced in the previous chapter, let us now attempt to account for some of the relevant expressions employing quantification over parts such as partitives and multiplier phrases as discussed in Chapter \ref{ch:partitives-and-part-whole-structures}, \ref{ch:exploring-topological-sensitivity}, and \ref{ch:multipliers}. The conceptual basis of the general mechanism of subatomic quantification outlined in Chapter \ref{ch:conceptual-background} can now be spelled out formally with the notions of parthood and connectedness playing the central role in modeling the meaning of expressions that provide objects to be counted as well as of those that do the counting. 
	
	In this chapter, I will propose an analysis of different types of partitives and multiplier phrases. I will start with the general discussion of the implausibility of atomicity-based approaches to subatomic quantification. Next, I will turn to my own proposal spelled out in terms of mereotopology, supplemented with additional components such as measure functions, partitions, and the individuation operation. First, I will address the mass/count distinction by distinguishing between different types of nominals, as well as give the meaning of the plural in the spirit of \citet{grimm2012number}. Subsequently, I will turn to defining other semantic objects I assume in the structure of different types of partitives and multiplier phrases including the meanings of different types of partitive words as well as numerical expressions, which I treat as complex constituents comprising a classifier component. Finally, I will propose a compositional analysis of the expressions in question by showing how the pieces fit together.
	
	\section{Doing without atoms}\label{sec:doing-without-atoms}
	
	In standard theories of pluralities and countability the mass/count distinction is often formulated in terms of atomicity \citep[e.g.,][]{link1983logical,landman1991structures,landman2000events,chierchia1998plurality,chierchia2010mass}. Though particular theories differ significantly with respect to the exact character of the alternation, the contrast between count and mass nouns usually boils down to the existence or lack of minimal building blocks in their denotation or, alternatively, to a distinct nature of those building blocks. The approach developed here rejects the view that what counts as one is best represented as an atomic entity. Instead, I build on the mereotopological notion of \cnst{mssc} introduced in  \sectref{sec:integrated-wholes} to capture what can be an object of counting. There are three main motivations behind this decision. The first one is the linguistic relevance of topological relations holding between parts of a whole object as empirically attested in multiple constructions in various languages discussed in Chapters \ref{ch:partitives-and-part-whole-structures} and \ref{ch:exploring-topological-sensitivity}. The second stems from the general counting principles described in  \sectref{sec:general-counting-principles}, which I believe constitute a deeply rooted part of human cognition and as such interact with that part of the language faculty that generates the grammar of countability. Finally, the last reason for abandoning atomicity is simply that it does not seem helpful with respect to subatomic quantification. Before we move on to the proper proposal, let us briefly consider why that is.

	Things that count as one are typically defined as atomic individuals, i.e., entities that have no proper parts, see \ref{ex:atom} for the definition of the notion \textsc{atom} and \ref{ex:atomicity} for an optional mereological axiom of \textsc{atomicity} that delivers atomic domains, i.e., domains consisting of atoms.
	
	\ex. Atom\\
	$\cnst{atom}(x) \leftrightarrow \neg \exists y[y \sqsubset x]$\\
	(An atom is an entity which has no proper parts.)\label{ex:atom}
	
	\ex. Atomicity\\
	$\forall x \exists y[y \sqsubseteq x \wedge \neg \exists z[z \sqsubset y]]$\\
	(For any element, there is a part for which there does not exist a proper part.)\label{ex:atomicity}
	
	At first blush, such an approach to what it means to count as one seems counterintuitive since we know from our everyday life experience that things we count as one very often do have multiple parts. However, one should not confuse intuitive concepts with formal notions devised to represent the semantics of natural language expressions, e.g., the meaning of the English partitive word \textit{part} with $\sqsubseteq$. Thus, my view is that there is nothing wrong in dissociating intuitive entities from their intuitive parts in terms of mereological modeling as long as it serves a certain purpose. For instance, \citet{champollion2010parts,champollion2017parts} assumes that the relation between, say, a teddy bear named Fuzzy Wuzzy and its paw is not mereological parthood. In other words, despite the fact that Fuzzy Wuzzy consists of a number of material parts it is considered a mereological atom, i.e., an inseparable unit in the domain of entities. This might seem plausible since it corresponds to yet another intuition, namely that though we know that things have parts, we seem to ignore this fact when we count those things. However, a significant problem for such an account arises when we decide to quantify not over whole things but over parts of those things. Thus, I will argue for two claims. First, having a notion of atomicity is not enough for a full analysis of quantification of parts. Second, atomicity is actually not needed to analyze quantification over parts since it can be reduced to topological notions which are required independently.
	
	To illustrate this, let us consider the following case involving subatomic quantification. Assume we wanted to count parts of Fuzzy Wuzzy, e.g., its paws. Of course, natural language allows for that since we can say \ref{ex:fuzzy-wuzzy-part}.
	
	\ex. English (Peter Sutton, p.c.)\\
	Exactly two parts of Fuzzy Wuzzy are brown.\label{ex:fuzzy-wuzzy-part}

	The sentence in \ref{ex:fuzzy-wuzzy-part} would be true if, say, Fuzzy Wuzzy's paws were brown where\-as all of its other parts were of a different color. However, if referents of proper names are mereological atoms, there are no parts to be counted since the entity designated by the expression \textit{Fuzzy Wuzzy} cannot be split. Without saying anything else, this is untenable if we want to account for \ref{ex:fuzzy-wuzzy-part}. Consequently, one could follow \citet{link1983logical} and propose to distinguish between a domain of entities and a domain of portions of matter over which a different parthood relation would be defined, i.e., the material parthood relation $\sqsubseteq_m$ as opposed to the individual parthood relation $\sqsubseteq_i$. Assuming the domains are related by a function from entities to portions of matter, one could argue that subatomic quantification is about triggering this mapping and operating on portions of matter corresponding to the mereological atom in the domain of entities. However, such a proposal raises questions. If the domain of portions of matter is atomless, which at first blush might seem intuitive, then how is it possible to count entities of such a domain? Since atomicity is a necessary condition for countability,\footnote{Notice that it is not a sufficient condition due to the existence of object mass nouns which are also assumed to have atomic reference yet are uncountable \citep[see, e.g.,][]{chierchia1998plurality,barner_snedeker2005quantity}.} we need to ensure that at least some portions of matter are atomic. However, this is problematic.\largerpage[-1]
	
	Let us now imagine that for some reason we wanted to count parts of a part of an individual. Again, natural language does allow for that. The sentence in \ref{ex:fuzzy-wuzzy-part-of-part} is perhaps not something one would frequently say but it sounds completely natural and it is definitely interpretable. 
	
		\ex. English (Peter Sutton, p.c.)\\
	Exactly two parts of this part of Fuzzy Wuzzy are brown.\label{ex:fuzzy-wuzzy-part-of-part}
	
	It would be true if, say, exactly two digits of Fuzzy Wuzzy's left paw were brown and the remaining part of the paw were not. Notice that truth-conditionally \ref{ex:fuzzy-wuzzy-part-of-part} is not equivalent to \ref{ex:fuzzy-wuzzy-part}. Consider, for instance, a scenario where the brown parts of Fuzzy Wuzzy included only its head, right leg, and two digits of its left paw. In such a scenario, \ref{ex:fuzzy-wuzzy-part-of-part} would be true but \ref{ex:fuzzy-wuzzy-part} would not. Now, if countable portions of matter are atomic, then the same problem as discussed with respect to atoms in the domain of entities arises. Specifically, if a portion of matter corresponding to Fuzzy Wuzzy's left paw is modeled as having no proper parts, then how can we quantify over its parts? To account for that we would be forced to postulate another domain and relate it with the domain in which the material part corresponding to Fuzzy Wuzzy's paw is an atom. However, this seems weird and even if we did it, the same problem would arise again if we wanted to count parts of a digit of Fuzzy Wuzzy's paw. Consequently, distinguishing between just two distinct parthood relations is not enough. In fact, since there are multiple possible divisions of matter, we might end up establishing numerous or even potentially infinitely many domains in order to be able to define parts as atomic objects in a particular domain. This, of course, is far from desirable and it seems to me that the discussed example strongly suggests that sorting domains does not offer a tenable solution to the issues concerning subatomic quantification and atomicity.
	
	Yet another argument against the two domain view can be put forward based on examples such as \ref{ex:fuzzy-wuzzy-paws}.
	
	    \ex. English (Peter Sutton, p.c.)\\
    Exactly two parts of Fuzzy Wuzzy are brown: his (two) paws.\label{ex:fuzzy-wuzzy-paws} 

    If one said that exactly two parts of Fuzzy Wuzzy are brown and this was true because of his paws, then \ref{ex:fuzzy-wuzzy-paws} would not be straightforwardly true since the paws would be interpreted realtive to a different domain than the two parts. This further suggests that a Linkean approach is not a good path to follow. 
	
	Since sorted domains proved unsatisfactory for our purpose, one might think of another idea. As discussed in  \sectref{sec:mass-parts-quantitifes-and-pieces}, one of the very few attempts to propose a solution to the problem of countability of portions of matter (as opposed to uncountability of matter) preserving atomicity was developed in \citet{chierchia2010mass}.\footnote{Another account worth mentioning was proposed by \citet{landman2016iceberg} but this theory also abandons atomicity.} In this system, an expression such as English \textit{part} is modeled as a variable over partitions of an entity, i.e., an expression of type $\langle e,\langle e,t\rangle\rangle$ that selects an entity and returns a set of relative atoms that are `spatio-temporally included' in that entity. Since in principle such partitioning can be applied recursively, we can obtain parts of parts of entities etc. This seems to be a significant improvement compared to an attempt to account for subatomic quantification in terms of sorted domains and distinct parthood relations. However, there are still several issues with this approach. 
	
	First, as already discussed, an analysis along the lines of \citeauthor{chierchia2010mass}'s proposal fails to account for the fact that parts can be either continuous or discontinuous, and, crucially, natural language turns out to be sensitive to this distinction. Second, the notion ``spatio-temporally included'' introduced above is very loose. Compared to the fine-grained concepts discussed in the previous chapter, it appears to be very unsatisfactory. In general, I agree with \citeauthor{chierchia2010mass}'s ``mereotopological'' intuition; however, its formulation leaves a lot to be desired. For instance, imagine that Fuzzy Wuzzy was left in a car. Does it mean that for linguistic purposes it is now part of that car? Or if an elephant accidentally swallowed Fuzzy Wuzzy in a zoo only to throw it up after some time, would the poor teddy bear be considered part of that elephant in the period between swallowing and throwing up? Intuitively, that does not seem to be the case. Finally, from my point of view postulating partitions of entities is an attempt to compensate deficits of atomicity rather than a genuine development. On the one hand, it seems like circumventing the ban on having proper parts by singular individuals, which are building blocks of denotations of count nouns. On the other, though it seems to point in a desirable direction, it somewhat stops halfway. As we will see, given the potential of a mereotopological account, the vague concept of a partition of entities does not make a tenable alternative. In particular, if maintaining one unified parthood relation proved successful, then partitions of entities could be reformulated in terms of such unified $\sqsubseteq$, as we will see in \sectref{sec:partitions}. Though I will return to partitioning as a useful tool in modeling partitive words denoting continuous parts, I do not find \citeauthor{chierchia2010mass}'s proposal an overall satisfactory analysis.
	
	I believe that at this point it is plausible to conclude that in order to account for subatomic quantification it is desirable to find a substitute for atomicity. One might think that a plausible alternative would be to adopt frameworks that model building blocks of denotations in terms of natural or object units, as proposed by \citet{krifka1989nominal,krifka1995common}. At first blush, this might seem attractive because what counts as one, i.e., an entity to which we can assign the number 1, is not defined in terms of not having proper parts. Thus, in principle accounting for quantification over parts of such an object would not have to face the issues described above. Nonetheless, such an approach would run into a nagging conceptual problem.
	
	Intuitively, it is quite straightforward what a natural unit of, say, entities that have the property of being an apple is. Alternatively, it is clear what kind of concept such entities realize as object units. But what is a natural unit of a part? Or what kind of concept does a part instantiate as an object unit? Given the systems mentioned above, these notions would have to refer in some way to the denotations of partitive words. But it seems to me that this is a very strange way of thinking about parts. However, even if one accepts the existence of natural units or concepts of this sort, other questions arise. Since there are numerous ways how to divide portions of matter making up an entity, does it mean there are numerous natural units or concepts corresponding to such portions, e.g., a natural unit/concept of a tiny part, half, great part, and not-that-small-but-not-that-big part? Can both continuous and discontinuous parts be considered as natural units? Are they realizations of the same concept or two different concepts? Is there a difference between natural units or concepts corresponding to parts of a singularity and those corresponding to parts of a plurality? If yes, then how do quantificational operations know that one is to be selected over the other? If no, then how can we distinguish between the semantic subtleties that the meaning of partitive words are associated with? It feels like all of these questions are quite odd questions but they would need to be addressed seriously in order to account for the data presented in Chapter \ref{ch:exploring-topological-sensitivity}. Therefore, it appears to me that trading in atomicity for natural or object units does not make a satisfactory alternative with respect to subatomic quantification. 
	
	Given the problems considered above, I conclude that an approach based on mereotopology proves more advantageous for accounting for subatomic quantification. Not only does it provide means to better capture what it means to be a whole but also, as we will see, it turns out to be extremely useful in modeling those entities that can be counted. In the following sections, I will introduce a minimal set of tools necessary to model two different types of constructions concerning quantification over parts, namely multiplier phrases and partitives. I treat this repertoire of means alongside the standard principle of Function Application as the first attempt to develop a compositional analysis of the syntactic structures in question that could shed new light on the somewhat neglected phenomenon of subatomic quantification as well as on countability in general. At the same time I believe the real journey begins afterwards.    
	
	\section{Common nouns}\label{sec:common-nouns}\largerpage
	
	I will start my proposal with a brief discussion of the semantics for different types of concrete nouns, i.e., expressions denoting sets of individuals such as apples and teddy bears as well as scattered entities such as juice and rice. Though everything I have to say about the semantics of nouns extends naturally to NPs, for the sake of brevity I will focus only on nouns. In general, I assume that all common nouns of the sort discussed in this study as well as nominals derived from them, e.g., by adjectival modification, are standard $\langle e,t\rangle$ predicates, i.e., functions from entities to truth values. Among such expressions there are some that are countable and some that are not. Though countability is, of course, a grammatical category, I assume that the mass/count distinction correlates with cognitive factors determining how human beings conceptualize things in the world, as discussed in Chapter \ref{ch:conceptual-background}. In particular, I take the counting principles of non-overlap, integrity, and maximality as the benchmark for deciding what is countable and what is not. In other words, I postulate that only nouns whose referents satisfy all three criteria are countable. In the next section, I will propose how the meaning of such predicates can be captured.
	
	\subsection{Count nouns}\label{sec:count-nouns}
	
	Building on \citet{grimm2012number}, I model the distinction between count and mass nouns in terms of mereotopological distinctions developed in the previous chapter. In particular, I propose that defining countable predicates is about ensuring that their extensions include only entities conceptualized as objects, i.e., discrete integrated wholes that are perceived as being disjoint from each other. Intuitively, the core distinction between count nouns, on the one hand, and mass nouns including substance terms, granulars, and object mass nouns, on the other, can be reduced to the fact that the former denote only individuated objects, whereas the latter either do not have integrated wholes in their extension at all, or they do but in addition they also refer to other types of entities. For instance, the noun \textit{apple} simply denotes a set of individuals, i.e., distinct apples. On the other hand, the referents of \textit{juice} do not have properties that integrated wholes have, i.e., they are not constrained as bounded non-overlapping discrete objects. The case of granular terms such as \textit{rice} is somewhat more complicated. Arguably, those predicates do denote integrated wholes since one can point at a single grain of rice and truthfully call it \textit{rice}. Intuitively, grains of rice have similar properties as apples, i.e., they are disjoint, integrated, and mereologically maximal, however \textit{rice} does not refer exclusively to individual grains of rice. In addition, it can also denote arbitrary sums as well as clusters of rice which lack those properties. Likewise, object mass nouns such as \textit{footwear} have both integrated wholes, i.e., individual shoes, as well as groups of such individuals, i.e., pluralities of shoes, in their extensions. Thus, though there are important differences between particular types of mass nouns, they all contrast with count nouns with respect to what kinds of entities fall into a denoted set.
	
	\begin{sloppypar}
	Since mereotopology provides powerful means to distinguish formally between objects conceptualized as integrated wholes and other types of entities, I will use the notion of \cnst{mssc} defined in \ref{ex:maximally-strongly-self-connected} in order to capture the main intuition behind the contrast discussed above. Specifically, I will employ the variant of \cnst{mssc} relativized to a property as provided in \ref{ex:maximally-strongly-self-connected-relative} in \sectref{sec:integrated-wholes} and repeated here as \ref{ex:maximally-strongly-self-connected-relative2}. As discussed in  \sectref{sec:integrated-wholes}, this allows us to distinguish between integrated objects such as individual apples and scattered entities like juice, on the one hand, and arbitrary sums such as pluralities of apples, on the other.
	\end{sloppypar}
	
	\ex. Maximally strongly self-connected relative to a property\label{ex:maximally-strongly-self-connected-relative2}\\
	$\cnst{mssc}(P)(x) \eqdef P(x) \wedge \cnst{ssc}(x) \wedge \forall y[P(y) \wedge \cnst{ssc}(y) \wedge \cnst{o}(y,x) \rightarrow y \sqsubseteq x]$\\
	(An m-individual is maximally strongly self-connected relative to a property if (i) every part of the individual is connected to (overlaps) the whole (strongly self-connected) and (ii) anything else which has the same property, is strongly self-connected, and overlaps it is once again part of it (maximality).)
	
	Given the mereotopological notion of \cnst{mssc}, we can now characterize the subset of predicates that refer exclusively to integrated wholes. In the following text, I use the symbol \cnst{pmssc} to refer to a higher order predicate of which such predicates are true. As stated in \ref{ex:mssc-predicate}, there is nothing in the denotation of a predicate satisfying \cnst{pmssc} that is not an \cnst{mssc} entity.\footnote{For convenience, I use the term \cnst{mssc} entity/individual to refer to objects that are maximally strongly-self connected relative to a relevant property.} This means that, e.g., \textit{apple} does satisfy \cnst{pmssc}, whereas \textit{juice}, \textit{rice}, and \textit{furniture} do not since either they do not refer to \cnst{mssc} individuals at all or they also denote entities that are not \cnst{mssc}.

    \ex. Predicate of \cnst{mssc} individuals\\
	$\cnst{pmssc}(P) \eqdef \forall x[P(x) \rightarrow \cnst{mssc}(P)(x)]$\\
	(Any m-individual of which a predicate satisfying \cnst{pmssc} is true is an \cnst{mssc} m-individual relative to the relevant property.)\label{ex:mssc-predicate}
	
	Let us now consider the semantics of the English count noun \textit{apple}, as provided in \ref{ex:count-noun}.
	
	\ex. Count noun\\
	$\llbracket \text{apple}\rrbracket = \lambda x[\cnst{mssc}(\textsc{apple})(x)]$\label{ex:count-noun}
	
	What the semantics in \ref{ex:count-noun} says is that \textit{apple} is an expression of type $\langle e,t\rangle$ which yields the truth value True for entities that are \cnst{mssc} with respect to the property \textsc{apple}. Given the definition in \ref{ex:maximally-strongly-self-connected-relative2}, \cnst{mssc}(\textsc{apple})(x) entails \textsc{apple}(x) which means that the discussed predicate denotes a set of apples that are integrated wholes. Therefore, the proposed semantics captures the intuition that referents of count nouns constitute discrete objects without any reference to the notion of atomicity.
	
	Since the main topic of this study concerns subatomic quantification, I will refrain from discussing the exact semantics for different types of mass nouns. Nonetheless, the meaning of all of the expressions mentioned above can be captured in terms of mereotopological distinctions. In particular, substance terms such as \textit{juice} can be modeled in terms of m-individuals having a certain property that are firmly connected to distinct m-individuals with the same property, formally involving the \cnst{fc} relation introduced in \ref{ex:firmly-connected} \citep[see][pp. 140--142]{grimm2012number}. On the other hand, granular mass nouns such as \textit{rice} can be accounted for by treating them as aggregates, i.e., expressions that can denote \cnst{mssc} individuals, their sums as well as clusters thereof, i.e., configurations of \cnst{mssc} entities transitively connected via the \cnst{tc} relation defined in \ref{ex:transitively-connected-new} \citep[see][pp. 142--148]{grimm2012number}.\footnote{Clusters can be further relativized to different types of connections, such as external connection, see \cnst{et} in \ref{ex:externally-connected}, and proximate connection, see \cnst{pc} in \ref{ex:proximately-connected} \citep{grimm2012number}.} Finally, object mass nouns such as \textit{footwear} can be thought of as denoting both \cnst{mssc} entities and sums thereof though definitely more work needs to be done with respect to this category.
	
	\begin{sloppypar}
	The mereological approach to nominal semantics allows us to distinguish countable nouns without postulating atomicity. As we will see, this will prove crucial in accounting for subatomic quantification. However, before I move to the discussion of the machinery behind measuring and counting, the issue of number morphology on nouns needs to be addressed. In the next section, I will take up the meaning of the plural.
	\end{sloppypar}
	
	\subsection{Artifacts and functional units}\label{sec:functional-units}
	
	Before I move on to the discussion of plurals, cardinal numerals and multipliers, and partitive words, I would like to address a class of apparent counterexamples to the proposed account. \citet[p. 21]{grimm2012number} explicitly restricts his meretopological approach to natural concrete entities, i.e., referents of expressions such as \textit{cat} and \textit{apple}. Aside from the reasons of simplicity stated by \citeauthor{grimm2012number} for why artifact nouns, e.g., \textit{clock} and \textit{pen}, are excluded from the analysis, there are also other reasons for doing this. In particular, it might seem that lots of artifacts that we count as one do not seem to be \cnst{mssc} individuals. For instance, a clock is a system of connected gears, hands, a spring, and a clock face. For some clocks, these parts may be merely touching, like interlaced cogs. Similarly, a pen can be the main part of the pen and its lid, which can merely touch. Other potential counterexamples include \textit{chain}, which designates a string of interlinked \cnst{mssc} individuals, nouns such as \textit{fence} and \textit{wall}, which might overlap at edges, and count counterparts of object mass nouns, e.g., a pestle and mortar can count as one piece of kitchenware \citep[see][]{sutton_filip2016counting}. 
	
	Admittedly, the examples provided above seem to pose a problem for the semantics of count nouns proposed in \ref{ex:count-noun} and for the principle of integrity in general since, strictly speaking, they are not \cnst{mssc} individuals. One possible reply is that, given the cognitive view of meaning adopted in this book, see \sectref{sec:cognitive-view-of-meaning}, what matters for grammar is how referents of artifact nouns are conceptualized rather than what objects, which we use language to talk about, exactly look like in the mind-external world. This means that there might be mismatches between mereotopological structures in the part of the human mind that is relevant for natural language semantics, on the one hand, and the internal make-up of (at least some) physical entities we use linguistic expressions to refer to, on the other. 
	
	Another possibility is to loosen the relationship between the notion of integrity and the \textsc{mssc} relation. In particular, it is possible that the referents of natural concrete entities exemplified by \ref{ex:count-noun} are only one particular subtype of things conceived of as integrated objects. Although artifactual entities are not integrated in the sense of \cnst{mssc} (in the physical space), there are still other mereotopological notions that might be relevant for the way such entities are conceptualized as integrated wholes. For instance, one could model artifacts as various types of clusters, i.e., collections of parts that (typically) either touch each other or remain in close proximity, see \sectref{sec:other-types-of-connection}. This solution would require some slight amending of the principle of integrity, and consequently the semantics of cardinal numerals and multipliers. However, it would definitely preserve the core idea and the general spirit of the approach. 
	
	Finally, I would like to suggest that it is also possible to have a unified account on which both natural entities and artifacts are modelled in terms of the \cnst{mssc} relation by extending mereotopology to abstract domains. Though this requires several non-standard assumptions, some of which might at first sight seem rather controversial, I believe that it is worth considering since it offers a very promising and advantageous perspective. 

	First of all, notice that artifacts are usually characterized by their intended function, e.g., measuring and indicating time for the clock and writing for the pen. This makes them conceptually different from natural concrete entities, which seem to lack any functional component. \citet{grimm_levin2017artifacts} argue that the semantics of an artifact noun includes an associated event, which often represents the intended function of the corresponding object. This is a very interesting idea and one could extend the framework proposed here with this kind of approach. However, there is yet another account I would like to suggest.
	
	As already mentioned in \sectref{sec:social-roles}, there are reasons to assume that natural language is sensitive to the distinction between individuals and roles, which are certain functions or capacities of entities. \citet{zobel2017sensitivity} provides a number of linguistic phenomena involving, e.g., English and German as-phrases, which demonstrate the relevance of this distinction, and proposes to model roles as independent ontological objects of the primitive type $r$ that can be also associated with individuals that perform them. Though previous research has focused on roles of human individuals, e.g., \textit{judge} \citep{zobel2017sensitivity} and \textit{clergy} \citep{wagiel-toappear-slavic}, it is conceptually conceivable that there might also be ``inanimate'' roles, which would represent functions or capacities of inanimate objects. For instance, consider the contrast in \ref{ex:function-use-as}.
	
	\ex. English (Katie Fraser, p.c.)\label{ex:function-use-as}
	\a. Jenny used that thing as a pen.\label{ex:function-use-as-artifact}
	\b. \#Jenny used that thing as an apple.\label{ex:function-use-as-natural} 
	
	The sentence in \ref{ex:function-use-as-artifact} is perfectly natural since it is easy to imagine that one could use various items for writing, e.g., a stick. On the other hand, \ref{ex:function-use-as-natural} is weird because apples have no intended function unless the context makes it clear that in this particular situation apples are used in a very special way, e.g., as some kind of offering in a religious ritual.\footnote{Of course, there are also nominals that can get coerced to an artifact meaning in a particular situation, e.g. a stone or a stick can have a function if they are used as a skipping stone or a walking stick, respectively \citep[see also][pp. 256--259]{asher2011lexical}. Importantly, however, the function seems to stem from the human intention to use an object in a particular way rather from some intrinsic property of that object.} 
	
	An important result in the last four decades of the study of part-wholes structures is that mereological relations hold not only between concrete physical objects but also between abstract entities such as events \citep{bach1986algebra}, information states \citep{krifka1996parametrized}, times \citep{arstein_francez2003plural}, and degrees \citep{dotlacil_nouwen2016comparative} as well as propositions \citep{lahiri2002questions} and functions \citep{schmitt2013more,schmitt2019pluralities}. Therefore, I see no a priori reason to assume that this cannot be true also for mereotopological notions. Of course, such an approach would require abstracting from the connectedness relation \cnst{c} as a relation between physical objects. Rather, it would need to be viewed as a purely abstract notion that can hold between entities of any type (similarly to the parthood relation $\sqsubseteq$). The core intuition behind such an idea is that the manner in which parts of a whole are arranged can be equally relevant for any type of entity including abstract objects.
	
	Importantly, there is evidence that mereological structures apply also in the domain of roles, see \ref{ex:conjunction}  \citep{wagiel-toappear-slavic}. 
	
	\ex. English \citep{wagiel-toappear-slavic}\label{ex:conjunction} 
	\a. Paul gave 4,000 euros to Tom and Amy.\label{ex:conjunction-individuals}
    \b. Paul earns 4,000 euros as a judge and a lecturer.\label{ex:conjunction-roles}
	
	It is well known that conjunction in examples such as \ref{ex:conjunction-individuals} gives rise to the distributive and the non-distributive reading. Specifically, the sentence can either mean that Tom and Amy got 4,000 euros each or that they got 4,000 euros between them. Interestingly, \ref{ex:conjunction-roles} displays the same type of ambiguity. On the distributive interpretation, Paul earns 4,000 euros working as a judge and 4,000 euros working as a lecturer, i.e., 8,000 euros in total. In addition, the sentence has the non-distributive reading on which Paul earns a total of 4,000 euros for both of those two professional roles.
	
	The existence of mereological structures in the domain of roles opens up the possibility for a topological extension. For the sake of presentation, let us assume that both individuals and roles are conceptualized as occupying positions within regions of space, the former as concrete things located in physical space whereas the latter as abstract entities inhabiting abstract functional space. We could then postulate connected roles if, e.g., two roles involved overlap. Such an extension of the model would allow for developing more complex mereotopological notions including \textsc{mssc}. As a result, an artifact could be modelled as an \textsc{mssc} inanimate role, i.e., abstract integrated functional unit, that can be in turn mapped onto entities that perform that function. This would allow us to talk about functional integrity as a notion that is at the same time independent and closely related to physical integrity.
	
	Since the main focus of this book concerns subatomic quantification, I will not pursue a detailed implementation of the idea of functional integrity here, but instead leave it for future research.\footnote{But see \citet{wagiel-toappear-slavic} for a mereotopological attempt to model social collective nouns as properties of clusters of roles.}  Nonetheless, I conclude that, given the analytical possibilities discussed above, the proposed approach can be extended to account also for potentially problematic artifact nouns without sacrificing its core components. 
	
	\subsection{Pluralization}\label{sec:pluralization}
	
	Following \citet{grimm2012number}, I assume that number morphology is sensitive to the mereotopological structure of referents of the basic, i.e., unmarked, form of a noun.\footnote{Alternatively, one could argue that this kind of information is already encoded in the semantics of the stem \citep[pace, e.g.,][]{pelletier_schubert1989mass,borer2005name}. However, for the sake of brevity I will not explore this hypothesis here.} Specifically, as we saw in the previous section in a language such as English only a subset of nouns can be characterized as denoting single objects conceptualized as \cnst{mssc} individuals. For the purpose of this study, I assume that in general this property corresponds to countability and thus the compatibility of such nouns with the plural morpheme.\footnote{This is, of course, a simplification. As mentioned before, for the sake of simplicity I ignore here collective nouns as well as count abstract nouns. However, I assume that in principle it is possible to extend the core mereotopological concepts to develop a framework based on notions derived from or inspired by \cnst{mssc} that would account for phrases such as \textit{two ideas} and \textit{three committees} \citep[see also][]{grimm2014individuating}.} In particular, the plural marker attaches only to predicates that have exclusively \cnst{mssc} entities in their extensions.\footnote{Without any additional assumptions, this approach fails to account for inherently plural mass nouns such as \textit{leftovers} \citep{acquaviva2008lexical} as well as pluralized mass nouns as observed, e.g., in Greek \citep{tsoulas2009grammar}. However, since the general topic of the meaning of the plural lies beyond the scope of this study, I leave it for future research.} 
	
	Keeping this in mind, the plural can be analyzed as an operation which selects a set of \cnst{mssc} individuals and applies the strict pluralization operation, see \ref{ex:algebraic-closure}--\ref{ex:plural}. 
	
    \ex. Algebraic closure \citep[see][]{link1983logical}\\
    $\text{*}P \eqdef \{x | \exists P' \subseteq P[x = \bigsqcup P']\}$
    \label{ex:algebraic-closure}
    
    \ex. Strict pluralization\\
	${}^+P \eqdef \text{*}P - P$\label{ex:strict-pluralization}
    
	\ex. Plural\\
	$\llbracket \text{PL}\rrbracket = \lambda P : \cnst{pmssc}(P) [{}^+P]$\label{ex:plural}
	
	The semantics in \ref{ex:plural} involves the ${}^+$ operator whose definition is provided in \ref{ex:strict-pluralization}. As one can see, ${}^+$ employs the classical \text{*} operator introduced by \citet{link1983logical}, which is defined as a closure under sum formation, see \ref{ex:algebraic-closure}. Specifically, what ${}^+$ does is that it selects a predicate, applies \text{*} to it and then removes all the \cnst{mssc} individuals from the pluralized set. The selectional requirement of the plural marker is introduced by what I refer to as the \textsc{individuation presupposition} represented by $\cnst{pmssc}(P)$ introduced after a colon following immediately the relevant $\lambda$.\footnote{For notational conventions used in this book see  \sectref{sec:notation}.} The individuation presupposition requires an argument of the $\lambda$ operator to be a predicate denoting only \cnst{mssc} individuals as defined in \ref{ex:mssc-predicate}.
	
	As indicated by \ref{ex:plural}, I adopt here the so-called exclusive analysis of the plural. Within this view, plurals are assumed to refer only within the domain of sums, i.e., singularities are excluded from the denotation, and thus the plural is treated as designating \textit{more than one} (cf. \citealt{hoeksema1983plurality,chierchia1998plurality,chierchia1998reference,grimm2013plurality}; see also \citealt{wagiel2017pairs}).\footnote{This decision is mainly motivated by independent reasons and as far as I can see for the most part nothing hinges on this from the perspective of the phenomena I attempt to account for here.  However, the exclusive analysis will prove advantageous in the context of set partitives. Notice also that many arguments have been proposed in favor of the inclusive view on which singularities are included in the meaning of the plural mainly based on the data concerning downward-entailing environments \citep[see, e.g.,][]{link1983logical,schwarzschild1991meaning,landman2000events,sauerland_et_al2005plural,spector2007aspects,zweig2009number}. However, see \citet{grimm2013plurality} for an analysis of such contexts maintaining the exclusive view.} Hence, what the plural does is that it selects a set including only \cnst{mssc} individuals and yields a new set consisting of all the sums formed by joining the members of the input set. For instance, let us consider the semantics of the bare plural \textit{apples}, see \ref{ex:plural-np}. 
	
	\ex. Plural NP\label{ex:plural-np}\\
	$\llbracket \text{apples}\rrbracket = \llbracket \text{PL}\rrbracket (\llbracket\text{apple}\rrbracket) = \lambda x\big[{}^+\big(\lambda y[\cnst{mssc}(\textsc{apple})(y)\big]\big)(x)\big]$
	
	In this case, the plural morpheme \textit{-s} combines successfully with the noun since the individuation presupposition is satisfied, i.e., \textit{apple} is a predicate of \cnst{mssc} entities. As a result, the set denoted by \textit{apple} is first closed under sum and then all the \cnst{mssc} individuals are removed. Hence, we obtain a new predicate which is true only of entities that are pluralities of apples.
	
	To see how this works, consider the following example. Let us assume a model with three individual apples $a$, $b$, and $c$. Given such a model, the common noun \textit{apple} denotes the set of singular apples as in \ref{ex:plural-np-sg-ex}. On the other hand, its plural counterpart \textit{apples} in \ref{ex:plural-np-pl-ex} denotes the set comprising all the sums obtained from the singular individuals, i.e., pluralities of apples. In both cases, the semantic type of the expression in question is $\langle e,t\rangle$.
	
	\ex. Plural NP\label{ex:plural-np-ex}
	\a. $\llbracket \text{apple}\rrbracket = \{a,b,c\}$\label{ex:plural-np-sg-ex}
	\b. $\llbracket \text{apples}\rrbracket = \{a\sqcup b, a\sqcup c, b\sqcup c, a\sqcup b\sqcup c\}$\label{ex:plural-np-pl-ex}
	
	An important advantage of such treatment of the plural is that it explains why the plural marker does not occur on mass nouns without triggering any additional semantic effect such as a portion interpretation. Or, alternatively, it is not the plural marking that triggers that effect but rather, the application of the Universal Packager is a prerequisite for the plural morpheme to be able to attach to a mass noun.\footnote{I refrain here, from the discussion of the role of the Universal Sorter since it is unclear whether kind-level entities can be considered subject to topological relations. Potentially, subkind readings constitute a problem for the described approach. For sure, more research is required on this topic.} Since packaging restricts the extension of a mass term in such a way that the resulting portion has properties similar to integrated objects, it is plausible to postulate that it does so by applying an \cnst{mssc} constraint or at least some related restriction. In any case, non-coerced mass nouns fail to satisfy the individuation presupposition simply because either they do not refer to \cnst{mssc} individuals at all as in the case of substance terms, or though their denotations include \cnst{mssc} individuals, they also include other entities such as sums (object mass nouns) and clusters (granulars).
	
	Apart from the individuation presupposition, the approach to the meaning of the plural provided here is essentially one of the two standard analyses. However, there is an important comment to be made concerning the morpho-syntax/ semantics interface. Namely, I assume that not all morphological plurals get the interpretation in \ref{ex:plural}. Leaving aside pluralia tantum which would require much unrelated consideration, I posit that only bare plurals are semantic plurals, i.e., expressions on which the plural marker is interpreted. To foreshadow, I will claim that plural morphology in quantificational constructions such as numeral phrases in languages such as English makes by contrast no semantic contribution and is merely triggered syntactically by agreement \citep[see, e.g.,][]{krifka1989nominal,krifka2007masses,ionin_matushansky2006singular,deal2017countability}.
	
	\subsection{Consequences}\label{sec:consequences}
	
	Adopting the mereological account for nominal semantics has some interesting consequences. First, as we saw in \sectref{sec:count-nouns} the notion of \cnst{mssc} allows us to model individuals denoted by singular count nouns without reference to atomicity. Therefore, there is no need to postulate a special category of entities that have no proper parts. To the contrary, unlike atoms \cnst{mssc} objects are viewed as certain configurations of parts, specifically configurations forming integrated wholes. This is a radically different approach from purely mereological theories. In my opinion, its great advantage is that it succeeds in capturing two ontological intuitions concerning individuals at the same time. As we saw in Chapter \ref{ch:conceptual-background}, there are good reasons to believe that cognitive structures underlying conceptualization of solid objects differ from those corresponding to scattered entities. On the other hand, human beings are able to perceive wholes simultaneously as discrete units and as configurations of parts. Unlike atomicity-based theories, a mereotopological account enables us to model natural language expressions in such a way that the former cognitive aspect can be captured without sacrificing the latter. From my point of view, this is an important advantage.
	
	The second favorable consequence of mereotopology is that it provides a very simple and intuitive way to distinguish between integrated sums of parts, i.e., singular individuals, and arbitrary sums, i.e., plural entities. The distinction simply boils down to whether topological notions are involved in the part-whole structure of an entity or not. As \cnst{mssc} individuals, referents of singular count nouns consist of connected elements, whereas pluralities encode no topological relations holding between their parts. Intuitively, the sum of two apples $a$ and $b$ remains the same entity $a\sqcup b$ irrespective of whether $a$ and $b$ touch each other, stay in proximity, or are situated in distant locations. On the other hand, $a$ would cease to exist if its halves were separated from each other by some contextually significant distance.\footnote{Often, when parts remain in proximity it is still possible to retrieve the original part-whole structure from the context. Though the constraints on such a `reconstruction' operation are potentially an interesting topic in cognitive science, I refrain here from any speculations on this issue and focus only on clear cases.} In other words, while plural entities can be modeled in purely mereological terms, singular individuals require an additional topological layer. It seems that such a distinction accounts for the intuition that, contrary to what standard mereology posits, singularities and pluralities are very different types of creatures. However, this approach does not exclude yet another ontological possibility, namely the existence of sums of solid entities that are topologically arranged in a particular way. This fact is of considerable significance since there is linguistic evidence indicating that introducing such type of entities might be necessary for the meaning of certain expressions in natural language. As we saw in  \sectref{sec:italian-irregular-plurals}, referents of Italian irregular plurals make a good candidate for such a class.\footnote{Another kind of expressions that are arguably of this sort are certain types of Slavic derived collectives \citep[see][]{docekal_wagiel2018decomposing,grimm_docekal-toappear-counting}.}
	
	There is another welcome result. The theory adopted here allows for distinguishing between different types of entities in terms of the distinction between mereotopological and purely mereological configurations. This fact in turn enables us to reduce the number of domains. In particular, there is no need to distinguish between the domain of entities and the domain of portions of matter as postulated by \citet{link1983logical}. Since referents of count nouns are not considered mereological atoms, i.e., entities without proper parts, but rather mereotopologically constrained sums of parts they have additional properties compared to arbitrary collections of portions of matter. Those properties are formulated in terms of connection, specifically \cnst{mssc}. Therefore, it is sufficient to assume only one domain of concrete things, namely the domain of entities, and different types of objects populating this domain can be distinguished in terms of different types of either mereological or mereotopological part-whole structures. 
	
	I believe that at this point it is useful to readdress the classical paradox in \ref{ex:ring-paradox} which led \citeauthor{link1983logical} to distinguish between the domain of entities and the domain of portions of matter.
	
	\ex. English \citep[adapted]{link1983logical}\label{ex:ring-paradox}
	\a. This is a gold ring.\label{ex:ring-paradox-ring}
	\b. The ring is new but the gold is old.\label{ex:ring-paradox-gold}
	
	 Assume that \ref{ex:ring-paradox-ring} points at a ring which was recently forged from some old Egyptian gold. If the ring is nothing more than a sum of portions of gold, it is surprising that \ref{ex:ring-paradox-gold} is not contradictory. To the contrary, in the described scenario it is true. In other words, the problem concerns the fact that an object appears to have different properties than the sum of its parts, but in mereological terms the two are identical (see also \citealt{rothstein2010counting,rothstein2017semantics} for discussion).
	
	Mereotopology offers a new explanation for the paradox. What is crucial here is that the ring is not just the sum of portions of gold it is made of. Rather, to be perceived as a ring an entity needs to remain in a particular spatial configuration, i.e., its parts need to be related by a topological relation. Therefore, the reason why \ref{ex:ring-paradox-gold} is not contradictory is that it says that while the parts are old, the configuration is new. For instance, this intuition could be captured by attributing different topological properties to the definite descriptions \textit{the ring} and \textit{the gold}. As a result, even if the extensions of these two expressions involved absolute material overlap, e.g., imagine that at a certain point in time the ring consisted of all of the gold, the two entities could still be differentiated since only the referent of \textit{the ring} would be conceptualized as an integrated whole.\footnote{Of course, it would be a topic for another study to work out what these properties are.}
	
	To sum up, the view developed here offers a radically different perspective on nominal semantics compared to standard approaches which utilize the notion of atomicity. Given the above discussion, I conclude that it turns out to be advantageous both from the conceptual and practical point of view because it better captures intuitions concerning the nature of individuals as well as allows for simpler models with less domains. Soon we will see that since it is free from the deficits of atomicity-based approaches discussed in  \sectref{sec:doing-without-atoms}, it also proves significantly favorable in modeling subatomic quantification. In the following section, I will make a first step towards developing an analysis along those lines. In particular, I will introduce the theoretical background for a unified analysis of numerical expressions.  
	
	\section{Measure functions}\label{sec:measure-functions}
	
	Following \citet{krifka1989nominal,krifka1995common}, I model quantification over both parts and wholes in numeral and measure constructions in terms of extensive measure functions, i.e., operations that directly relate entities to numbers. The core mechanics behind a measure function is that when applied to a plurality of individuals or quantity of substance it maps it onto a real number corresponding to the number of individuals or units making up the plurality or quantity in question. Such operations need to satisfy what we expect from counting, i.e., they need to be additive. The definition in \ref{ex:mf-additive} ensures that no entity will be counted twice.
	
	\ex. Additive measure function\\
	$\mu$ is an additive measure function iff it satisfies the following requirement\\
	$\forall x\forall y\big[\neg\cnst{o}(x,y) \rightarrow [\mu(x \sqcup y) = \mu(x) + \mu(y)]\big]$\\
	(The measure of a sum consisting of non-overlapping parts equals the arithmetic sum of the measures of parts.)\label{ex:mf-additive}
	
	Assuming supplementation for the join operation as defined in \ref{ex:supplementation} in  \sectref{sec:mereology} and repeated here in \ref{ex:mf-supplementation} alongside the Archimedean property of measure functions, see \ref{ex:mf-archmimedean-property}, it can be demonstrated that an additive Archimedean measure function is monotonic, see \ref{ex:mf-monotonicity}  \citep[see][]{schwarzschild2002grammar}.
	
	\ex. Supplementation\\
	$\forall x\forall y[x \sqsubset y \rightarrow \exists z[z \sqsubseteq y \wedge \neg\cnst{o}(z,x)]]$\\
	(Whenever a thing has a proper part, it has more than one.)\label{ex:mf-supplementation}
	
	\ex. Archimedean property\\
	$\forall x\forall y[\mu(x) > 0 \wedge y \sqsubseteq x \rightarrow \mu(y) > 0]$\\
	(If a measure of a thing is greater than zero, then a measure of its part is also greater than zero.)\label{ex:mf-archmimedean-property}
	
	\ex. Monotonicity\\
	$\forall x\forall y[x \sqsubset y \rightarrow \mu(x) < \mu(y)]$\\
	(A measure of a whole is greater than a measure of its proper part.)\label{ex:mf-monotonicity}
	
	Examples of extensive additive measure functions include \textsc{liter}, \textsc{meter}, and \textsc{calorie}, whereas measure functions such as \textsc{degree-celsius} and \textsc{carat} are non-additive.\footnote{This distinction corresponds to the monotonicity/non-monotonicity distinction in \citet{schwarzschild2002grammar}. In the following part of the chapter, I will focus only on additive/monotonic measure functions.} For instance, for an entity $x$ the measure function \textsc{liter} yields a number of liters corresponding to the volume of that entity. In other words, it measures the quantity of space occupied by its argument in units of liters. Analogously, \textsc{meter} and \textsc{calorie} return the number of meters and calories of a particular entity, respectively. 
	
	\subsection{Counting operation}\label{sec:counting-operation}
	
	With this machinery in place, we can now start developing an account for counting. Let us begin with an operation to which I will refer as $\mu_\#$. Its formal definition is provided in \ref{ex:mf-hash}.\footnote{The symbol $\#$ in the subscript indicates quantification in terms of number of elements.} 
	
	\ex. Measure function $\mu_\#$\\
	$\mu_\#$ is an additive measure function standardized by the following requirement\\
	$\forall x[\mu_\#(x) = 1$ iff $\cnst{mssc}(x)]$\label{ex:mf-hash}
	
	In  \sectref{sec:general-counting-principles}, I demonstrated that counting and measuring differ in that the former is topology-sensitive, whereas the latter is not. Despite this fact, for convenience I will talk about measure functions when referring to both operations yielding a cardinality of objects as well as operations giving a measure in terms of, e.g., volume. Given the definitions introduced above, the counting operation $\mu_\#$ defined in \ref{ex:mf-hash} is a kind of measure function. Nevertheless, it is crucial to emphasize that it is a special kind of measure function. Similarly to other additive measure functions, it cares about non-overlap, but in addition it is also sensitive to the mereotopological structure of the entities that it applies to. In particular, as specified in \ref{ex:mf-hash} it assigns the value 1 exclusively to \cnst{mssc} individuals.\footnote{In the original theory of \citet{krifka1989nominal} the equivalent of $\mu_\#$ measures in terms of natural units, hence the \cnst{nu} operation. On the other hand, \citet{krifka1995common} introduces the \cnst{ou} and \cnst{ku} operations (for `object unit' and `kind unit', respectively), whereas in newer versions of the theory \citep[e.g.,][]{krifka2007masses} the \cnst{ac} measure function (for `atomic count') is defined to count atoms.} This fact makes it distinctively different from measure functions such as \textsc{liter} which ignore whether their arguments refer to integrated entities that come in one piece.
	
	To see the measure function $\mu_\#$ at work, let us consider the example provided in \ref{ex:counting-via-hash}. Assume that there are two distinct \cnst{mssc} individuals $a$ and $b$, i.e., $a \neq b$. In such a case, after we apply $\mu_\#$ to the sum of $a$ and $b$, the returned value will be 2 which of course is a correct result.
	
	\ex. Counting via $\mu_\#$\\
	$\cnst{mssc}(a) \wedge \cnst{mssc}(b) \wedge \neg\cnst{o}(a,b) \rightarrow \mu_\#(a \sqcup b) = 2$\\
	(If both $a$ and $b$ are \cnst{mssc} and they do not overlap, i.e., they are not identical, then the measure function $\mu_\#$ applied to the sum of $a$ and $b$ yields the number 2.)\label{ex:counting-via-hash}
	
	The measure function $\mu_\#$ seems to do what we would expect an operation intended to capture the core intuitions concerning counting to do. However, there is yet another improvement to be implemented. As we discussed in  \sectref{sec:integrated-wholes}, it is desirable to relativize \cnst{mssc} to a particular property in order to avoid ending up with the single integrated whole corresponding to the entire material universe. Therefore, a proper device intended to account for counting should also be relativized to a property so that it does not count absolutely \cnst{mssc} entities but rather objects that are \cnst{mssc} with respect to a particular characteristic, e.g., apples. This can be ensured by introducing a new operation to which I will refer as $\#$. As can be seen in the definition in \ref{ex:mf-hash(P)}, it is similar to $\mu_\#$ in that the notion of \cnst{mssc} plays a crucial role here. 
	
	\ex. Measure function $\#(P)$\\
	$\#(P)$ is an additive measure function standardized by the following requirement\\
	$\forall P\forall x[\#(P)(x) = 1$ iff $\cnst{mssc}(P)(x)]$\label{ex:mf-hash(P)}
	
	Importantly, $\#$ differs from $\mu_\#$ in that it takes a property $P$ and yields a measure function that returns a number of \cnst{mssc} individuals relative to $P$. For instance, assume there are two distinct apples $a$ and $b$. Then, $\#$ can yield a measure function that will count entities that are \cnst{mssc} with respect to the property of being an apple. As a result, we obtain the number 2.
	
	Before we move on to introducing another measure function that will prove useful for modeling multiplier phrases, let us discuss the mechanism of contextual conditioning which will allow us to account for proportional partitive words.
	
	\subsection{Contextual conditioning}\label{sec:contextual-conditioning}
	
	The evidence presented in Chapter \ref{ch:partitives-and-part-whole-structures} shows that cross-linguistically partitive words can appear both in entity and mass partitives, on the one hand, and set partitives, on the other. In general, in the first two cases the partitive word quantifies over portions of matter constituting either a particular individual or a quantity of substance, whereas in the third situation it quantifies over integrated wholes making up a plurality. To account for this double compatibility a procedure is required which will determine what is to be measured or counted when. For this purpose, I build on \citeposst{bale_barner2009interpretation} proposal that assumes a generalized context-dependent measure function $\mu$. Such an approach posits a mechanism of contextual conditioning along the lines defined in \ref{ex:contextual-conditioning-mf}. The core of the idea is that particular measure functions are ordered and depending on a context those that are ranked higher in the series are favored over those that are ranked lower. 
	
	\ex. Contextual conditioning \citep[p. 245; adapted]{bale_barner2009interpretation}\\
	$\mu$ is interpreted as one of the measure functions $m_z$ in the series\\ $\langle m_1, m_2, m_3\dots m_n\rangle$ such that the argument for $\mu$ is in the range of $m_z$;\\ furthermore, contextually $m_z$ is preferred to $m_y$ if $z<y$\label{ex:contextual-conditioning-mf}
	
	\citeauthor{bale_barner2009interpretation} use the mechanism in \ref{ex:contextual-conditioning-mf} to account for various meanings of \textit{more} observed in comparative constructions such as \textit{X has more NP than Y}. Specifically, depending on the NP comparison may be specified in terms of cardinality, volume, intensity etc. For the purpose of this study, I will make use of the very same mechanism for a procedure determining that partitive words can quantify over portions of matter in entity and mass partitives and over individuals in set partitives. In particular, I propose a partial ordering of measure functions, as provided in \ref{ex:partial-ordering-mf}. 
	\ex. Partial ordering of measure functions\\
	$m_1 = \#(P) < m_n \in \textsc{volume}$\\
	where \textsc{volume} is a set of measure functions measuring entities in terms of volume including \textsc{m$^3$}, \textsc{liter}, \textsc{pint} etc.\label{ex:partial-ordering-mf} 
	
	The first measure function in the series, i.e., the one ranked highest, is the measure function $\#(P)$ which is devised to count integrated wholes. I remain agnostic with respect to the exact position in the ranking other measure functions occupy, but for our purposes establishing the ordering between $\#(P)$ and measure functions measuring in terms of volume is certainly sufficient.
	
	To anticipate, the contextually conditioned generalized measure function $\mu$ can cover the meanings of partitive words in entity and mass partitives, on the one hand, and set partitives, on the other. In particular, $\mu$ is interpreted as a measure function quantifying over units of volume when the partitive word combines with a DP involving a singular count noun which refers to an individual as well as with a DP with a mass term referring to a quantity of substance. On the other hand, when the partitive word combines with a DP comprising a regular plural noun, $\mu$ is interpreted an operation counting individuated integrated wholes. This is because whenever the context allows for it, $\#(P)$ is preferred over the functions from the set \textsc{volume}, e.g., \textsc{cm$^3$}. An additional positive consequence of the adopted mechanism of contextual conditioning is that it allows for quantification over individuals in partitives with object mass nouns in the embedded DP \citep[see][]{barner_snedeker2005quantity,bale_barner2009interpretation}.
	
	\subsection{Counting essential parts}\label{sec:counting-essential-parts}
	
	The final component of the analysis related to measure functions has to do with modeling multipliers. As we saw in Chapter \ref{ch:multipliers}, multipliers quantify over cognitively salient parts of individuals. In order to capture this intuition, I propose an additional measure function for which I will use the $\boxplus$ symbol.\footnote{There is a mnemonic here. The $\boxplus$ symbol is a square constituted by four smaller squares each of which is a part of the whole with comparable properties. Hence, $\boxplus$ represents a type of object multiplier phrases usually refer to.} This operation takes a property and yields an additive measure function $\boxplus(P)$ which counts essential parts of a whole. The formal definition of $\boxplus(P)$ is provided in \ref{ex:mf-boxplus(P)}.
	
	\ex. Measure function $\boxplus(P)$\\
    $\forall P\forall x[\cnst{mssc}(P)(x) \rightarrow \boxplus(P)(x) = \#(\lambda y[y \sqsubseteq x \wedge \cnst{essential}(P)(y)])]$\label{ex:mf-boxplus(P)}
	
	Notice that the $\boxplus(P)$ measure function quantifies only over parts of \cnst{mssc} individuals as ensured by the antecedent in the requirement. Furthermore, the consequent imposes that $\boxplus(P)$ employs $\#$ in order to count the number of parts of $x$ that are essential for an ascription of $P$ to some entity. Then, since there are no other $P$ objects overlapping with $x$, these must be parts that are essential for $P$ to hold of $x$. The parthood relation captures the intuition that only entities within the part-whole structure of an object are subject to quantification and the use of improper parthood, i.e., $\sqsubseteq$, rather than $\sqsubset$ is to account for cases of homogeneous entities without any distinguishable salient parts such as individuated portions of coffee or martini. In particular, one might want to talk about a single portion as opposed to a double portion rather than to unrelated two portions, since there is a clear semantic contrast between \textit{single coffee} $\sim$ \textit{double coffee} $\sim$ \textit{two coffees}.\footnote{It is possible that \ref{ex:mf-boxplus(P)} is too strong to account for examples like \textit{double coffee} since it requires conceptualizing the referents of such expressions as \textsc{mssc} individuals. However, I will leave this issue for future research.}
	
	As witnessed by comparing \ref{ex:mf-hash(P)} and \ref{ex:mf-boxplus(P)}, the main difference between $\#(P)$ and $\boxplus(P)$ lies in what type of objects they map onto numbers. The former simply associates any \cnst{mssc} object with a number, whereas the latter counts only what I refer to as an essential part of an entity, i.e., a cognitively salient element. The notion \cnst{essential} in \ref{ex:mf-boxplus(P)} is relativized to a property and can be defined as in \ref{ex:essential-parts}. 
	
	\ex. Essential parts\\
	$\cnst{essential}(P)$ is true of an \cnst{mssc} part of an entity in the extension of $P$ that is essential for that entity to be considered as having a property $P$\label{ex:essential-parts}
	
	I keep the definition in \ref{ex:essential-parts} somewhat vague on purpose since what is considered essential for an entity to be perceived as having a certain property is often not entirely clear and may differ with respect to a particular context. For instance, as we discussed in Chapter \ref{ch:multipliers} multiplier phrases such as \textit{double garage} can denote objects of a different internal structure as long as they can hold two vehicles. Nevertheless, one important feature of an essential part is that it is an integrated and recognizable entity within a whole that can be easily individuated against other parts.
	
	An example of an essential part would be a patty in a hamburger. Intuitively, it makes the most important element of the whole sandwich. Out of the blue, it seems that the crucial part of a bed is an area covered with a mattress on which a person can sleep. Finally, in a natural context the most salient piece of, say, a shotgun is its barrel since it guarantees functionality as well as constitutes a significant portion of the weapon. However, notice that the broad definition in \ref{ex:essential-parts} also covers self-sufficient parts, i.e., parts that have a property comparable, i.e., very similar or identical, to the property of a whole. I argue that a self-sufficient part can be considered an extreme case of an essential part. If for some reason it is considered that what is essential cannot be reduced to one particular element, then $\cnst{essential}(P)$ yields an \cnst{mssc} part that resembles the object it is a part of. For instance, let us once again consider the case of \textit{double crown} discussed in Chapter \ref{ch:multipliers}. Although there can be multiple cognitively salient parts within the make-up of a crown, e.g., jewels, orbs, crosses, or \textit{fleurs-de-lis}, intuitively no such element is essential for a thing to be considered a crown. On the other hand, an object such as the Pschent or papal tiara counts as a double or triple crown, respectively, because it comprises parts with comparable properties as the whole. Similar cases arguably involve objects such as doors and layers, thus a double door involves two connected door leaves and a double layer is an entity consisting of two merged layers. For the reasons discussed above, I leave the issue concerning when an essential part is a self-sufficient part vague and dependent on how a particular artifact is conceptualized.
	
	All things considered, given \ref{ex:mf-boxplus(P)} and \ref{ex:essential-parts} the $\boxplus(P)$ measure function returns a number of essential parts in each object being a member of a given set. The properties of such a device might help us to account for the interpretation of multiplier phrases. With the set-up and all the relevant ingredients necessary to account for the numerical expressions I am interested here in place, let us now propose the semantics for cardinals and multipliers that will allow us to account for the observed phenomena in count explicit and proportional partitives as well as in multiplier phrases.
	
	\section{Numerical expressions}\label{sec:numerical-expressions}
	
	In the history of formal semantics spanning over almost half a century there were not that many expressions that received as much attention as numerals. Though a lot of revealing research has been done in this field (see, e.g., \citealt{horn1972semantic,barwise_cooper1981generalized,scha1981distributive,krifka1999least,landman2003predicate,landman2004indefinites,hofweber2005number,ionin_matushansky2006composition,geurts2006take,nouwen2010two,kennedy2013scalar,rothstein2013fregean,rothstein2017semantics} to name just a few influential studies), in my opinion there is a shared misconception concerning numerals since they are commonly treated as simplex, i.e., non-decomposable, expressions.\footnote{A notable exception is the theory of \citet{kennedy2013scalar} who proposes that numerals are generalized quantifiers over degrees and discusses their possible semantic decomposition.} On the other hand, it has been observed that in a language such as English cardinals have multiple uses and therefore it is misleading to search for ``the'' meaning of numerals but rather it is more appropriate to talk about multiple meanings associated with such expressions \citep[e.g.,][]{bultinck2005numerous,geurts2006take}. The interplay of these two aspects results in treating cardinal numerals either as ambiguous or postulating some shifting mechanism to account for at least some of the uses. However, I believe there are good reasons to claim that the assumption that numerals are simplex is incorrect. 
	
	In my opinion, the prevailing view most probably stems from the limited scope of the mainstream research on numerals. In particular, most work on numerical quantification has been done on the basis of English data and since English lacks a rich morphology, certain semantic distinctions are not marked by means of different formal exponents. However, even in English there are complex numericals such as \textit{twice}, \textit{twofold}, \textit{twosome}, and adjectival \textit{two-time}, but this set of expressions for the most part has been surprisingly neglected in the study of numerical quantification. On the other hand, evidence from languages that have a broader repertoire of derivational means suggests that numerals are in fact compositional. For instance, recent research on cardinals as well as derivationally more complex numerical expressions in Slavic and Semitic indicates that different numerical affixes correspond to distinct semantic operations responsible for deriving various meanings (see, e.g., \citealt{docekal2012atoms,docekal2013numerals,docekal_wagiel2018event} for Czech, \citealt{wagiel2014boys,wagiel2015sums,wagiel-toappear-grammatical} for Polish, \citealt{khrizman2015cardinal} for Russian and \citealt{fassi_fehri2016semantic,fassi_fehri2018constructing} for Arabic). 
	
	In the following sections, I posit a compositional account for numerical expressions in natural language. I assume that a prerequisite for such an approach is that it accounts for the three counting principles discussed in  \sectref{sec:general-counting-principles}. In particular, it needs to be guaranteed that the numerical expression ignores overlapping, non-integrated, and non-maximal entities. In other words, it needs to be devised in such a way that it counts right.
	
	\subsection{Numeral roots}\label{sec:numeral-roots}
	
	Many numerical expressions in numerous languages are morphologically complex and can be sequenced into separate morphemes including numeral roots and various additional affixes. In this study, I follow my previous work in arguing that it is plausible to treat numeral roots as names of natural numbers, i.e., expressions referring to abstract objects of type $n$, see \ref{ex:numeral-root-general} \citep{wagiel2015sums,wagiel2020entities,wagiel2020several,wagiel-toappear-grammatical}. For instance, the English root $\sqrt{\textit{tw}}$ as in, e.g., \textit{two} and \textit{twenty}, simply names the integer 2, as specified in \ref{ex:numeral-root-tw}.  
	
	\ex. Numeral root\label{ex:numeral-root}
	\a. $\llbracket \sqrt{\text{Numeral}}\rrbracket = n$\label{ex:numeral-root-general}
	\b. $\llbracket \sqrt{\text{tw}}\rrbracket = 2$\label{ex:numeral-root-tw}
	
	Furthermore, I assume that virtually all meanings numerical expressions can have can be derived from this basic semantics by application of additional operations encoded by different types of classifiers. Those classifiers can be either introduced overtly by a particular morpheme or silent. For the sake of coherence and in order to stay as close to the main argument as possible, in the following sections I will focus on derivations of only two kinds of numerical expressions, namely cardinals and multipliers.\footnote{For some preliminary proposals of how to treat compositionally other complex numerical expressions in Slavic, see \citet{wagiel2015sums,wagiel2020entities,wagiel2020several}.}
	
	\subsection{Cardinals}\label{sec:cardinals}
	
	I propose that cardinals are born as singular terms, specifically as names of numbers at type $n$. This meaning is preserved in contexts clearly calling for numerical arguments such as the mathematical statements in \ref{ex:cardinals-singular-terms} (see \citealt{rothstein2017semantics} for extensive discussion of such environments). However, the singular term semantics can also serve as a basis to derive expressions that are more abstract than number-denoting.  
	
	\ex. English \citep{rothstein2013fregean}\label{ex:cardinals-singular-terms}
	\a. Two plus two is four.
	\b. Two is the only even prime number.
	
	Since this study concerns subatomic quantification, in the following part of this section I will restrict my proposal to the use of numerals as nominal modifiers. In particular, I posit that in attributive position the meaning associated with the numeral root is shifted to an expression of type $\langle\langle e,t\rangle,\langle e,t\rangle\rangle$, i.e., the type of a predicate modifier \citep[see][]{ionin_matushansky2006composition}. I postulate that this shift is performed by a classifier element which I will refer to as CL$_\#$ (see also \citealt{sudo2016semantic} for a similar proposal based on Japanese data). As suggested by the abbreviation and provided in \ref{ex:classifier-hash}, CL$_\#$ is an expression of type $\langle n,\langle\langle e,t\rangle,\langle e,t\rangle\rangle\rangle$, i.e., a function from numbers to predicate modifiers, that introduces the $\#$ operation. 
	
	\ex. Classifier$_\#$\\
	$\llbracket \text{CL}_\#\rrbracket = \lambda n \lambda P : \cnst{pmssc}(P)\ \lambda x [\text{*}P(x) \wedge \#(P)(x) = n]$\label{ex:classifier-hash}
	
	Notice, however, that triggering $\#$ is not the only thing CL$_\#$ does. In order to ensure that counting goes right, the classifier imposes a particular constraint on predicates it can select, specifically via the individuation presupposition it selects only predicates that denote exclusively \cnst{mssc} individuals. Consequently, the resulting cardinal will not be compatible with mass nouns or pluralia tantum since they do not fulfill this requirement. Notice also that a predicate $P$ in the first conjunct is pluralized by the classical * operator. Such a semantics guarantees that an integer $n$ will be associated with the number of \cnst{mssc} individuals making up a plurality in the extension of $P$, see \ref{ex:mf-hash}--\ref{ex:mf-hash(P)}. The use of * rather than the strict pluralization operator ${}^+$ postulated for semantic plurals, see \ref{ex:strict-pluralization}, is motivated by the existence of the numeral \textit{one} which requires singularities.
	
	Given the semantics of CL$_\#$ proposed above, it is straightforward how particular cardinals are derived. When the classifier combines with a numeral root, e.g., $\sqrt{\textit{tw}}$, the number variable in \ref{ex:classifier-hash} is saturated by a particular integer and this gives rise to a full-fledged attributive numeral as exemplified in \ref{ex:cardinal-numeral}. 
	
	\ex. Cardinal numeral\\
	$\llbracket \text{two} \rrbracket = \llbracket \text{CL}_\# \rrbracket(\llbracket \sqrt{\text{tw}} \rrbracket) = \lambda P : \cnst{pmssc}(P)\ \lambda x [\text{*}P(x) \wedge \#(P)(x) = 2]$\label{ex:cardinal-numeral}
	
	Given the formula above, \textit{two} in a phrase such as \textit{two apples} takes a property of being an apple and yields a set of pairs of objects that are \cnst{mssc} with respect to that property, i.e., a set of pairs of apples. Admittedly, the way it is ensured in \ref{ex:cardinal-numeral} that cardinals do not combine with mass terms is by stipulating the individuation presupposition $\cnst{pmssc}(P)$. Notice, however, that in Chapter \ref{ch:conceptual-background} I argued extensively that a requirement like this stems form the general counting principles, which in turn arguably result from the way human cognition works. Therefore, the proposed denotation of cardinals seems to relate to more general facts regarding human mind, which I believe provides a valuable perspective. 
	
	In general, I take \ref{ex:cardinal-numeral} to be a welcome result. However, an important comment is required. For the semantics in \ref{ex:cardinal-numeral} to work, one needs to assume that the morphological plural on the noun that the cardinal modifies is not interpreted semantically since otherwise the pluralized property would fail to satisfy the individuation presupposition. In other words, the source of plurality is the * operator introduced by CL$_\#$, whereas the number marker on the noun is a mere agreement plural triggered syntactically with no semantic contribution \citep[see][]{krifka1989nominal,ionin_matushansky2006composition}. At first sight this might seem implausible, but there are well-documented cases of mismatches between the plural as a morpho-syntactic category and the semantic notion of plurality, as in \ref{ex:zero-1.0} (see \citealt{nouwen2016plurality} for an overview). Furthermore, there are examples of languages such as Finnish, Hungarian, and Turkish that despite having plural morphology do not employ it in numeral phrases or employ it only with numerals higher than `four', as in Russian and BCS.\footnote{I simplify here for the sake of brevity since, e.g., Turkish was argued to have semantically number-neural singular count nouns \citep[e.g.,][]{gorgulu2012semantics}. For a general discussion on different strategies languages use to combine cardinals with nouns and a proposal of an alternative approach see \citet{bale_gagnon_khanjian2011crosslinguistic}. One thing that should be mentioned as a potential problem for the proposed approach concerns collective modification below the numeral, e.g., \textit{two parallel streets} and \textit{two similar objects} (though see \citealt{ionin_matushansky2018cardinals} for a proposed solution). However, since this study does not focus on such expressions, I leave this issue for future research.}
	
	\ex. English (\citealt{krifka1989nominal}; adapted)\label{ex:zero-1.0}
	\a. zero \{cows/*cow\}\label{ex:zero}
	\b. 1.0 \{cows/*cow\}\label{ex:1.0}
	
	However, it is important to emphasize that if it turned out that the assumption that cardinals combine with semantically singular nouns is untenable, this would not be a serious problem for the proposed account. An alternative would be to rework the semantics of CL$_\#$ in \ref{ex:classifier-hash} on the assumption of defining a function that applies to sets of pluralities and returns the set of \cnst{mssc} individuals relative to that set, which would be the mereotopological equivalent of \citeposst{chierchia2010mass} approach. Without going into details, let us call such an operation \cnst{obj}. Applying it would give us an alternative denotation of CL$_\#$, as provided in \ref{ex:classifier-hash-alt}, and consequently, no need for the number vacuousness assumption in numeral phrases.\footnote{I would like to thank Peter Sutton for this suggestion.} 
	
	\ex. Classifier$_\#$ (alternative)\\
	$\llbracket \text{CL}_\#\rrbracket = \lambda n \lambda P : \cnst{pmssc}(P)\ \lambda x [P(x) \wedge \#\big(\cnst{obj}(P)\big)(x) = n]$\label{ex:classifier-hash-alt}
	
	The treatment postulated in \ref{ex:cardinal-numeral} builds on a well-established way of thinking about cardinals since the idea that numerals are in fact names of numbers dates back to \citet{frege1884grundlagen}. An early formal semantic account for cardinals proposed by \citet{scha1981distributive} assumes a complex syntactic structure for numerical determiners decomposing them into the bottom-most Number projection which is interpreted in terms of reference to an integer and the Numeral and Det projections above which transform a singular term into a predicate based upon that integer and generalized quantifier, respectively. Finally, in a recent theory of counting and measuring by \citet{rothstein2012numericals,rothstein2013fregean,rothstein2017semantics} it has been acknowledged that when used as singular terms in examples such as \ref{ex:cardinals-singular-terms} cardinals seem to refer to abstract objects rather than anything else and this meaning should be accounted for by a special shifting operation.
	
	Apart from providing an explanation for why cardinals are incompatible with mass terms, an important advantage of the semantics developed here is that it also allows for a unified analysis of numerical expressions in classifier and non-classifier languages.\footnote{See, e.g., \citet{kobuchi-philip2006identity} and \citet{wagiel_caha2020universal} for some arguments why this is desirable.}  In the next section, I will extend the proposed classifier semantics for numerals to multipliers.
	
	\subsection{Multipliers}\label{sec:multipliers}\largerpage[1.5]

	As we saw in Chapter \ref{ch:multipliers}, multipliers are similar to cardinals in that they both do what a counting expression does, i.e. establish a one-to-one correspondence between entities and natural numbers. On the other hand, the two expressions in question differ significantly in that the former do not count whole individuals but rather quantify over certain parts thereof. Furthermore, they appear to be merged with the modified noun before other quantifiers, and thus their scope is significantly limited since they allow for modification by numerals and by the universal quantifier. In other words, they do not infer a plurality of objects, instead they imply that an object involves a plurality of parts having particular features. These facts suggest that the analysis of multipliers should resemble that of cardinals and at the same time differ in such a way that would capture the relevant distinctions.
	
	The morphological evidence from Slavic presented in Chapter \ref{ch:multipliers} strongly suggests that just like cardinals multipliers should be treated as complex compositional expressions. Just like in the previous section, I propose that multipliers are decomposable into two elements. In a language such as Polish, the first is the name of a number corresponding to a numeral root present in the morphological make-up of the multiplier, whereas the second is a special classifier introduced by the multiplicative affix. I will refer to that classifier as CL$_\boxplus$. As indicated by the superscript, this element introduces the $\boxplus$ operation defined in \ref{ex:mf-boxplus(P)} in \sectref{sec:counting-essential-parts} above, see \ref{ex:classifier-boxplus}. 
	
	\ex. Classifier$_\boxplus$\\
	$\llbracket \text{CL}_\boxplus\rrbracket = \lambda n \lambda P : \cnst{pmssc}(P)\ \lambda x [P(x) \wedge \boxplus(P)(x) = n]$\label{ex:classifier-boxplus}
	
	As CL$_\#$, CL$_\boxplus$ is a function from numbers to predicate modifiers, i.e., an expression of type $\langle n,\langle\langle e,t\rangle,\langle e,t\rangle\rangle\rangle$. It also involves the individuation presupposition which accounts for the fact that multipliers can only modify expressions denoting \cnst{mssc} individuals such as count nouns and mass terms coerced by the Universal Packager. However, unlike CL$_\#$ the element CL$_\boxplus$ does not pluralize predicates it applies to. This is an important difference between the two classifiers since it explains why multiplier phrases do not denote pluralities and allow for numeral modification.
	
	Again, it is straightforward how to get a multiplier by combining the meaning of a numeral root, see \ref{ex:numeral-root}, with the semantics of CL$_\boxplus$ in \ref{ex:classifier-boxplus} above. For instance, consider the denotation of the Polish multiplier \textit{podwójny} `double' in \ref{ex:multiplier}.  
	
	\ex. Polish multiplier\\
	$\llbracket \text{podwójny} \rrbracket = \llbracket \text{CL}_\boxplus \rrbracket(\llbracket \sqrt{\text{dw}} \rrbracket) = \lambda P : \cnst{pmssc}(P)\ \lambda x [P(x) \wedge \boxplus(P)(x) = 2]$\label{ex:multiplier}
	
	The number referred to by the numeral root $\sqrt{\textit{dw}}$ simply saturates the first argument slot in \ref{ex:classifier-boxplus} and as a result an expression of type $\langle\langle e,t\rangle,\langle e,t\rangle\rangle$ is obtained, see \ref{ex:multiplier}. The derived predicate modifier selects for a set that has only \cnst{mssc} individuals in their extensions and returns a subset of such \cnst{mssc} individuals that have two essential parts. Such a predicate can further serve as an argument for another quantificational expression, e.g., a cardinal numeral.
	
	The proposed semantics appears to successfully account for the subset of data I focus on here, namely multiplier phrases involving concrete nouns. Furthermore, as indicated in  \sectref{sec:less-obvious-cases} it also offers a way of thinking about examples I do not deal with here. In particular, after extending the ontology with primitive semantic types for events and roles a very similar mechanism can be proposed to account for the meaning of NPs like \textit{double murder} and \textit{double president}. Such considerations, however, reach far beyond the main topic of this study and I will refrain from discussing an exact implementation of the idea. Instead, in the following section I will present a brief overview of how the proposed semantics fits into a bigger typological picture and how cross-linguistic data supports the treatment of cardinals and multipliers as complex expressions.

	\subsection{Cross-linguistic support}\label{sec:cross-linguistic-support}
	
    	The structure of cardinal numerals proposed above is further supported by several empirical findings made in linguistic typology. First, it has been observed that cross-linguistically numerals and classifiers are always adjacent \citep{greenberg1972numeral}.  \tabref{tab:orderings-numerals-classifiers} provides the four attested orderings of the numeral, the classifier, and the noun, out of six logically possible patterns. Notice that in the missing two the noun would separate the numeral and classifier.

	\begin{table}[h]
		\centering
		\begin{tabular}{ll}
			\lsptoprule
			\textsc{language} & \textsc{ordering} \\ \midrule
			Vietnamese                                          & [\textsc{num}-\textsc{clf}]-\textsc{n} \\
			Thai                                                & \textsc{n}-[\textsc{num}-\textsc{clf}] \\
			Ibibio                                              & [\textsc{clf}-\textsc{num}]-\textsc{n} \\
			Bodo                                                & \textsc{n}-[\textsc{clf}-\textsc{num}] \\ \lspbottomrule
		\end{tabular}
        \caption{Attested relative orderings of numerals and classifiers \citep{greenberg1972numeral}}
		\label{tab:orderings-numerals-classifiers}
	\end{table}

	Second, in classifier languages classifiers are often suffixes on numerals as witnessed, e.g., in \ref{ex:classifiers-suffixes-japanese} and \ref{ex:classifiers-suffixes-yucuna} (see \citealt{aikhenvald2000classifiers} for more data).
	
	\ex. Japanese \citep{sudo2016semantic}\label{ex:classifiers-suffixes-japanese}
	\bg.[] ichi-rin-no hana\\
	one-\textsc{clf}-\textsc{gen} flower\\
	`one flower'
	
	\ex. Yucuna \citep[p. 106]{aikhenvald2000classifiers}\label{ex:classifiers-suffixes-yucuna}
	\bg.[] pajluhua-na yahui\\
	one-\textsc{clf} dog\\
	`one dog'
	
	Furthermore, intriguing data from partly classifier languages such as Mi'gmaq and Chol show that it can be the case that in a single language some cardinals require classifiers in order to combine with nouns whereas others do not, see \ref{ex:partly-classifier-languages} \citep{bale_coon2014classifiers}. This further suggests that the classifier makes a constituent with the numeral and compensates its semantic deficits.
	
	\ex. Mi'gmaq \citep{bale_coon2014classifiers}\label{ex:partly-classifier-languages}
	\ag. na'n \minsp{(*} te's)-ijig ji'nm-ug\\
	five {} \textsc{clf}-\textsc{agr} man-\textsc{pl}\\
	`five men'
	\bg. asugom \minsp{*(} te's)-ijig ji'nm-ug\\
	six {} \textsc{clf}-\textsc{agr} man-\textsc{pl}\\
	`six men'

	Finally, it has been observed that in genetically and typologically diverse languages there exists an alternation between attributive and specialized so-called counting cardinals, i.e., referential expressions that cannot be used as modifiers, see \tabref{tab:attributive-counting-numerals-across-languages} (\citealt{hurford1998interaction,hurford2001languages}; see also \citealt{wagiel_caha2020universal}). This fact demonstrates that in some cases an exponent of the name of a number can be formally distinct from a related attributive expression. All things considered, I take the data discussed here to indicate that the analysis postulated for cardinals is on the right track.

	\begin{table}[h]
		\centering
		\begin{tabular}{llll}
			\lsptoprule
			\textsc{language}  & \textsc{number} & \textsc{attributive} & \textsc{counting} \\ \midrule
			German    & 2      & zwei                 & zwo             \\
			Maltese   & 2      & żewg                 & tnejn            \\
			Chinese   & 2      & li{\v{a}}ng          & {\`{e}}r         \\
			Hungarian & 2      & két                  & kettö            \\
			Basque    & 2      & bi                   & biga             \\ \lspbottomrule
		\end{tabular}
        \caption{Attributive and counting numerals across languages \citep{hurford2001languages}}
		\label{tab:attributive-counting-numerals-across-languages}
	\end{table}

	Moreover, the semantics postulated in \ref{ex:multiplier} can be further supported by the fact that in many languages the numeral root can be easily distinguished in the morphological make-up of multipliers. As we saw in Chapter \ref{ch:multipliers}, Slavic multipliers are morphologically complex expressions derived from numeral roots by means of affixation which I argue encodes the classifier element CL$_\boxplus$. Though many European languages borrowed their multipliers from Latin, the existence of multiplicative affixes is by no means a Slavic idiosyncrasy since similar patterns can be observed in Baltic and Finnic, as witnessed by the correspondences given in \tabref{tab:multipliers-across-languages}.

	\begin{table}[h]
		\centering
		\begin{tabular}{llll}
			\lsptoprule
			\textsc{language}  & \textsc{number} & \textsc{cardinal} & \textsc{multiplier} \\ \midrule
			Russian   & 2 & dva      & dvojnoj                \\
			Lithuanian    & 2 & du      & dvigubas                   \\
			Finnish & 2 & kaksi      & kaksinkertainen                  \\ \lspbottomrule
		\end{tabular}
        \caption{Multipliers across languages}
		\label{tab:multipliers-across-languages}
	\end{table}

	I conclude that the proposed account is plausible both in terms of what multipliers mean, i.e., what kind of semantic effects they give rise to, and what they often look like, i.e., their morphological complexity in some languages suggests their semantic decomposability. The system developed here is advantageous in that the same compositional mechanism allows for accounting for different numerical expressions. Though in this study I focus only on cardinals and multipliers, after extending it with additional classifiers a number of other numerals can be derived by its means (see \citealt{docekal2012atoms,docekal2013numerals,wagiel2014boys,wagiel2015sums,wagiel2020several,docekal_wagiel2018event} for a discussion of other types of complex numerals in Slavic). This unified treatment also explains why only one ordering of cardinals and multipliers is possible. However, before we discuss how a multiplier phrase modified by a cardinal is composed step by step from the pieces defined here, let us introduce additional tools to account for explicit and set partitives.
	
	\section{Partitives}\label{sec:partitives-analysis}
	
	The main data set presented in Chapter \ref{ch:partitives-and-part-whole-structures} concerned different types of partitives in various languages. In this study, I do not attempt to provide a detailed analysis of the syntax-semantics interface regarding partitive constructions. Rather, based on the cross-linguistic evidence I focus on general issues concerning the interaction between partitivity, topological sensitivity and countability. In particular, I will postulate a minimal set of ingredients necessary for an attempt to explain what happens in count explicit partitives and different types of proportional partitives involving topology-neutral as well as topology-sensitive partitive words. However, before I propose particular denotations of selected expressions of that type, there are several semantic components to comment on.
	
	\subsection{Partitivity}\label{sec:partitivity}
	
	An important component of partitive constructions considers what kind of DP can be combined with the partitive word. As already mentioned in  \sectref{sec:partitives}, partitives are subject to the so-called Partitive Constraint which disallows certain expressions from the embedded DP \citep[e.g.,][]{jackendoff1977x-bar,selkirk1977some,barwise_cooper1981generalized,ladusaw1982semantic}. I will follow here the semantic reanalysis of the Partitive Constraint, as proposed by \citet{de_hoop1997semantic}, see \ref{ex:partitive-constraint}, which states that the whole downstairs DP is an expression of type $e$.
	
	\ex. Partitive Constraint (\citealt{de_hoop1997semantic}; adapted)\\
	In a partitive, the embedded DP must be entity-denoting, i.e., definite or specific.\label{ex:partitive-constraint}

	The formulation in \ref{ex:partitive-constraint} means that the semantics of the element combining with the embedded DP needs to be devised in such a way that it does not take sets as its input but rather individual things. Consequently, since \citet{barker1998partitives} it is commonly assumed that the preposition \textit{of} in English partitives is a function from entities to sets of their parts, i.e., type $\langle e,\langle e,t\rangle\rangle$, expressing partialness by means of proper parthood.\footnote{But see \citet{ionin_matushansky_ruys2006parts} and \citet{marty2017implicatures} for a treatment based on improper parthood.} Specifically, the semantics of partitive \textit{of} establishes the $\sqsubset$ relation between entities stating that each member of the output set is a proper part of the input entity. 
	
	However, as we saw in Chapter \ref{ch:partitives-and-part-whole-structures} and \ref{ch:exploring-topological-sensitivity}, there are languages that do not employ prepositions in partitive constructions. For instance, Russian, Hungarian, and Basque make use of case marking instead, whereas Japanese and Chinese can wield a number of strategies to express partitivity including classifier structures. Moreover, in German and Brazilian Portuguese NPs can be modified by adjectival half-words in order to yield proportional meaning. These facts seem to suggest that from a cross-linguistic perspective the mapping between semantics and morphology is not that obvious. After all, given that $\sqsubset$ is introduced by some other element, it is not clear what the exact semantic contribution of part-words is since they seem to do precisely the same thing. Therefore, for convenience I incorporate the general partitive semantics into the meaning of partitive words themselves, see \ref{ex:partitivity}.\largerpage[-1]
	
	\ex. Partitivity\\
	$\llbracket \text{PART}\rrbracket = \lambda y \lambda x [x \sqsubset y]$\label{ex:partitivity}
	
	Since I will focus here on the composition of German and Polish partitives, the reason behind this move is mainly to avoid an unnecessary detour into the semantics of case.\footnote{See \citet{kagan2013semantics} for a detailed discussion of some aspects of the meaning of genitive case in Russian.} However, if required the distribution of particular components to be discussed below could be rearranged so that, say, the genitive case in a language such as German and Polish is associated with the semantics in \ref{ex:partitivity}, whereas particular partitive words contribute additional meaning or are treated as void.
	
	Notice that since I assume only one domain, i.e., the domain of entities, with one parthood relation $\sqsubseteq$ defined over it, the partitivity semantics in \ref{ex:partitivity} simply employs the basic mereological notion of proper parthood, which unlike $\sqsubseteq$ is not reflexive. As a result, there is no need to distinguish between entity and set partitives in terms of different parthood relations. Consequently, it is straightforward to expect that in general partitive words are able to combine both with expressions referring to singularities and with those denoting pluralities. As witnessed in  \sectref{sec:the-analogy}, this kind of behavior is exactly what is observed cross-linguistically.
	
	As we saw, the semantic re-formulation of the Partitive Constraint in \ref{ex:partitive-constraint} urges us to ensure that the downstairs DP position in partitives is occupied by an expression of type $e$. Putting aside the straightforward case of proper names, in the next sections I will discuss two mechanisms how to guarantee adequate extensions of definites and specific indefinites, namely maximization and choice functions, respectively.
	
	\subsection{Maximization}\label{sec:maximization}
	
	In order to account for definite DPs, I assume the standard maximization operation \cnst{max}, see \ref{ex:maximization-operator} \citep[cf.][]{sharvy1980more,link1983logical}, which yields the unique maximal entity in the denotation of a predicate.\footnote{In the literature, the symbol $\sigma$ is often used instead of \cnst{max}.} 
	
	\ex. Maximization operator\\
	$\cnst{max}(P) = \iota x[P(x) \wedge \forall y[P(y) \rightarrow y \sqsubseteq x]]$\\
	defined if there is a unique $x$ in $P$ of which all other  things are a part; otherwise undefined\\
	(If defined, $\cnst{max}(P)$ is the element in $P$ which all other things in $P$ are part of.)\label{ex:maximization-operator}
	
    In other words, \cnst{max} maps any set which may include singularities, pluralities, or both onto the element that all other elements in that set are part of. In cases it selects a pluralized predicate, it returns the supremum, i.e., the largest plurality in the set. On the other hand, if \cnst{max} is applied to a singular predicate, it returns a singularity, provided that that predicate denotes a singleton, i.e., there is only one relevant entity in a given context. Maximality thus allows for a unified account for both singular and plural definite DPs.

	In article languages such as English and German, the \cnst{max} operator is introduced by the definite article, e.g., \textit{the} and \textit{der}/\textit{die}/\textit{das}/\textit{die}, respectively. On the other hand, for article-less languages such as Polish I assume a DP projection over NP headed by a silent determiner that can be interpreted as a definite article \citep[e.g.,][]{veselovska1995phrasal,progovac1998determiner,rutkowski2002noun}. The semantics for such an overt or covert definite DP head is given in \ref{ex:definite-article}.
	
	\ex. Definite article\\
	$\llbracket \text{DEF}\rrbracket = \lambda P[\cnst{max}(P)]$\label{ex:definite-article}
	
	Let us now contemplate how the proposed semantics delivers the meaning of definite singulars and plurals. Assuming a model for \ref{ex:definite-singular-dp} with a unique apple $a$, the noun \textit{apple} denotes a singleton set involving $a$, see \ref{ex:definite-singular-sg}. After the NP is merged with the determiner the resulting DP \textit{the apple} in \ref{ex:definite-singular-def-sg} simply refers to $a$. On the other hand, in a model with three apples $a$, $b$, and $c$ the pluralized noun \textit{apples} denotes the set of all pluralities derived from $a$, $b$, and $c$, see \ref{ex:definite-plural-pl}, whereas the definite \textit{the apples} has only the maximal sum in its extension, see \ref{ex:definite-plural-def-pl}. This shows that the mechanics of \cnst{max} captures the intuition that singular definites such as \textit{the apple} are true of a unique apple, whereas a definite plural DP like \textit{the apples} corresponds to a plurality consisting of all the apples.
	
	\ex. Definite singular DP\label{ex:definite-singular-dp}
	\a. $\llbracket \text{apple}\rrbracket = \{a\}$\label{ex:definite-singular-sg}
	\b. $\llbracket \text{DEF apple}\rrbracket = a$\label{ex:definite-singular-def-sg}

	\ex. Definite plural DP\label{ex:definite-plural-dp}
	\a. $\llbracket \text{PL apple}\rrbracket = \{a\sqcup b, a\sqcup c, b\sqcup c, a\sqcup b\sqcup c\}$\label{ex:definite-plural-pl}
	\b. $\llbracket \text{DEF [PL apple]}\rrbracket = a\sqcup b\sqcup c$\label{ex:definite-plural-def-pl}
	
	The \cnst{max} operation allows us to model the meaning of both singular and plural definites in partitives in accordance with the Partitive Constraint. In order to account for specific indefinites, I assume choice functions which will be briefly described in the next section.
	
	\subsection{Choice functions}\label{sec:choice-functions}
	
	A choice function is an operator selecting a member from a set \citep{reinhart1997quantifier,kratzer1998scope}. On the adopted view, the choice function variable remains free at the level of semantic composition and its value is provided by the context, i.e., the choice of a particular member of a given set may vary depending on extra-linguistic circumstances. In general, choice functions can be applied to any kind of set. As a result, they have been widely used in cross-linguistic research on different types of specific indefinites \citep[see, e.g.,][]{matthewson1998interpretation,kratzer_shimoyama2002indeterminate,alonso-ovalle_menendez-benito2003some,yanovich2005choice,wagiel2020several}. In this study, however, I will limit my focus to the domain of entities. In particular, I embrace an approach that a choice function \cnst{ch} over entities is an expression of type $\langle\langle e, t\rangle, e\rangle$ such that when it is applied to a non-empty set of entities $P$, it yields a specific entity from $P$ relative to a particular context, see \ref{ex:choice-function}.
	
	\ex. Choice function\\
	For any $\cnst{ch}_{\langle\langle e, t\rangle, e\rangle}$ and $P_{\langle e, t\rangle}$, $\cnst{ch}$ is a choice function if $P(\cnst{ch}(P)) = \cnst{true}$\label{ex:choice-function}
	
	I assume that choice functions are introduced by a specificity element which can either be expressed formally or be silent. For instance, in Russian it is encoded by a special suffix on indefinite pronouns \citep[see, e.g.,][]{yanovich2005choice} but lack any exponent in case an indefinite DP including a common noun gets a specific interpretation. Either way, the specificity element takes a predicate as its input and applies \cnst{ch} on that predicate as specified in \ref{ex:specificity-element}.
	
	\ex. Specificity element\\
	$\llbracket \text{SPEC}\rrbracket = \lambda P [\cnst{ch}(P)]$\label{ex:specificity-element}
	
	Provided a model with three apples $a$, $b$, and $c$, the denotations of singulars and plurals are exactly as such cases discussed above, see \ref{ex:specific-indefinite-singular-sg} and \ref{ex:specific-indefinite-plural-pl}, whereas the extensions of specific indefinites can be as follows. For instance, in a given context \textit{apple} can be interpreted as referring to a certain apple, say, $b$ in \ref{ex:specific-indefinite-singular-spec-sg}. Analogously, the plural \textit{apples} can be understood to designate a specific though indefinite plurality of apples, e.g., $a\sqcup b$ in \ref{ex:specific-indefinite-singular-spec-sg}.\footnote{Of course, SPEC and DEF extend also to mass nouns.}
	
	\ex. Specific indefinite singular DP\label{ex:specific-indefinite-singular}
	\a. $\llbracket \text{apple}\rrbracket = \{a,b,c\}$\label{ex:specific-indefinite-singular-sg}
	\b. $\llbracket \text{SPEC apple}\rrbracket = b$ (in a certain context)\label{ex:specific-indefinite-singular-spec-sg}
	
	\ex. Specific indefinite plural DP\label{ex:specific-indefinite-plural}
	\a. $\llbracket \text{PL apple}\rrbracket = \{a\sqcup b, a\sqcup c, b\sqcup c, a\sqcup b\sqcup c\}$\label{ex:specific-indefinite-plural-pl}
	\b. $\llbracket \text{SPEC [PL apple]}\rrbracket = a\sqcup b$ (in a certain context)\label{ex:specific-indefinite-plural-spec-pl}
	
	The discussed mechanism will allow us to account for specific DPs in partitives. However, for the sake of brevity, in the following sections I will focus only on constructions involving definite DPs downstairs. For the treatment of specific indefinites in the embedded DP, it is sufficient to substitute the choice function for the maximization operator.
	
	\subsection{Topology-sensitive transitivity}\label{sec:set-partitive-constraint}
	
	Having provided means to ensure that the complement of the partitive component is of type $e$, let us return to the meaning of partitive words in different constructions. One of the observations discussed in Chapter \ref{ch:partitives-and-part-whole-structures} concerned the cross-linguistic analogy between entity and set partitives. In particular, in languages such as German and Polish the same partitive word can combine with both singular and plural DPs. In the first case, it gives rise to a part-of-a-singularity reading, whereas in the latter configuration a part-of-a-plurality interpretation is inferred. This suggests that either the meanings of partitive words are general enough to cover both readings or such expressions are ambiguous cross-linguistically. 
	
	Furthermore, in  \sectref{sec:ambiguity-or-indeterminacy} I showed that at least in German it is implausible to postulate ambiguity since partitive words can give rise to both discussed readings simultaneously in partitives with coordinated structures involving singular and plural DPs. Therefore, the data suggest that the proper treatment should be formulated in terms of one general meaning. Given that no sorted domains are postulated and singular individuals are distinguished from plural entities in terms of the distinction between mereotopological as opposed to purely mereological configurations, respectively, the partitivity semantics introduced in \ref{ex:partitivity} is exactly what we need. However, utilizing one unified parthood relation for singularities and pluralities in both entity and set partitives turns out to be problematic in one respect.
	
	The issue is the following. If there is only one domain with one parthood relation, there is nothing that stops set partitives from referring to an arbitrary material part of any singular entity making up a plurality. For instance, given the plurality of apples $a\sqcup b\sqcup c\sqcup d$ a part-word taking that plurality as an argument could return some part of, say, $b$. However, this is not how set partitives are interpreted. To the contrary, under ordinary circumstances they denote a set of individual parts of a plurality, e.g., in the above example the extension could include $a\sqcup b$, $b\sqcup c$, $a\sqcup b\sqcup c$ etc. In order to exclude material part interpretations, I postulate a special condition, which I call \textsc{topology-sensitive transitivity}, as formulated in \ref{ex:set-partitive-contraint}, where $\cnst{imssc}(x)$ refers \cnst{mssc} individuals, see \ref{ex:mssc-individuals}.\footnote{I would like to thank Adam Przepiórkowski and Peter Sutton for their suggestions concerning the formalization of the condition.} 
	
	\ex. Topology-sensitive transitivity\\
	$\forall x \forall y \forall z[ [x \sqsubset y \wedge y \sqsubset z] \rightarrow [x \sqsubset z \leftrightarrow \neg\cnst{imssc}(y)] ]$\label{ex:set-partitive-contraint}
	
	\ex. \cnst{mssc} individual\\
	$\cnst{imssc}(x) \leftrightarrow \exists P[\cnst{mssc}(P)(x)]$\label{ex:mssc-individuals}
	
	The condition in \ref{ex:set-partitive-contraint} has the following effect. If $y$ is not an \cnst{mssc} individual, then transitivity works as usual. However, if $y$ is an \textsc{mssc} individual, then transitivity is blocked. Hence, topology-sensitive transitivity ensures that an entity that is part of a plurality is either an \cnst{mssc} individual or a sum of such individuals. In other words, \ref{ex:set-partitive-contraint} excludes all material parts from the denotations of set partitives.
	
	It is worth noting that an important consequence of adopting \ref{ex:set-partitive-contraint} is that in a certain sense we have abandoned the unconditioned transitivity of parthood. Though it is definitely not a mainstream assumption, the behavior of explicit set partitives described above suggests that it is empirically correct as far as natural language is concerned (see also \citealt{moltmann1997parts} for related arguments). Moreover, given the evidence for the significance of topological notions for grammar examined in Chapter \ref{ch:exploring-topological-sensitivity} it would not be surprising that topological sensitivity applies also to some primitive notions governing linguistic part-whole structures. 
	
	I conclude that topology-sensitive transitivity guarantees the proper interpretation of partitives involving plural DPs. It seems that postulating this restriction is a necessary move for an analysis of partitive constructions building on the unified parthood relation. However, I believe that it is not a big price to pay compared to the gains such an approach can offer. In the next section, I will readdress the issue of partitioning discussed briefly in  \sectref{sec:doing-without-atoms}, in order to provide a background for the analysis of particular types of partitive words. The notion of partition will turn out useful in modeling continuous parts.
	
	\subsection{Partitions}\label{sec:partitions}\largerpage
	
	As the evidence discussed in this study indicates, quantification over both wholes and parts can only operate on sets consisting of disjoint members, i.e., entities that do not share a part. However, since there are numerous ways how to divide a thing into parts, many of such divisions would yield multiple entities that do overlap. But such divisions are undesirable since only those sets can serve as the domain of subatomic quantification that comprise separate, i.e., non-overlapping, continuous parts. For instance, if one counted one half of a teddy bear as one, another half as two, and its head and right paw as three and four, one would count some things multiple times. By doing so they would violate one of the crucial counting principles and, as a result, fail to establish a one-to-one correspondence between discrete entities and numbers. In order to avoid that, a mechanism ruling out undesirable divisions is required. Such a mechanism is supposed to guarantee that the domain of quantification does not include overlapping parts. In other words, what we want is an operation that will be able to carve up a whole into distinct separate entities. For this purpose, I will make use of a device well-known in the study of pluralities, namely partitions \citep[see, e.g,][]{schwarzschild1996pluralities,chierchia2010mass}. 
	
	The partitioning operation $\pi$ is a function of type $\langle\langle e, t\rangle, \langle e, t\rangle\rangle$ which selects a set of entities, i.e., a predicate $P$, and yields its subset $\pi(P)$, i.e., a set of those elements in $P$ that do not overlap. Given the standard mereological definition of overlap, see \ref{ex:overlap}, partitioning imposes a condition that no two members of a partition $\pi(P)$ share a part. The formal definition of the partitioning operation $\pi$ is provided in \ref{ex:partition}.\footnote{Notice that this definition diverges from what is usually expected from partitions, i.e., not only non-overlap but also complete cover. In that sense, $\pi$ does not deliver proper partitions, as defined in set theory. However, for the sake of convenience I will continue to use the term following some other authors \citep[e.g.,][]{scontras2014semantics}.}

	\ex. Partitioning function $\pi$\\
	For all properties $P$, $\pi(P)$ is a subset of $P$ such that for any $x$ and $y$ in $\pi(P)$ the following requirement is satisfied\\
	$\neg \exists z[z \sqsubseteq x \wedge z \sqsubseteq y]$\\
	(No two members of a partition overlap.)\label{ex:partition}

	To see how partitioning works let us consider what happens when $\pi$ is applied to the set A in \ref{ex:partitioning-set}. As a result, we get \ref{ex:partitioning-partition} which includes only those members of A that do not share a part. On the other hand, the set B in \ref{ex:partitioning-non-partition} does not satisfy the condition defined in \ref{ex:partition} since $a\sqcup b$ and $a\sqcup c$ share a part, namely $a$. Therefore, although B is a subset of A, it is not its partition.\largerpage
	
	\ex. Partitioning in context\label{ex:partitioning}
	\a. $\text{A} = \{a \sqcup b, a \sqcup c, a \sqcup d, b \sqcup c, b \sqcup d, c \sqcup d, a \sqcup b \sqcup c, a \sqcup b \sqcup d, a \sqcup c \sqcup d, \\ b \sqcup c \sqcup d, a \sqcup b \sqcup c \sqcup d\}$\label{ex:partitioning-set}
	\b. $\pi(\text{A}) = \{a\sqcup b, c\sqcup d\}$\label{ex:partitioning-partition}
	\b. $\text{B} = \{a\sqcup b, a\sqcup c\}$\label{ex:partitioning-non-partition}
	
	It is important to emphasize that \ref{ex:partitioning-partition} represents only one of numerous possible partitions. For many sets there are multiple subsets that would satisfy the non-overlap requirement after applying $\pi$. Depending on the context some of those possible partitions may be considered more natural than others since a particular arrangement of parts making up an entity may suggest a particular division. For instance, a teddy bear consists of cognitively salient elements recognizable as legs, paws, a head, and a body. The fact that we perceive the whole in such a structured way may result in partitioning the set of material parts of the toy in question accordingly.  However, different partitions are not excluded, e.g., into its left and right half or its brown and white part. Crucially, once a particular partition is established in a given context it is fixed. This seems to be a welcome result since it satisfies the intuition that, despite multiple possible divisions of a whole into parts, once a choice concerning a particular division has been made, it cannot be altered. Figuratively speaking, one cannot change what is being counted while already counting.
	
	\subsection{Individuation}\label{sec:individuation}
	
	The partitioning function $\pi$ guarantees that a reference set is free of entities sharing a part. This is an important improvement since it accounts for the ban of overlap in the domain of quantification. However, as discussed in Chapter \ref{ch:conceptual-background} there are two more principles that need to be accommodated within a theory of countability, namely the conditions of integrity and maximality. Since in this study I am primarily concerned with subatomic quantification, I will focus on what makes some part \textit{a} part, i.e., on how to model the contrast between an arbitrary material portion of a whole and a stable integrated and recognizable entity within a whole. In order to do so, we need a more fine-grained device than mere partitioning. For this purpose, I assume an additional semantic element which I will refer to as the individuating element (IND). This operation is a function of type $\langle\langle e,t\rangle,\langle e,t\rangle\rangle$ which selects a set of entities and returns a subset of the input consisting of non-overlapping integrated objects. This can be achieved by incorporating into the semantics of IND two components, specifically the partitioning function $\pi$ and the mereotopological \cnst{mssc} relation. To be precise, I propose that IND introduces the \cnst{mssc} restriction relative to a partitioned property, as defined in \ref{ex:individuating-element}. Consequently, IND turns a set consisting of different types of entities into a set including only \cnst{mssc} individuals.\footnote{One might wonder whether the combination of contextual partitioning would suffice with a weaker notion of \cnst{ssc}, recall \ref{ex:strongly-self-connected} in \sectref{sec:integrated-wholes}. There are two reasons why I use \cnst{mssc} instead. First, the aim of this study is to provide a unified account for both quantification over wholes and subatomic quantification. More importantly, however, though in fact partitioning plus \cnst{ssc} yields a set of disjoint entities, i.e., non-overlapping parts of the whole, it does not guarantee that that one cannot count \cnst{ssc} parts of those entities, i.e., parts of those non-overlapping parts, as one. Arguably, in many cases it would not matter, but there are some cases where it would, see \sectref{sec:general-counting-principles}.}
	
	\ex. Individuating element\\
	$\llbracket \text{IND}\rrbracket = \lambda P \lambda x[\cnst{mssc}\big(\pi(P)\big)(x)]$\label{ex:individuating-element}
	
	In general, an idea of an individuating operation is by no means new. In fact, theories postulating similar operators in order to account for the mass/count distinction date back at least to \citet{sharvy1979indeterminacy}. For instance, one of the most prominent and innovative systems was developed by \citet{borer2005name} who proposes that count syntax involves a classifier element, which is absent in mass expressions. It is introduced by a null head merged in the region between the NP and DP and expresses division by partitioning the  domain of reference encoded by the nominal stem. In a similar vein, \citet{bale_barner2009interpretation} posit a count noun functional head interpreted as a function from unindividuated denotations to individuated denotations. Admittedly, a closely related idea to IND was proposed by \citet[p. 97]{scontras2014semantics} for atomizers such as \textit{grain} in, e.g., \textit{grain of rice}. On his view, partitioning is defined in such a way that it applies to a kind and the result is a set of \cnst{mssc} instantiations of that kind. However, what is new about the notion I propose is that, as we will see, it allows for the individuation of material parts of individuals, something that \citeauthor{scontras2014semantics}'s system fails to account for, unless postulating kinds corresponding to denotations of particular partitive words, which seems both conceptually undesirable and empirically inappropriate. Given the linguistic relevance of subatomic quantification discussed in Chapters \ref{ch:partitives-and-part-whole-structures}, \ref{ch:exploring-topological-sensitivity}, and \ref{ch:multipliers} I argue that the semantics proposed in \ref{ex:individuating-element} proves to be more advantageous.
	
	Despite the fact that in many cases the individuating element is silent, postulating it is well motivated by the restricted meaning of partitive words in count explicit and proportional partitives, as we saw from the contrasts provided in Chapter \ref{ch:partitives-and-part-whole-structures}. Though bare topology-neutral expressions of that kind can denote both contiguous and discontiguous entities such as scattered portions of matter and arbitrary sums of individuals, the moment they combine with the cardinal numeral they can refer exclusively to entities perceived as integrated objects. Furthermore, as discussed in Chapter \ref{ch:exploring-topological-sensitivity}, Polish and German provide evidence that there are expressions that encode the individuating element formally. In particular, as witnessed by the morphological complexity of Polish topology-sensitive partitive words such as \textit{połówka} and \textit{ćwiartka} as opposed to topology-neutral \textit{połowa} and \textit{ćwierć} there are good reasons to postulate that the suffix \textit{-k-} is the exponent of IND. Similarly, German stacked \textit{eine} in proportional partitives of the the type \textit{die eine Hälfte} DP also appears to express the semantics in \ref{ex:individuating-element}.
	
	The next section will provide the semantics of some of the various types of partitive words discussed in Chapters \ref{ch:partitives-and-part-whole-structures} and \ref{ch:exploring-topological-sensitivity}.
	
	\subsection{Partitive words}\label{sec:partitive-words}
	
	Having defined the generalized contextually conditioned measure functions $\mu$ in \ref{ex:contextual-conditioning-mf} as well as the individuating element IND in \ref{ex:individuating-element}, we are ready to propose the semantics of both topology-neutral and topology-sensitive partitive words. Let us start with German \textit{Teil}. As discussed in \sectref{sec:partitivity}, for convenience I assume that parthood or, more precisely, proper parthood is encoded in the semantics of partitive words. Thus, the meaning of \textit{Teil} is given in \ref{ex:german-topology-neutral-teil}. 
	
	\ex. German topology-neutral part-word \textit{Teil}\\
	$\llbracket \text{Teil}\rrbracket = \lambda y \lambda x [x \sqsubset y]$\label{ex:german-topology-neutral-teil}
	
	\begin{sloppypar}
	The purely mereological nature of $\sqsubset$ makes it a topology-neutral expression which can denote a set of scattered entities, arbitrary sums, or integrated individuals. This characteristic corresponds to the compatibility of \textit{Teil} with explicit entity, set, and mass partitives reported in \sectref{sec:the-analogy}. On the other hand, the constraint on set partitives postulated in \ref{ex:set-partitive-contraint} guarantees that when it selects a plural genitive DP, it does not yield an arbitrary material part or parts of an \cnst{mssc} individual making up a plurality. However, when \textit{Teil} combines with a count singular DP, the resulting set can consist of various parts of the relevant object since the proposed semantics bears no topological commitments with respect to the to spatial arrangement of portions of matter the whole phrase denotes. In other words, explicit entity partitives can refer to continuous as well as discontinuous parts.
	\end{sloppypar}
	
	However, as we saw in \sectref{sec:partitivity-and-countability}, in count explicit partitives the numeral in accordance with the counting principle of integrity can quantify only over contiguous portions of matter. This means that in such a syntactic configuration the possible denotation of \textit{Teil} is significantly restricted. In other words, the part-word that is normally topology-neutral is shifted into a topologically sensitive expression that yields a set of those parts of the relevant whole that are spatially integrated. I propose that this shift is performed by a silent IND element. As specified in \ref{ex:individuating-element}, IND applies \cnst{mssc} to a partition of the set denoted by the input predicate. As a result, all of those parts denoted by the phrase combining with \textit{Teil} that are overlapping as well as discontinuous are excluded from the extension of the shifted expression. All the details concerning composition will be discussed in \sectref{sec:composition}.
	
	Let us now turn to different types of Polish half-words. In general, the semantics of topology-neutral \textit{połowa} given in \ref{ex:polish-topology-neutral-polowa} is quite similar to the one of \textit{Teil}.  
	
	\ex. Polish topology-neutral half-word \textit{połowa}\\
	$\llbracket \text{połowa}\rrbracket = \lambda y \lambda x [x \sqsubset y \wedge \mu(x) \approx \mu(y) \times 0.5]$\label{ex:polish-topology-neutral-polowa}
	
	Both \textit{Teil} and \textit{połowa} are expressions of type $\langle e,\langle e,t\rangle\rangle$, i.e., functions that take an entity and yield a set of its parts. Furthermore, as in the case of \textit{Teil} nothing restricts \textit{połowa} from selecting any particular type of thing, i.e., it is perfectly compatible with individuals, scattered entities, and pluralities of objects. Finally, the resulting set involves both continuous and discontinuous halves. The main difference between the two is that \textit{połowa} is a proportional expression which denotes only those parts that constitute a certain portion of a whole. This is captured by the generalized measure function $\mu$. Given the mechanism of contextual conditioning defined in \ref{ex:contextual-conditioning-mf}, $\mu$ returns an integer corresponding to a number of individuals if its argument is a plurality, whereas when applied to a singular individual or scattered entity, it yields a value corresponding to the measure of its volume in some contextually salient units, e.g., cm$^3$. Since as discussed in \sectref{sec:inherent-vagueness} half-words are inherently vague and can indicate parts that are either smaller or greater than an exact half, the approximately equal relation $\approx$ is preferred over $=$ in the formula in \ref{ex:polish-topology-neutral-polowa}. Overall, what \textit{połowa} does is that it selects an entity of any sort and returns a set of its continuous and discontinuous parts such that they constitute approximately 50\% of the total cardinality or volume of that entity.\largerpage
	
	Compared to \textit{połowa}, its topology-sensitive counterpart \textit{pół} fails to combine with predicates denoting scattered entities and arbitrary sums of individuals. Given the framework developed here, it is possible to account for its distributional restrictions discussed in \sectref{sec:polish-half-words} in terms of mereotopological distinctions. In particular, I propose that the semantics of \textit{pół}, see \ref{ex:polish-topology-sensitive-pol}, is almost identical to \ref{ex:polish-topology-neutral-polowa} with the exception that it involves a special selectional requirement which I refer to as the \textsc{integrated individual presupposition}. Specifically, \ref{ex:polish-topology-sensitive-pol} presupposes that the first argument, i.e., $y$, is an individual that is \cnst{mssc} relative to some property, recall \ref{ex:mssc-individuals}. 
	
	\ex. Polish topology-sensitive half-word \textit{pół}\\
	$\llbracket \text{pół}\rrbracket = \lambda y : \cnst{imssc}(y)\ \lambda x [x \sqsubset y \wedge \mu(x) \approx \mu(y) \times 0.5]$\label{ex:polish-topology-sensitive-pol}

	The integrated individual presupposition in \ref{ex:polish-topology-sensitive-pol} explains why \textit{pół} does not combine with plural and mass DPs, since these expressions do not refer to \cnst{mssc} individuals in their extensions. Therefore, \textit{pół} can only head proportional entity partitives and after it takes an integrated object, it yields a set that consists of various continuous as well as discontinuous halves of that object.
	
	Finally, it is time to account for the most formally as well as semantically complex of the Polish half-words, namely \textit{połówka}. As we saw in \sectref{sec:polish-half-words}, it is a morphologically complex expression involving a special suffix \textit{-k-} shared with a number of derived partitive words in Polish such as \textit{ćwiartka} `quarter' and \textit{cząstka} `part'. All of those expressions are similar to \textit{pół} in that they require \cnst{mssc} individuals as their inputs. However, they differ in that their denotation consist only of continuous parts. Therefore, I propose that the suffix \textit{-k-} in fact introduces the semantics of the IND element specified in \ref{ex:individuating-element}, see \ref{ex:polish-individuating-suffix}. Thus, the half-word \textit{połówka} is a complex expression derived from the meaning encoded in a simpler topology-sensitive partitive word.
	
	\ex. Polish individuating suffix \textit{-k-}\\
	$\llbracket \text{-k-}\rrbracket = \llbracket \text{IND}\rrbracket = \lambda P \lambda x[\cnst{mssc}\big(\pi(P)\big)(x)]$\label{ex:polish-individuating-suffix}
	
	Given both the morphological complexity as well as semantic behavior of \textit{połówka} discussed in detail in  \sectref{sec:continuous-discontinuous-parts}, I postulate that proportional partitive constructions headed by \textit{połówka} are derived from partitives headed by \textit{pół} via the derivational suffix \textit{-k-}.\footnote{More precisely, I assume that in terms of morphology \textit{połówka} is derived from \textit{pół} rather than from \textit{połowa} and that the element \textit{-ów-} in \textit{połówka} is semantically vacuous. Such an analysis seems to be corroborated by the fact that the morpheme \textit{-ów-} appears sometimes as a linking element in Polish deadjectival nominalizations such as \textit{złoty} `golden; złoty (Polish currency)'~$\sim$~\textit{złotówka} `1-złoty coin'. Notice the ungrammaticality of \textit{*złotowy}.} The resulting expression is a topology-sensitive half-word with the semantics in \ref{ex:polish-topology-sensitive-polowka}.
	
	\ex. Polish topology-sensitive half-word \textit{połówka}\\
	$\llbracket \text{połówka DP}\rrbracket = \llbracket \text{-k-}\rrbracket(\llbracket \text{pół DP}\rrbracket)$\label{ex:polish-topology-sensitive-polowka}
	
	The meaning defined above ensures that \textit{połówka} not only selects entities that are \cnst{mssc} individuals but also denotes only parts that are spatially continuous. The first aspect is ensured by the integrated individual presupposition encoded in \textit{pół}, whereas the semantics of \textit{-k-} guarantees the latter. In particular, the partitioning function $\pi$ provides a contextually salient partition of the set delivered by the partitive headed by \textit{pół}. Thus, the \cnst{mssc} operation is relativized to a property from whose denotation all overlapping parts have been eliminated as a result of partitioning. Ultimately, the resulting set involves only disjoint integrated parts, exactly as desired.
	
	At first blush, it might seem odd to propose that the suffix \textit{-k-} combines with the entire proportional partitive DP rather than with the sole half-word to which it linearly attaches. However, there is an independent reason suggesting that such an analysis is on the right track. The suffix \textit{-k-} in Polish is polyfunctional and appears also in a number of other morphological environments. Among others, it is used to derive group numerals, see \ref{ex:polish-numerals} \citep[see][]{wagiel2015sums}.\footnote{Notice that the attachment of \textit{-k-} triggers standard Polish morphophonological alternations represented orthographically in \ref{ex:polish-numerals} as \textit{ę}~:~\textit{ą} and \textit{ć}~:~\textit{t}.} 
	
	\ex. Polish\label{ex:polish-numerals}
	\ag. pięć dziewczyn\\
	five girls\textsc{.gen}\\
	`five girls'\label{ex:polish-numerals-basic}
	\bg. piątka dziewczyn\\
	five\textsc{.group} girls\textsc{.gen}\\
	`(group of) five girls'\label{ex:polish-numerals-group}
	
	The form \textit{pięć} in \ref{ex:polish-numerals-basic} is a basic cardinal numeral meaning simply `five'. Attaching the suffix \textit{-k-} in \ref{ex:polish-numerals-group} changes it to \textit{piątka}, which is an obligatorily collective numerical expression probably best paraphrased as `group of five'. Slavic collectivizers of this type are typically analyzed via some sort of group-forming operation, e.g., \citeposst{landman1989groupsi} $\uparrow$ operator \citep[see, e.g.][]{docekal2012atoms,docekal2013numerals,wagiel2015sums}. In particular, $\uparrow$ applies to a sum of entities and returns a group, i.e., a plural individual consisting of those entities that behaves as a unit in its own right. This, however, requires the meaning of the suffix \textit{-k-} to take the entire NP as its argument rather than the numeral, contrary to what the linear morpheme ordering suggests. In other words, in order to interpret \ref{ex:polish-numerals-group} as `group of five girls', \textit{-k-} needs to be interpreted above `five girls'.

	For the reason described above, in the next section I will follow the account in \ref{ex:polish-topology-sensitive-polowka}. However, it is important to emphasize that having the suffix \textit{-k-} interpreted above the partitive DP is not a crucial assumption of the proposed analysis. Rather its key component is the association of \textit{-k-} with the semantics of the individuating element IND, as defined in \ref{ex:individuating-element}. For instance, an alternative to the treatment in \ref{ex:polish-individuating-suffix} would be to make the semantics of \textit{-k-} more complex so that it can take \textit{pół} as its argument. For instance, consider \ref{ex:polish-individuating-suffix-alt}.\footnote{I would like to thank Peter Sutton for suggesting this.}
	
	\ex. Polish individuating suffix \textit{-k-} (alternative)\\
	$\llbracket \text{-k-}\rrbracket = \lambda R \lambda y : \cnst{imssc}(y)\ \lambda x[\llbracket \text{IND}\rrbracket(R(y))(x)] = \\ = \lambda R \lambda y : \cnst{imssc}(y)\ \lambda x[\cnst{mssc}\big(\pi(R(y))\big)(x)]$\label{ex:polish-individuating-suffix-alt}
	
	The semantics in \ref{ex:polish-individuating-suffix-alt} enables the meaning of IND to compose with a function from entities to predicates. If $R$ is an expression of type $\langle e,\langle e,t \rangle\rangle$, we obtain an alternative analysis of \textit{połówka} in \ref{ex:polish-topology-sensitive-polowka-alt}, which does not rely on \textit{-k-} being interpreted above the partitive DP and allows it to combine with \textit{pół} directly.
	
    \ex. Polish topology-sensitive half-word \textit{połówka} (alternative)\\
	$\llbracket \text{połówka}\rrbracket = \llbracket \text{-k-}\rrbracket(\llbracket \text{pół}\rrbracket) = \\ = \lambda y : \cnst{imssc}(y)[\cnst{mssc}\big(\pi([x \sqsubset y \wedge \mu(x) \approx \mu(y) \times 0.5])\big)]$\label{ex:polish-topology-sensitive-polowka-alt}
	
	Yet another approach would be to combine \textit{-k-} with \textit{pół} via Function Composition, rather than in terms of the perhaps more familiar Function Application rule. Though I leave this option unexplored here, I believe that in principle it would not be a problem to pursue such an alternative.
	
	Having developed all the machinery necessary to account for subatomic quantification in natural language, let us see how the pieces fit together. In the next section, I will walk through the derivation of different types of partitives including German explicit and set partitives and Polish proportional partitives as well as multiplier phrases. The section will be concluded by a proposal how the system developed here could be extended to Italian count explicit partitives involving irregular plurals.
	
	\section{Composition}\label{sec:composition}
	
	In the previous sections, I provided all the ingredients necessary for an analysis of natural language expressions utilizing subatomic quantification. Now, let us see how those elements interact. I will start with the simplest cases of explicit entity and set partitives. Next, I will walk through the most complex case of count explicit partitives, which involves the interplay of almost all postulated components. Subsequently, I will discuss the difference between topology-neutral and topology-sensitive proportional partitives. Finally, I will present the derivation of multiplier phrases modified by numerals. As in the previous sections, I will illustrate how the proposed system works on the German and Polish data. For all partitives, I provide examples involving embedded definite DPs. Though for the sake of brevity I do not provide explicit derivations, I assume that for specific DPs nothing changes except from the use of a choice function, see \ref{ex:choice-function}, instead of the maximization operator. 
	
	\subsection{Explicit entity partitives}\label{sec:explicit-entity-partitives}
	
	I begin with a relatively simple example of a German explicit entity partitive as the one provided in \ref{ex:derivation-explicit-entity-partitive}. The analysis of this case will allow us to see how the essential components introduced in previous sections interact with each other and as such it will serve as a reference point for other more complex cases. 
	
	\ex. Explicit entity partitive\label{ex:derivation-explicit-entity-partitive}
	\bg.[] Teil des Apfels\\
	part the\textsc{.gen} apple\textsc{.gen}\\
	`part of the apple'
	
	All things considered, I assume that what gets interpreted at the level of semantic composition is the structure in \figref{fig:derivation-explicit-entity-partitive-tree}.\footnote{For the sake of simplicity, I ignore the genitive case marking here.} The whole phrase is an expression of type $\langle e,t\rangle$ which gets a part-of-a-singularity reading. The description of the whole derivation is provided in \ref{ex:derivation-explicit-entity-partitive-interpretation}.

\begin{figure}
\qtreecenterfalse\centering
    \Tree[.$\langle e,t\rangle$ {${\langle e,\langle e,t\rangle\rangle}$\\\textit{Teil}} [.$e$ {$\langle\langle e,t\rangle,e\rangle$\\\text{DEF}} {$\langle e,t\rangle$\\\textit{Apfel}} ] ]
    \caption{Structure of \ref{ex:derivation-explicit-entity-partitive}}
    \label{fig:derivation-explicit-entity-partitive-tree}
\end{figure}    
	
	\ex. Interpretation\label{ex:derivation-explicit-entity-partitive-interpretation}
	\a. $\llbracket \text{Apfel}\rrbracket = \lambda x [\cnst{mssc}(\textsc{apple})(x)]$\label{ex:derivation-explicit-entity-partitive-a}
	\b. $\llbracket \text{DEF}\rrbracket = \lambda P [\cnst{max}(P)]$\label{ex:derivation-explicit-entity-partitive-b}
	\b. Function Application\\
	$\llbracket \text{DEF Apfel}\rrbracket = \cnst{max}(\llbracket \text{Apfel}\rrbracket) = \\
    = \cnst{max}\big(\lambda x [\cnst{mssc}(\textsc{apple})(x)]\big)$\label{ex:derivation-explicit-entity-partitive-c}
	\b. $\llbracket \text{Teil}\rrbracket = \lambda y \lambda x [x \sqsubset y]$\label{ex:derivation-explicit-entity-partitive-d}
	\b. Function Application\\
	$\llbracket \text{Teil [DEF Apfel]}\rrbracket = \lambda x [x \sqsubset \llbracket \text{DEF Apfel}\rrbracket] =\\
	= \lambda x \big[x \sqsubset \cnst{max}\big(\lambda y [\cnst{mssc}(\textsc{apple})(y)]\big)\big]$\label{ex:derivation-explicit-entity-partitive-e}

	The predicate \textit{Apfel} in \ref{ex:derivation-explicit-entity-partitive-a} denotes a set of \cnst{mssc} individuals that have a property of being an apple. If in a given context this set is a singleton, the maximization operation \cnst{max} introduced by the definite article in \ref{ex:derivation-explicit-entity-partitive-b} selects the unique maximal element from that set, e.g., it turns $\{a\}$ into $a$, see \ref{ex:derivation-explicit-entity-partitive-c}. Subsequently, the part-word \textit{Teil} with the semantics specified in \ref{ex:derivation-explicit-entity-partitive-d} selects that entity and returns a set of its material parts. Since there are no additional restrictions, the resulting phrase represented in \ref{ex:derivation-explicit-entity-partitive-e} denotes all possible divisions of the apple in question including various overlapping continuous as well as discontinuous parts. This is exactly what we expect from the extension of a topology-neutral explicit entity partitive.
	
	\subsection{Explicit set partitives}\label{sec:explicit-set-partitives}
	
	The next example to be discussed is the German explicit set partitive \ref{ex:derivation-explicit-set-partitive}. An analysis of such a construction should account for the fact that it can get only a part-of-a-plurality interpretation. 
	
	\ex. Explicit set partitive\label{ex:derivation-explicit-set-partitive}
	\bg.[] Teil der Äpfel\\
	part the\textsc{.gen} apples\textsc{.gen}\\
	`some of the apples'

I assume the semantic structure in \figref{fig:derivation-explicit-set-partitive-tree} for the phrase in question. The only difference between \figref{fig:derivation-explicit-set-partitive-tree} and \ref{fig:derivation-explicit-entity-partitive-tree} from the previous section is that the set partitive tree involves the PL node which introduces the pluralization of the predicate whereas the entity partitive tree does not. How the structure corresponding to the phrase in \ref{ex:derivation-explicit-set-partitive} is interpreted is demonstrated in \ref{ex:derivation-explicit-set-partitive-interpretation}.

\begin{figure}
    \qtreecenterfalse\centering
    \Tree[.$\langle e,t\rangle$ {${\langle e,\langle e,t\rangle\rangle}$\\\textit{Teil}} [.$e$ {$\langle\langle e,t\rangle,e\rangle$\\\text{DEF}} [.$\langle e,t\rangle$ {$\langle\langle e,t\rangle,\langle e,t\rangle\rangle$\\\text{PL}} {$\langle e,t\rangle$\\\textit{Apfel}} ] ] ]
    \caption{Structure of \ref{ex:derivation-explicit-set-partitive}}
    \label{fig:derivation-explicit-set-partitive-tree}
\end{figure}    

	\ex. Interpretation\label{ex:derivation-explicit-set-partitive-interpretation}
	\a. $\llbracket \text{Apfel}\rrbracket = \lambda x [\cnst{mssc}(\textsc{apple})(x)]$\label{ex:derivation-explicit-set-partitive-interpretation-a}
	\b. $\llbracket \text{PL}\rrbracket = \lambda P : \cnst{pmssc}(P)\ \lambda x[{}^+P(x)]$\label{ex:derivation-explicit-set-partitive-interpretation-b}
	\b. Function Application: presupposition satisfied\\
	$\llbracket \text{PL Apfel}\rrbracket = \lambda x[{}^+\llbracket \text{Apfel}\rrbracket(x)] = \\
	= \lambda x\big[{}^+\big(\lambda y[\cnst{mssc}(\textsc{apple})(y)]\big)(x)\big]$\label{ex:derivation-explicit-set-partitive-interpretation-c}
	\b. $\llbracket \text{DEF}\rrbracket = \lambda P [\cnst{max}(P)]$\label{ex:derivation-explicit-set-partitive-interpretation-d}
	\b. Function Application\\
	$\llbracket \text{DEF [PL Apfel]}\rrbracket = \cnst{max}(\llbracket \text{PL Apfel}\rrbracket) = \\
	= \cnst{max}\Big(\lambda x\big[{}^+\big(\lambda y[\cnst{mssc}(\textsc{apple})(y)]\big)(x)\big]\Big)$\label{ex:derivation-explicit-set-partitive-interpretation-e}
	\b. $\llbracket \text{Teil}\rrbracket = \lambda y \lambda x [x \sqsubset y]$\label{ex:derivation-explicit-set-partitive-interpretation-f}
	\b. Function Application\\
	$\llbracket \text{Teil [DEF [PL Apfel]]}\rrbracket = \lambda x \big[x \sqsubset \llbracket \text{DEF [PL Apfel]}\rrbracket\big] = 
	\\= \lambda x \Big[x \sqsubset \cnst{max}\Big(\lambda y\big[{}^+\big(\lambda z[\cnst{mssc}(\textsc{apple})(z)]\big)(y)\big]\Big)\Big]$\label{ex:derivation-explicit-set-partitive-interpretation-g}

	As in \ref{ex:derivation-explicit-entity-partitive-interpretation}, the noun in \ref{ex:derivation-explicit-set-partitive-interpretation-a} denotes a set of individuals that are \cnst{mssc} relative to the property \textsc{apple}. Since the plural NP in \ref{ex:derivation-explicit-set-partitive} is semantically interpreted, the plural marker contributes the meaning in \ref{ex:derivation-explicit-set-partitive-interpretation-b}. Specifically, it takes the predicate \textsc{apfel} and applies the strict pluralization operation ${}^+$. As a result, the extension of \ref{ex:derivation-explicit-set-partitive-interpretation-c} consists of sums of apples. Next, the definite article yields the maximal sum plurality from the set which will serve as the argument of \textit{Teil}, see \ref{ex:derivation-explicit-set-partitive-interpretation-d}--\ref{ex:derivation-explicit-set-partitive-interpretation-f}. Finally, after the entity variable is saturated as in \ref{ex:derivation-explicit-set-partitive-interpretation-g}, the explicit set partitive denotes a set of overlapping parts of the largest plurality of apples in a given context. Topology-sensitive transitivity defined in \ref{ex:set-partitive-contraint} ensures that this set consists only of parts that are either \cnst{mssc} entities or sums thereof, i.e., material proper parts of particular individuals are not included. This is a welcome result.

\subsection{Count explicit partitives}\label{sec:count-explicit-partitives}

	The last type of explicit partitive construction to be considered here concerns the significantly more complex example in \ref{ex:derivation-count-explicit-partitive}.  
	
	\ex. Count explicit partitive\label{ex:derivation-count-explicit-partitive}
	\bg.[] zwei Teile des Apfels\\
	two parts the\textsc{.gen} apple\textsc{.gen}\\
	`two parts of the apple'

In general, I assume the  structure for count explicit partitives along the lines in \figref{fig:derivation-count-explicit-partitive-tree}. Notice that in accordance with the semantics of cardinals proposed in \sectref{sec:cardinals}, the numeral root and CL$_\#$ form a constituent. Furthermore, unlike in the explicit partitive examples in \figref{fig:derivation-explicit-entity-partitive-tree} and \ref{fig:derivation-explicit-set-partitive-tree} in order to ensure that counting is possible, I postulate that there is a special IND node present in the tree representing semantic composition, see \sectref{sec:individuation}. In the discussed German example, IND has no formal exponent. However, I argue that the semantic evidence implies that though null it is interpreted. In particular, depending on the context, explicit entity partitives can designate either contiguous or discontiguous parts. On the other hand, count explicit partitives refer only to pluralities of integrated parts. Thus, the presence of the numeral licenses the obligatory occurrence of IND. This of course is in accordance with the counting principles discussed in Chapter \ref{ch:conceptual-background} and requires a formal account. The exact step-by-step derivation is given in \ref{ex:derivation-count-explicit-partitive-interpretation}.

\begin{figure}
    \qtreecenterfalse\centering
    \Tree[.$\langle e,t\rangle$ [.$\langle\langle e,t\rangle,\langle e,t\rangle\rangle$ {$n$\\$\sqrt{\textit{zw}}$} {$\langle n, \langle\langle e,t\rangle,\langle e,t\rangle\rangle\rangle$\\\text{CL$_\#$}} ] [.$\langle e,t\rangle$ {$\langle\langle e,t\rangle,\langle e,t\rangle\rangle$\\\text{IND}} [.$\langle e,t\rangle$ {${\langle e,\langle e,t\rangle\rangle}$\\\textit{Teil}\\`part'} [.$e$ {$\langle\langle e,t\rangle,e\rangle$\\\text{DEF}} {$\langle e,t\rangle$\\\textit{Apfel}\\`apple'} ] ] ] ]
    \caption{Structure of \ref{ex:derivation-count-explicit-partitive}}
    \label{fig:derivation-count-explicit-partitive-tree}
\end{figure}

	\ex. Interpretation of \ref{ex:derivation-count-explicit-partitive}\label{ex:derivation-count-explicit-partitive-interpretation}
	\a. $\llbracket \text{Apfel}\rrbracket = \lambda x [\cnst{mssc}(\textsc{apple})(x)]$\label{ex:derivation-count-explicit-partitive-interpretation-a}
	\b. $\llbracket \text{DEF}\rrbracket = \lambda P [\cnst{max}(P)]$\label{ex:derivation-count-explicit-partitive-interpretation-b}
	\b. Function Application\\
	$\llbracket \text{DEF Apfel}\rrbracket = \cnst{max}(\llbracket \text{Apfel}\rrbracket) = \cnst{max}\big(\lambda x [\cnst{mssc}(\textsc{apple})(x)]\big)$\label{ex:derivation-count-explicit-partitive-interpretation-c}
	\b. $\llbracket \text{Teil}\rrbracket = \lambda y \lambda x [x \sqsubset y]$\label{ex:derivation-count-explicit-partitive-interpretation-d}
	\b. Function Application\\
	$\llbracket \text{Teil [DEF Apfel]}\rrbracket = \lambda x [x \sqsubset \llbracket \text{DEF Apfel}\rrbracket] =\\
	=\lambda x \big[x \sqsubset \cnst{max}\big(\lambda y [\cnst{mssc}(\textsc{apple})(y)]\big)\big]$\label{ex:derivation-count-explicit-partitive-interpretation-e}
	\b. $\llbracket \text{IND}\rrbracket = \lambda P \lambda x[\cnst{mssc}\big(\pi(P)\big)(x)]$\label{ex:derivation-count-explicit-partitive-interpretation-f}
	\b. Function Application\\
	$\llbracket \text{IND [Teil [DEF Apfel]]}\rrbracket = \\
    = \lambda x\big[\cnst{mssc}\big(\pi(\llbracket \text{Teil [DEF Apfel]}\rrbracket)\big)(x)\big] = \\
	= \lambda x\Big[\cnst{mssc}\Big(\pi\big(\lambda z\big[z \sqsubset \cnst{max}\big(\lambda y [\cnst{mssc}(\textsc{apple})(y)]\big)(z)\big]\big)\Big)(x)\Big]$\label{ex:derivation-count-explicit-partitive-interpretation-g}
	\b. $\llbracket \text{$\sqrt{\text{zw}}$}\rrbracket = 2$\label{ex:derivation-count-explicit-partitive-interpretation-h}
	\b. $\llbracket \text{CL$_\#$}\rrbracket = \lambda n \lambda P : \cnst{pmssc}(P)\ \lambda x [\text{*}P(x) \wedge \#(P)(x) = n]$\label{ex:derivation-count-explicit-partitive-interpretation-i}
	\b. Function Application\\
	$\llbracket \text{$\sqrt{\text{zw}}$ CL$_\#$}\rrbracket = \lambda P : \cnst{pmssc}(P)\ \lambda x [\text{*}P(x) \wedge \#(P)(x) = \llbracket \text{$\sqrt{\text{zw}}$}\rrbracket] = \\
    = \lambda P : \cnst{pmssc}(P)\ \lambda x [\text{*}P(x) \wedge \#(P)(x) = 2]$\label{ex:derivation-count-explicit-partitive-interpretation-j}
	\b. Function Application: presupposition satisfied \\
	$\llbracket \text{[$\sqrt{\text{zw}}$ CL$_\#$] [IND [Teil [DEF Apfel]]]}\rrbracket = \\ 
    = \lambda x \big[\text{*}\llbracket \text{IND [Teil [DEF Apfel]]}\rrbracket(x) \\ \wedge
	\#(\llbracket \text{IND [Teil [DEF Apfel]]}\rrbracket)(x) = 2\big] = \\
	= \lambda x \Big[\text{*}\Big(\lambda w \big[\cnst{mssc}\Big(\pi\big(\lambda z\big[z \sqsubset \cnst{max}\big(\lambda y [\cnst{mssc}(\textsc{apple})(y)]\big)(z)\big]\big)\Big)\big](w)\Big) \\ (x) \wedge  \#\Big(\lambda w\big[\cnst{mssc}\big(\pi\big(\lambda z[z \sqsubset \cnst{max}\big(\lambda y [\cnst{mssc}(\textsc{apple})(y)]\big)(z)]\big)\big)\big](w)\Big)\\(x) = 2\Big]$\label{ex:derivation-count-explicit-partitive-interpretation-k}

	The steps in  \ref{ex:derivation-count-explicit-partitive-interpretation-a}--\ref{ex:derivation-count-explicit-partitive-interpretation-e} are exactly the same as in the case of the explicit entity partitive derived in \ref{ex:derivation-explicit-entity-partitive-interpretation}, i.e., starting from the predicate of \cnst{mssc} individuals we end up with the node denoting a set of overlapping continuous and discontinuous parts of a contextually unique apple. Given the counting principles of non-overlap, integrity, and maximality, that expression cannot serve as the domain of quantification since its extension resembles those of mass terms. In particular, it includes both scattered entities and entities that are not disjoint. In Chapter \ref{ch:conceptual-background}, I argued that the deeply rooted mechanism of counting is based upon an algorithm that establishes a one-to-one correspondence with numbers only for discrete objects, i.e., non-overlapping integrated entities in their mereological maximality, and that the same applies to subatomic quantification. Therefore, if parts are to be counted, it needs to be ensured that they constitute entities of that sort. This is what the individuating element IND introduced in \ref{ex:derivation-count-explicit-partitive-interpretation-f} is for. IND combines with the set of overlapping parts and partitions it in such a way that the new set consists only of disjoint parts. However, non-overlap itself does not guarantee that parts are integrated. This is why IND restricts the denotation of \ref{ex:derivation-count-explicit-partitive-interpretation-e} even further by applying the \cnst{mssc} constraint. As a result, the expression in \ref{ex:derivation-count-explicit-partitive-interpretation-g} denotes a set of objects that are \cnst{mssc} with respect to the property of being a disjoint proper part of a contextually unique apple. In other words, \ref{ex:derivation-count-explicit-partitive-interpretation-g} refers to a set of integrated parts that can be put in a one-to-one-correspondence with numbers.
	
	As proposed in \sectref{sec:cardinals}, 
    the cardinal numeral \textit{zwei} is a complex expression consisting of a numeral root simply naming the integer 2, see \ref{ex:derivation-count-explicit-partitive-interpretation-h}, which serves as an argument for the classifier element CL$_\#$, see \ref{ex:derivation-count-explicit-partitive-interpretation-i}. As indicated in \ref{ex:derivation-count-explicit-partitive-interpretation-j}, the whole cardinal is a predicate modifier that for a predicate of \cnst{mssc} entities returns a set of two-element sums of such individuals. This is ensured by the following elements. The individuation presupposition restricts the predicate arguments to those that denote only \cnst{mssc} entities. On the other hand, the first conjunct of the assertion of CL$_\#$ pluralizes the predicate by means of the classical * operator, and thus provides sums for the $\#(P)$ measure function which yields the number of individuals that are \cnst{mssc} relative to the property the whole cardinal modifies. After the predicate variable gets saturated by the entity partitive individuated by IND, the resulting count explicit partitive in \ref{ex:derivation-count-explicit-partitive-interpretation-k} returns a set of pairs of non-overlapping integrated parts of a contextually unique apple. And that is exactly what we want.
	
	Though the computation in \ref{ex:derivation-count-explicit-partitive-interpretation} gets quite complex, I believe that the underlying mechanism deriving count explicit partitives is quite simple. Its great advantage is that it allows us to account for the semantic subtleties concerning topological issues highlighted by the novel data I presented in Chapters \ref{ch:partitives-and-part-whole-structures} and \ref{ch:exploring-topological-sensitivity} via the interplay of the semantics of partitives, the individuating element, and cardinals. In the following sections, I will go in detail through the proposed account for the contrast between Polish topology-neutral and topology-sensitive half-words discussed in  \sectref{sec:partitive-words}.
	
	\subsection{Topology-neutral proportional partitives}\label{sec:topology-neutral-proportional-partitives}
	
	Let us now turn to Polish proportional partitives. First, I will discuss the step-by-step derivation of the least complex case, i.e., entity partitives involving the topology-neutral half-word \textit{połowa} exemplified by the phrase in \ref{ex:topology-neutral-proportional-entity-partitive}. 
	
	\ex. Topology-neutral proportional entity partitive\label{ex:topology-neutral-proportional-entity-partitive}
	\bg.[] połowa jabłka\\
	half$_2$ apple\textsc{.gen}\\
	`half of the apple'

In general, such expressions are very similar to German explicit entity partitives such as those discussed in \sectref{sec:explicit-entity-partitives}. The semantic tree corresponding to \ref{ex:topology-neutral-proportional-entity-partitive} is provided in \figref{fig:topology-neutral-proportional-entity-partitive-tree} and \ref{ex:topology-neutral-proportional-entity-partitive-interpretation} demonstrates how the structure is interpreted.\footnote{As in German explicit partitives, I ignore case marking at the level of semantic composition.}

\begin{figure}
    \qtreecenterfalse\centering
    \Tree[.$\langle e,t\rangle$ {${\langle e,\langle e,t\rangle\rangle}$\\\textit{połowa}\\`half'} [.$e$ {$\langle\langle e,t\rangle,e\rangle$\\\text{DEF}} {$\langle e,t\rangle$\\\textit{jabłko}\\`apple'} ] ]
    \caption{Structure of \ref{ex:topology-neutral-proportional-entity-partitive}}
    \label{fig:topology-neutral-proportional-entity-partitive-tree}
\end{figure}
	
	\ex. Interpretation of \ref{ex:topology-neutral-proportional-entity-partitive} \label{ex:topology-neutral-proportional-entity-partitive-interpretation}
	\a. $\llbracket \text{jabłko}\rrbracket = \lambda x [\cnst{mssc}(\textsc{apple})(x)]$\label{ex:topology-neutral-proportional-entity-partitive-interpretation-a}
	\b. $\llbracket \text{DEF}\rrbracket = \lambda P [\cnst{max}(P)]$\label{ex:topology-neutral-proportional-entity-partitive-interpretation-b}
	\b. Function Application\\
	$\llbracket \text{DEF jabłko}\rrbracket = \cnst{max}(\llbracket \text{jabłko}\rrbracket) = \cnst{max}\big(\lambda x [\cnst{mssc}(\textsc{apple})(x)]\big)$\label{ex:topology-neutral-proportional-entity-partitive-interpretation-c}
	\b. $\llbracket \text{połowa}\rrbracket = \lambda y \lambda x [x \sqsubset y \wedge \mu(x) \approx \mu(y) \times 0.5]$\label{ex:topology-neutral-proportional-entity-partitive-interpretation-d}
	\b. Function Application\\
	$\llbracket \text{połowa [DEF jabłko]}\rrbracket = \lambda x [x \sqsubset \llbracket \text{DEF jabłko}\rrbracket \\
	\wedge \mu(x) \approx \mu(\llbracket \text{DEF jabłko}\rrbracket) \times 0.5] = \\
	= \lambda x \Big[x \sqsubset \cnst{max}\big(\lambda y [\cnst{mssc}(\textsc{apple})(y)]\big) \\
	\wedge \mu(x) \approx \mu\Big(\cnst{max}\big(\lambda y [\cnst{mssc}(\textsc{apple})(y)]\big)\Big) \times 0.5\Big]$\label{ex:topology-neutral-proportional-entity-partitive-interpretation-e}

	Analogously to German \textit{Apfel}, the predicate \textit{jabłko} in \ref{ex:topology-neutral-proportional-entity-partitive-interpretation-a} is true of entities that are \cnst{mssc} with respect to the property \textsc{apple}. Though Polish lacks articles, as indicated in \sectref{sec:maximization} 
    I assume the silent definite element DEF specified in \ref{ex:topology-neutral-proportional-entity-partitive-interpretation-b} which applies the maximization operator \cnst{max} to the set denoted by \textit{jabłko}. If it is a singleton, then the definite DP refers to the unique individual from that set, see \ref{ex:topology-neutral-proportional-entity-partitive-interpretation-c}. The denotation of the half-word \textit{połowa} is very similar to that of a part-word with the exception that it indicates how big the parts it denotes are, see \ref{ex:topology-neutral-proportional-entity-partitive-interpretation-d}. It takes the \cnst{mssc} entity denoted by the definite DP as it first argument, and the resulting proportional partitive denotes a set of parts of a contextually unique apple. The mechanism of contextual conditioning accounts for the fact that the generalized measure function $\mu$ measures its argument in terms of volume. Thus, each part in the denotation of the whole proportional partitive constitutes approximately 50\% of the volume of that apple. Notice that among the members of the set there are continuous as well as discontinuous halves, many of which overlap. Such a denotation is incompatible with cardinal numerals. In order to deliver countable entities, the individuating element IND needs to be applied similarly as in count explicit partitives discussed in  \sectref{sec:count-explicit-partitives}.
	
	Since the half-word \textit{połowa} is topologically neutral, there are no restrictions concerning its distribution. Hence, it can also combine with plurals as in \ref{ex:topology-neutral-proportional-set-partitive}. 

	\ex. Topology-neutral proportional set partitive\label{ex:topology-neutral-proportional-set-partitive}
	\bg.[] połowa jabłek\\
	half$_2$ apples\textsc{.gen}\\
	`half of the apples'

I assume the semantic structure of the phrase in \figref{fig:topology-neutral-proportional-set-partitive-tree} and the interpretation in \ref{ex:topology-neutral-proportional-set-partitive-interpretation}.

\begin{figure}
    \qtreecenterfalse\centering
    \Tree[.$\langle e,t\rangle$ {${\langle e,\langle e,t\rangle\rangle}$\\\textit{połowa}\\`half'} [.$e$ {$\langle\langle e,t\rangle,e\rangle$\\\text{DEF}} [.$\langle e,t\rangle$ {$\langle\langle e,t\rangle,\langle e,t\rangle\rangle$\\\text{PL}} {$\langle e,t\rangle$\\\textit{jabłko}\\`apple'} ] ] ]
    \caption{Structure of \ref{ex:topology-neutral-proportional-set-partitive}}
    \label{fig:topology-neutral-proportional-set-partitive-tree}
\end{figure}
    
	\ex. Interpretation of \ref{ex:topology-neutral-proportional-set-partitive}\label{ex:topology-neutral-proportional-set-partitive-interpretation}
	\a. $\llbracket \text{jabłko}\rrbracket = \lambda x [\cnst{mssc}(\textsc{apple})(x)]$\label{ex:topology-neutral-proportional-set-partitive-interpretation-a}
	\b. $\llbracket \text{PL}\rrbracket = \lambda P : \cnst{pmssc}(P)\ \lambda x[{}^+P(x)]$\label{ex:topology-neutral-proportional-set-partitive-interpretation-b}
	\b. Function Application: presupposition satisfied\\
	$\llbracket \text{PL jabłko}\rrbracket = \lambda x[{}^+\llbracket \text{jabłko}\rrbracket(x)] =\\
    = \lambda x\big[{}^+\big(\lambda y[\cnst{mssc}(\textsc{apple})(y)]\big)(x)\big]$\label{ex:topology-neutral-proportional-set-partitive-interpretation-c}
	\b. $\llbracket \text{DEF}\rrbracket = \lambda P [\cnst{max}(P)]$\label{ex:topology-neutral-proportional-set-partitive-interpretation-d}
	\b. Function Application\\
	$\llbracket \text{DEF [PL jabłko]}\rrbracket = \cnst{max}(\llbracket \text{PL jabłko}\rrbracket) = \\
    = \cnst{max}\Big(\lambda x\big[{}^+\big(\lambda y[\cnst{mssc}(\textsc{apple})(y)]\big)(x)\big]\Big)$
	\b. $\llbracket \text{połowa}\rrbracket = \lambda y \lambda x [x \sqsubset y \wedge \mu(x) \approx \mu(y) \times 0.5]$\label{ex:topology-neutral-proportional-set-partitive-interpretation-e}
	\b. Function Application\\
	$\llbracket \text{połowa [DEF [PL jabłko]]}\rrbracket = \lambda x [x \sqsubset \llbracket \text{DEF [PL jabłko]}\rrbracket \\
	\wedge \mu(x) \approx \mu(\llbracket \text{DEF [PL jabłko]}\rrbracket) \times 0.5] =\\
	= \lambda x \Big[x \sqsubset \cnst{max}\Big(\lambda y\big[{}^+\big(\lambda z[\cnst{mssc}(\textsc{apple})(z)]\big)(y)\big]\Big) \\
	\wedge \mu(x) \approx \mu\Big(\cnst{max}\big(\lambda x\big[{}^+\big(\lambda y[\cnst{mssc}(\textsc{apple})(y)]\big)(x)\big]\big)\Big) \times~0.5\Big]$\label{ex:topology-neutral-proportional-set-partitive-interpretation-f}

	The sole difference between topology-neutral proportional entity and set partitives is that in the case of the latter the noun is pluralized. The plural marker in \ref{ex:topology-neutral-proportional-set-partitive-interpretation-b} applies to the predicate of \cnst{mssc} individuals in \ref{ex:topology-neutral-proportional-set-partitive-interpretation-a} and yields a set consisting of sums formed from those singularities. Subsequently, given topology-sensitive transitivity and the mechanism of contextual conditioning the measure function $\mu$ in \ref{ex:topology-neutral-proportional-set-partitive-interpretation-f} is interpreted as $\#$, i.e., counts the number of \cnst{mssc} parts of its arguments. As a result, the whole construction denotes plural entities whose cardinality corresponds to approximately 50\% of the total number of individuals of the maximal sum of apples. 
	
	Despite the straightforward differences resulting from the distinct semantics of part- and half-words, the denotations of topology-neutral explicit and proportional partitives are very much alike. If not modified by a cardinal, both types of expressions can refer to overlapping continuous and discontinuous parts of a singularity as well as to overlapping parts of a plurality. On the other hand, topology-sensitive partitives are significantly more restricted. In the following section, I focus on demonstrating how exactly.
	
	\subsection{Topology-sensitive proportional partitives}\label{sec:topology-sensitive-proportional-partitives}
	
	An important insight concerning the Polish topology-sensitive half-word \textit{pół} presented in \sectref{sec:polish-half-words} is that it does not appear in set and mass partitives. It is not a weird property of one partitive word since the same behavior is shared, e.g., with the quarter-word \textit{ćwierć}. In any case, I argue that this distributional constraint results from a particular topological restriction imposed on the arguments of \textit{pół}. An example of a topology-sensitive proportional partitive is provided in \ref{ex:topology-sensitive-proportional-entity-partitive}.
	
	\ex. Topology-sensitive proportional entity partitive\label{ex:topology-sensitive-proportional-entity-partitive}
	\bg.[] pół jabłka\\
	half$_1$ apple\textsc{.gen}\\
	`half of the apple'

    The structure of \ref{ex:topology-sensitive-proportional-entity-partitive} I assume at the level of semantic composition is given in \figref{fig:topology-sensitive-proportional-entity-partitive-tree}. The complete derivation is presented in \ref{ex:topology-sensitive-proportional-entity-partitive-interpretation}.
	
	\ex. Interpretation of \ref{ex:topology-sensitive-proportional-entity-partitive} \label{ex:topology-sensitive-proportional-entity-partitive-interpretation}
	\a. $\llbracket \text{jabłko}\rrbracket = \lambda x [\cnst{mssc}(\textsc{apple})(x)]$\label{ex:topology-sensitive-proportional-entity-partitive-interpretation-a}
	\b. $\llbracket \text{DEF}\rrbracket = \lambda P [\cnst{max}(P)]$\label{ex:topology-sensitive-proportional-entity-partitive-interpretation-b}
	\b. Function Application\\
	$\llbracket \text{DEF jabłko}\rrbracket = \cnst{max}(\llbracket \text{jabłko}\rrbracket) = \\
    = \cnst{max}\big(\lambda x [\cnst{mssc}(\textsc{apple})(x)]\big)$\label{ex:topology-sensitive-proportional-entity-partitive-interpretation-c}
	\b. $\llbracket \text{pół}\rrbracket = \lambda y :  \cnst{imssc}(y)\ \lambda x [x \sqsubset y \wedge \mu(x) \approx \mu(y) \times 0.5]$\label{ex:topology-sensitive-proportional-entity-partitive-interpretation-d}
	\b. Function Application: presupposition satisfied\\
	$\llbracket \text{pół [DEF jabłko]}\rrbracket = \lambda x [x \sqsubset \llbracket \text{DEF jabłko}\rrbracket \\
	\wedge \mu(x) \approx \mu(\llbracket \text{DEF jabłko}\rrbracket) \times 0.5] = \\
	= \lambda x \Big[x \sqsubset \cnst{max}\big(\lambda y [\cnst{mssc}(\textsc{apple})(y)]\big) \\
	\wedge \mu(x) \approx \mu\Big(\cnst{max}\big(\lambda y [\cnst{mssc}(\textsc{apple})(y)]\big)\Big) \times 0.5\Big]$\label{ex:topology-sensitive-proportional-entity-partitive-interpretation-e}
	
	\begin{figure}[h]
    \qtreecenterfalse\centering
    \Tree[.$\langle e,t\rangle$ {${\langle e,\langle e,t\rangle\rangle}$\\\textit{pół}\\`half'} [.$e$ {$\langle\langle e,t\rangle,e\rangle$\\\text{DEF}} {$\langle e,t\rangle$\\\textit{jabłko}\\`apple'} ] ]
    \caption{Structure of \ref{ex:topology-sensitive-proportional-entity-partitive}}
    \label{fig:topology-sensitive-proportional-entity-partitive-tree}
\end{figure}

	The derivation in \ref{ex:topology-sensitive-proportional-entity-partitive-interpretation} is almost identical to its twin counterpart in \ref{ex:topology-neutral-proportional-entity-partitive-interpretation}. However, notice that in \ref{ex:topology-sensitive-proportional-entity-partitive-interpretation-d} there is a tiny but crucial difference in the semantics of \textit{pół} as proposed in \ref{ex:polish-topology-sensitive-pol}. Specifically, \textit{pół} incorporates the integrated individual presupposition, which requires the first argument to be an integrated entity. This constrains the distribution of \textit{pół} to count singulars since plurals and mass terms denote arbitrary sums and scattered entities, respectively, i.e., things distinct from \cnst{mssc} objects. However, \textit{pół} does not impose any topological restriction on parts of the relevant individual it yields. The semantics of the whole partitive in \ref{ex:topology-sensitive-proportional-entity-partitive-interpretation-e} thus states that the denoted set includes both continuous and discontinuous halves. However, as we saw in Chapter \ref{ch:exploring-topological-sensitivity} there are also partitive words that are restrictive with respect to the topological properties of entities they deliver.
	
	This brings us finally to the most intriguing case of Polish proportional partitives headed by the topology-sensitive half-word \textit{połówka} such as the one in \ref{ex:topology-sensitive-individuating-proportional-partitive}. 

	\ex. Topology-sensitive individuating proportional partitive\label{ex:topology-sensitive-individuating-proportional-partitive}
	\bg.[] połówka jabłka\\
	half$_3$ apple\textsc{.gen}\\
	`half of the apple'
    
    Though I restrict my focus here to the structure and the interpretation of this example, see \figref{fig:topology-sensitive-individuating-proportional-partitive-tree} and \ref{ex:topology-sensitive-individuating-proportional-partitive-interpretation}, respectively, as we saw in \sectref{sec:more-topology-sensitive-partitive-words} there are also other expressions that behave similarly, e.g., the part-word \textit{cząstka} and quarter-word \textit{ćwiartka}.
    
\begin{figure}
    \qtreecenterfalse\centering
    \Tree[.$\langle e,t\rangle$ {$\langle\langle e,t\rangle,\langle e,t\rangle\rangle$\\\textit{-k-}} [.$\langle e,t\rangle$ {${\langle e,\langle e,t\rangle\rangle}$\\$\sqrt{\textit{pół}}$} [.$e$ {$\langle\langle e,t\rangle,e\rangle$\\\text{DEF}} {$\langle e,t\rangle$\\\textit{jabłko}\\`apple'} ] ] ]
    \caption{Structure of \ref{ex:topology-sensitive-individuating-proportional-partitive}}
    \label{fig:topology-sensitive-individuating-proportional-partitive-tree}
\end{figure}
	
	\ex. Interpretation of \ref{ex:topology-sensitive-individuating-proportional-partitive}\label{ex:topology-sensitive-individuating-proportional-partitive-interpretation}
	\a. $\llbracket \text{jabłko}\rrbracket = \lambda x [\cnst{mssc}(\textsc{apple})(x)]$\label{ex:topology-sensitive-individuating-proportional-partitive-interpretation-a}
	\b. $\llbracket \text{DEF}\rrbracket = \lambda P [\cnst{max}(P)]$\label{ex:topology-sensitive-individuating-proportional-partitive-interpretation-b}
	\b. Function Application\\
	$\llbracket \text{DEF jabłko}\rrbracket = \cnst{max}(\llbracket \text{jabłko}\rrbracket) = \cnst{max}(\lambda x [\cnst{mssc}(\textsc{apple})(x)])$\label{ex:topology-sensitive-individuating-proportional-partitive-interpretation-c}
	\b. $\llbracket \sqrt{\text{pół}}\rrbracket = \lambda y : \cnst{imssc}(y)\ \lambda x [x \sqsubset y \wedge \mu(x) \approx \mu(y) \times 0.5]$\label{ex:topology-sensitive-individuating-proportional-partitive-interpretation-d}
	\b. Function Application: presupposition satisfied\\
	$\llbracket \text{$\sqrt{\text{pół}}$ [DEF jabłko]}\rrbracket = \\
    = \lambda x [x \sqsubset \llbracket \text{DEF jabłko}\rrbracket \wedge \mu(x) \approx \mu(\llbracket \text{DEF jabłko}\rrbracket) \times 0.5] = \\
	\lambda x \Big[x \sqsubset \cnst{max}\big(\lambda y [\cnst{mssc}(\textsc{apple})(y)]\big) \\
	\wedge \mu(x) \approx \mu\Big(\cnst{max}\big(\lambda y [\cnst{mssc}(\textsc{apple})(y)]\big)\Big) \times 0.5\Big]$\label{ex:topology-sensitive-individuating-proportional-partitive-interpretation-e}
	\b. $\llbracket \text{-k-}\rrbracket = \llbracket \text{IND}\rrbracket = \lambda P \lambda x[\cnst{mssc}\big(\pi(P)\big)(x)]$\label{ex:topology-sensitive-individuating-proportional-partitive-interpretation-f}
	\b. Function Application\\
	$\llbracket \text{IND [$\sqrt{\text{pół}}$ [DEF jabłko]]}\rrbracket = \\
    = \lambda x[\cnst{mssc}\big(\pi(\llbracket \text{$\sqrt{\text{pół}}$ [DEF jabłko]}\rrbracket)\big)(x)] = \\
	= \lambda x\Big[\cnst{mssc}\Big(\pi\big(\lambda z \Big[z \sqsubset \cnst{max}\big(\lambda y [\cnst{mssc}(\textsc{apple})(y)]\big) \\
	\wedge \mu(x) \approx \mu\Big(\cnst{max}\big(\lambda y [\cnst{mssc}(\textsc{apple})(y)]\big)\Big) \times 0.5\Big]\big)\Big)(x)\Big]$\label{ex:topology-sensitive-individuating-proportional-partitive-interpretation-g}

	The only difference between \ref{ex:topology-sensitive-proportional-entity-partitive-interpretation} and \ref{ex:topology-sensitive-individuating-proportional-partitive-interpretation} lies in that in the former the half-word \textit{pół} is used whereas the latter involves the derivationally complex \textit{połówka}. Thus, the composition in \ref{ex:topology-sensitive-individuating-proportional-partitive-interpretation-a}--\ref{ex:topology-sensitive-individuating-proportional-partitive-interpretation-c} remains unchanged and I will start the discussion from the step in \ref{ex:topology-sensitive-individuating-proportional-partitive-interpretation-d}. As proposed in \ref{ex:polish-topology-sensitive-polowka}, the morphological complexity of \textit{połówka} translates into its semantic complexity. In particular, I assume that the root $\sqrt\textit{pół}$ is a topologically sensitive expression introducing the integrated individual presupposition, see \ref{ex:topology-sensitive-individuating-proportional-partitive-interpretation-d}, which ensures that that the half-word combines only with \cnst{mssc} individuals. On the other hand, the suffix \textit{-k-} is interpreted as the individuating element IND, see \ref{ex:topology-sensitive-individuating-proportional-partitive-interpretation-e}, i.e., it restricts the denotation of the partitive to non-overlapping continuous parts. As a result, the whole phrase refers to integrated halves of a contextually unique apple that can be subject to counting. 
	
	This concludes the analysis of distinct types of partitive constructions. The relevance of Polish topology-sensitive individuating proportional partitives is that they show that IND can be formally expressed. As demonstrated in  \sectref{sec:cross-linguistic-parallels}, there are more constructions in different natural languages that can be considered as involving formal exponents of IND. Though I will refrain here from further investigation, if the analysis proposed here is on the right track, I expect to find even more such expressions cross-linguistically. In the next section, I will move on to another type of construction involving subatomic quantification, namely multiplier phrases.
	
	\subsection{Multiplier phrases}\label{sec:multiplier-phrases}
	
	In the previous sections, I discussed the composition of different kinds of partitives. The only construction involving numerical expressions discussed so far was the German count explicit partitive in \ref{ex:derivation-count-explicit-partitive} where the cardinal modified an entity partitive headed by the part-word \textit{Teil}. In this section, I will discuss in detail the semantics of constructions involving derived numerical expressions specialized precisely for subatomic quantification, namely multipliers. In order to present the overall picture, I will consider multiplier phrases modified by cardinals exemplified by \ref{ex:multiplier-phrase-modified-by-a-cardinal}. 
	
	\ex. Multiplier phrase modified by a cardinal\label{ex:multiplier-phrase-modified-by-a-cardinal}
	\bg.[] trzy podwójne hamburgery\\
	three double\textsc{.pl} hamburgers\\
	`three double hamburgers'

I posit that the semantic structure underlying such expressions is essentially as in \figref{fig:multiplier-phrase-modified-by-a-cardinal-tree}, whereas in \ref{ex:multiplier-phrase-modified-by-a-cardinal-interpretation} I provide an exact step-by-step derivation of the main example.

\begin{figure}
    \qtreecenterfalse\centering
    \Tree[.$\langle e,t\rangle$ [.$\langle\langle e,t\rangle,\langle e,t\rangle\rangle$ {$n$\\$\sqrt{trz}$} {$\langle n,\langle\langle e,t\rangle,\langle e,t\rangle\rangle\rangle$\\\text{CL$_\#$}} ] [ [.$\langle\langle e,t\rangle,\langle e,t\rangle\rangle$ {$n$\\$\sqrt{dw}$} {$\langle n,\langle\langle e,t\rangle,\langle e,t\rangle\rangle\rangle$\\\text{CL$_\boxplus$}} ] {$\langle e,t\rangle$\\\textit{hamburger}\\`hamburger'} ] ]
    \caption{Structure of \ref{ex:multiplier-phrase-modified-by-a-cardinal}}
    \label{fig:multiplier-phrase-modified-by-a-cardinal-tree}
\end{figure}
	
	\ex. Interpretation  of \ref{ex:multiplier-phrase-modified-by-a-cardinal}\label{ex:multiplier-phrase-modified-by-a-cardinal-interpretation}
	\a. $\llbracket \text{hamburger}\rrbracket = \lambda x [\cnst{mssc}(\textsc{hamburger})(x)]$\label{ex:multiplier-phrase-modified-by-a-cardinal-interpretation-a}
	\b. $\llbracket \sqrt{\text{dw}}\rrbracket = 2$\label{ex:multiplier-phrase-modified-by-a-cardinal-interpretation-b}
	\b. $\llbracket \text{CL$_\boxplus$}\rrbracket = \lambda n \lambda P : \cnst{pmssc}(P)\ \lambda x [P(x) \wedge \boxplus(P)(x) = n]$\label{ex:multiplier-phrase-modified-by-a-cardinal-interpretation-c}
	\b. Function Application\\
	$\llbracket \text{$\sqrt{\text{dw}}$ CL$_\boxplus$}\rrbracket = \lambda P : \cnst{pmssc}(P)\ \lambda x [P(x) \wedge \boxplus(P)(x) = \llbracket \sqrt{\text{dw}}\rrbracket] = \\ 
	= \lambda P : \cnst{pmssc}(P)\ \lambda x [P(x) \wedge \boxplus(P)(x) = 2]$\label{ex:multiplier-phrase-modified-by-a-cardinal-interpretation-d}
	\b. Function Application: presupposition satisfied\\
	$\llbracket \text{[$\sqrt{\text{dw}}$ CL$_\boxplus$] hamburger}\rrbracket = \\
    = \lambda x [\llbracket \text{hamburger}\rrbracket(x) \wedge \boxplus(\llbracket \text{hamburger}\rrbracket)(x) = 2] = 
	\\
	= \lambda x [\cnst{mssc}(\textsc{hamburger})(x) \wedge \boxplus\big(\cnst{mssc}(\textsc{hamburger})\big)(x) = 2]$\label{ex:multiplier-phrase-modified-by-a-cardinal-interpretation-e}
	\b. $\llbracket \text{$\sqrt{\text{trz}}$}\rrbracket = 3$\label{ex:multiplier-phrase-modified-by-a-cardinal-interpretation-f}
	\b. $\llbracket \text{CL$_\#$}\rrbracket = \lambda n \lambda P : \cnst{pmssc}(P)\ \lambda x [\text{*}P(x) \wedge \#(P)(x) = n]$\label{ex:multiplier-phrase-modified-by-a-cardinal-interpretation-g}
	\b. Function Application\\
	$\llbracket \text{$\sqrt{\text{trz}}$ CL$_\#$}\rrbracket = \lambda P : \cnst{pmssc}(P)\ \lambda x [\text{*}P(x) \wedge \#(P)(x) = \llbracket \text{$\sqrt{\textit{trz}}$}\rrbracket] = \\
	= \lambda P : \cnst{pmssc}(P)\ \lambda x [\text{*}P(x) \wedge \#(P)(x) = 3]$\label{ex:multiplier-phrase-modified-by-a-cardinal-interpretation-h}
	\b. Function Application: presupposition satisfied \\
	$\llbracket \text{[$\sqrt{\text{trz}}$ CL$_\#$] [[$\sqrt{\text{dw}}$ CL$_\boxplus$] hamburger]}\rrbracket = \\
    = \lambda x \big[\text{*}\llbracket \text{[$\sqrt{\text{dw}}$ CL$_\boxplus$] hamburger}\rrbracket(x) \\
	\wedge \#(\llbracket \text{[$\sqrt{\text{dw}}$ CL$_\boxplus$] hamburger}\rrbracket)(x)~=~3\big] = \\
	\lambda x \Big[\text{*}\Big(\lambda y [\cnst{mssc}(\textsc{hamburger})(y) \wedge \boxplus\big(\cnst{mssc}(\textsc{hamburger})\big)(y)~=~2]\Big)\\(x) 
	\wedge \#\Big(\lambda y [\cnst{mssc}(\textsc{hamburger})(y) \wedge \boxplus\big(\cnst{mssc}(\textsc{hamburger})\big)(y) = 2]\Big)\\(x) =~3\Big]$\label{ex:multiplier-phrase-modified-by-a-cardinal-interpretation-i}

	In analogy to the previous cases, the noun \textit{hamburger} is a predicate of type $\langle e,t\rangle$ denoting a set of entities that are \cnst{mssc} with respect to the property \textsc{hamburger}, i.e., singular hamburgers. On the other hand, as suggested by morphology the multiplier in \ref{ex:multiplier-phrase-modified-by-a-cardinal-interpretation-d} consists of the classifier element CL$_\boxplus$ of type $\langle n,\langle\langle e,t\rangle,\langle e,t\rangle\rangle\rangle$, whose number argument, see \ref{ex:multiplier-phrase-modified-by-a-cardinal-interpretation-c}, is saturated by the number 2 referred to by the numeral root in \ref{ex:multiplier-phrase-modified-by-a-cardinal-interpretation-b}. Due to the individuation presupposition in the semantics of CL$_\boxplus$ the resulting numerical expression is a predicate modifier selecting only predicates of \cnst{mssc} individuals. Since the noun \textit{hamburger} satisfies this requirement, it can serve as the argument for the multiplier. After the two combine in \ref{ex:multiplier-phrase-modified-by-a-cardinal-interpretation-e}, the resulting multiplier phrase, i.e., a function from entities to truth values, denotes a set of singular hamburgers such that each includes two patties. This meaning stems from the fact that the $\boxplus(P)$ measure function defined in \ref{ex:mf-boxplus(P)} returns 2 as the number of parts that are essential for the property \textsc{hamburger}, and given the relevant extra-linguistic factors such a part happens to be a patty. Notice also that the fact that the first conjunct in \ref{ex:multiplier-phrase-modified-by-a-cardinal-interpretation-e} is not pluralized, unlike in cardinals, results in that the multiplier phrase is a predicate of \cnst{mssc} entities, i.e., it denotes singular double hamburgers. As such it can be modified by the cardinal numeral.
	
	Given the compatible expression, the cardinal in \ref{ex:multiplier-phrase-modified-by-a-cardinal-interpretation-h} is ready for the saturation of the predicate variable. It itself is a complex expression of type $\langle\langle e,t\rangle,\langle e,t\rangle\rangle$ consisting of the numeral root in \ref{ex:multiplier-phrase-modified-by-a-cardinal-interpretation-f}, whose denotation, i.e., the number 3, serves as the argument for the classifier element CL$_\#$. After the cardinal composes with the multiplier phrase in \ref{ex:multiplier-phrase-modified-by-a-cardinal-interpretation-i}, what we obtain is a set of triples of hamburgers with two essential parts each. Counting is done by the measure function $\#(P)$ which puts \cnst{mssc} individuals in a one-to-one correspondence with numbers and yields the integer corresponding to the total number of objects in a plurality delivered by the * operator introduced in the first conjunct. The result is exactly what we wanted to derive, i.e., a set of pluralities consisting of three double hamburgers.
	
	This concludes the analysis of some of the expressions involving subatomic quantification. In this study, I focused on several different types of partitives as well as multiplier phrases. I proposed that though both types of constructions quantify over parts of entities, they employ different means to achieve the goal. While partitives employ the proper parthood relation $\sqsubset$ and in more complicated cases also the individuating element IND and cardinals, multipliers utilize a special measure function $\boxplus(P)$ dedicated to counting essential parts of integrated wholes. In the next section, I will briefly suggest how the analysis I proposed could be extended to cover more peculiar examples such as Italian partitives involving irregular plurals.
	
	\section{Possible extensions}\label{sec:possible-extensions}
	
	The last part of this chapter concerns a brief informal discussion of possible extensions of the system developed here to other types of constructions in different languages. Though mereotopological structures proposed in this book could interact with other semantic phenomena making the empirical landscape even more complex, I believe that the perspective argued for in this chapter offers an inspiring perspective for the study of such possible interactions. 
	
	\subsection{Topology-sensitive partitives in other languages}\label{sec:topology-sensitive-partitives-in-other-languages}
	
	In principle, a number of distinct kinds of topology-sensitive proportional partitives discussed in  \sectref{sec:cross-linguistic-parallels} could be accounted for by incorporating the meaning of the individuating element IND into the semantics of some morpheme. For instance, if the approach proposed here is on the right track and there are no additional facts I have not reported here concerning phrases such as \ref{ex:german-topology-sensitive-proportional-partitive}, a straightforward application of the analysis appears to be plausible. Recall that the phrase is reported to felicitously refer only to a half of the flag that constitutes a continuous part of the whole, see \sectref{sec:cross-linguistic-parallels}. It seems that the morpheme \textit{eine} is what encodes IND, which accounts for the reported results of the flag test.
	
	\ex. German topology-sensitive proportional partitive\label{ex:german-topology-sensitive-proportional-partitive}
	\bg.[] die eine Hälfte der Fahne\\
	the a/one half the\textsc{.gen} flag\\
	`the half of the flag'
	
    Similar, in principle the account developed here could be extended to English constructions such as \ref{ex:english-topology-sensitive-proportional-partitive}. However, most probably this case would be more challenging since it might be necessary to capture interactions between the partitive word and the determiner, on the one hand, as well as with the preposition \textit{of}, on the other.
    
    \ex. English topology-sensitive proportional partitive\\
    a half of the flag\label{ex:english-topology-sensitive-proportional-partitive}
    
	Furthermore, it might turn out that a similar analysis is appropriate for the Mandarin classifier construction in \ref{ex:mandarin-topology-sensitive-proportional-partitive}. Here, the classifier \textit{mi{\`{a}}n} is arguably responsible for introducing IND.\pagebreak
	
	\ex. Mandarin topology-sensitive proportional partitive\label{ex:mandarin-topology-sensitive-proportional-partitive}
	\bg.[] b{\`{a}}n-mi{\`{a}}n gu{\'{o}} q{\'{i}}\\
	half-\textsc{clf} national flag\\
	`a half of the national flag'
	
	Of course, much more work needs to be done to verify whether what I proposed in fact accounts for the data from German, Mandarin, and other languages displaying a similar pattern. However, I believe that at this stage of the research it is a plausible hypothesis to postulate that what makes topology-sensitive partitives different from their topology-neutral counterparts is the fact that the former involve a formal exponent of IND. 
    
    The next section will be dedicated to the discussion of how the proposed analysis could be applied to yet another type of expression.
	
	\subsection{Italian partitives with irregular plurals}\label{sec:italian-partitives-with-irregular-plurals}
	
	The last case of a possible extension of the developed system to be discussed here concerns count explicit partitives involving irregular plurals in Italian, such as those in \ref{ex:italian-partitive-irregular-plurals}. 
	
	\ex. Italian partitive with irregular plurals\label{ex:italian-partitive-irregular-plurals}
	\bg.[] due   parti delle  mura\\
	two parts of-the wall\textsc{.coll}\\
	`two parts of the walled complex'
	
	In Chapter \ref{ch:partitives-and-part-whole-structures}, a considerable amount of attention was dedicated to the intriguing effect these expressions give rise to. We saw that at least a subset of Italian irregular plurals does not merely encode plurality but also implies a certain spatial configuration that the individual entities making up a sum remain in. Arguably, this kind of behavior can be captured in terms of the topological notion of connection. The Italian data served as a crucial source of evidence for the role of integrity in quantification over parts. In particular, the evidence strongly suggests that entities are countable as long as they are conceptualized as spatially continuous regardless whether they are singularities or pluralities.
	
	Given the arguments in favor of the connected nature of pluralities denoted by the relevant class of Italian irregular plurals, I postulate that it is plausible to model them in terms of clusters \citep[see][p. 144]{grimm2012number}. As discussed in \sectref{sec:other-types-of-connection}, a cluster is a special type of plural individual that consists of a number of topologically arranged objects. It is defined in terms of the \cnst{tc} relation formulated in \ref{ex:transitively-connected-new} in \sectref{sec:other-types-of-connection}, repeated here as \ref{ex:transitively-connected-new2}. Specifically, the definition in \ref{ex:cluster-new} in \sectref{sec:other-types-of-connection}, repeated here as \ref{ex:cluster-new2}, states that an entity is a cluster if it involves a plurality of individuals that are transitively connected, i.e., form a connected sequence.  
	
	\ex. Transitively connected (revised)\\
    $\cnst{tc}(x,y,P,C,Z) \eqdef z_1 = x \wedge z_n = y \wedge \forall i[1 \leq i < n \rightarrow C(z_i,z_{i+1})]\\
    \wedge \forall i[1 \leq i \leq n \rightarrow P(z_i)]$, where $Z = \langle z_1, \dots, z_n\rangle$\\
    (Entities $x$ and $y$ are transitively connected relative to a property $P$, a connection relation $C$, and a finite sequence of entities $Z$, when all members of $Z$ satisfy $P$ and $x$ and $y$ are connected through the sequence of $z_i$s in $Z$.)\label{ex:transitively-connected-new2}
	
	\ex. Cluster (revised)\\
    $\cnst{clstr}(x,P,C) \eqdef \exists Z[x = \bigsqcup Z \wedge \forall z \forall z' \in Z \exists Y \subseteq Z[\cnst{tc}(z,z',P,C,Y)]]$\\
    (An entity $x$ is a cluster relative to a property $P$ and a connection relation $C$ iff $x$ is a sum of entities falling under the same property which are all transitively connected relative to $Y$ which is a subset of a sequence $Z$ under the same property and connection relation.)\label{ex:cluster-new2}
    
    Intuitively, clusters are entities that are conceptualized as bundled pluralities whose minimal units are cognitively not salient enough or insignificant to be considered objects in their own rights. For instance, a pile of rice or a heap of gravel constitute clusters since each minimal unit, i.e., a grain or a pebble, in a pile or a heap is connected to another minimal unit. Since \cnst{tc} is relativized to a connection, clusters can involve bundled pluralities of entities that are externally or even proximately connected.\footnote{Those connection can be formally captured by the notions of \cnst{ec} and \cnst{pc} introduced in \ref{ex:externally-connected} and \ref{ex:proximately-connected}, respectively.} In the case of Italian irregular plurals, the former would correspond, e.g., to the referents of \textit{mura} `walls (in a complex)', whereas the latter might be useful to model the denotation of, e.g., \textit{ossa} `bones (in a skeleton)'.
    
	If the proposed analysis of Italian irregular plurals in terms of clusters is correct, one would probably need to elaborate on the meaning of cardinal numerals. Notice that clusters can consist of parts that are \cnst{mssc} entities, e.g., individual bones, as well as sequences of individuals, e.g., pluralities of transitively connected bones. This might require to develop a more general device able to capture more fine-grained shades of integrity than the current formulation of the individuation presupposition. Though such an enterprise lies beyond the scope of this study and I leave it for future research, I believe that the account proposed here provides an inspiring background for such an attempt.
	
	\section{Summary}\label{sec:summary-ch7}
	
	In this chapter, I presented a formal account for subatomic quantification in natural language grounded in the conceptual framework described in Chapter \ref{ch:conceptual-background}. I rejected atomicity as a useful concept for that purpose and instead proposed an analysis based on notions developed within mereotopology, i.e., the theory of wholes extending standard mereology with topological distinctions, introduced in Chapter \ref{ch:theory-of-parts-and-wholes}. The mereotopological account for nominal semantics allows us to get rid of atomicity as an undesirable notion as well as to distinguish between singularities and pluralities in terms of the distinction between part-whole structures that either involve or do not involve the topological notion of connectedness. In particular, I adopted a view on which singular count nouns denote sets of integrated objects modeled in terms of \cnst{mssc} individuals that can be pluralized in order to yield sets of pluralities of such entities. On the other hand, I postulated a compositional semantics for numerical expressions such as cardinals and multipliers that involves numeral roots treated here as names of numbers as well as classifiers that turn expressions of type $n$ into predicate modifiers. In particular, I proposed that those classifier elements employ measure functions, i.e., operations that relate entities with integers. While cardinals utilize a measure function that simply counts \cnst{mssc} entities, multipliers involve quantification over essential parts of an individual. Compatibility of the numerical expressions in question is ensured by the individuation presupposition that requires the nodes cardinals and multipliers combine with to be predicates of \cnst{mssc} objects.
	
	Another component of the analysis regarded the semantics of partitive words. Given the advantages of mereotopology, it was possible to model subatomic quantification without postulating sorted domain with different parthood relations defined over them. Instead, particular partitive words were treated as expressions employing the unified parthood relation. Specifically, what they do is that they select an entity and return a set of its parts. Different types of partitive words differ with respect to the nature of those parts. For instance, part-words denote sets of any proper parts, whereas half-words refer to parts that constitute approximately 50\% of the cardinality of a plurality or the volume of an \cnst{mssc} individual or a scattered entity such as substance. On the other hand, topology-neutral partitive words denote all kinds of overlapping continuous as well as discontinuous parts, and thus are compatible with singular count nouns, mass terms, and plurals, whereas denotations of topology-sensitive partitive words are more restricted. For instance, the Polish half-word \textit{pół} introduces the integrated individual presupposition, which imposes a constraint that it can only select \cnst{mssc} entities. This fact explains why it does not combine with mass and plural complements. On the other hand, morphologically complex \textit{połówka} is a compositional expression involving the suffix \textit{-k-}, which I argue is a formal exponent of the individuating element that partitions the set of parts so that overlapping halves are excluded as well as removes those parts that are discontinuous. As a result, the extension of a partitive headed by \textit{połówka} involves only a set of disjoint integrated halves that could serve as an argument for cardinals. 
	
	Though the individuating element is formally expressed in the morphology of some Polish partitive words, it can also be silent. I argued that that is the case in, e.g., German count explicit partitives since, given the counting principles postulated in Chapter \ref{ch:conceptual-background}, one can only count non-overlapping continuous parts as one. The chapter was concluded by considerations how the proposed system could be extended to some of the other expressions discussed in this study including Italian partitives involving irregular plurals, which could be analyzed as clusters, i.e., bundled pluralities of transitively connected entities. However, the precise implementation of the suggested ideas is left for future research.
